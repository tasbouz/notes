%! suppress = EscapeUnderscore
\section{Interest Rates}

\bd[Interest]
\textbf{Interest} is the cost of borrowing money, or equivalently, the income from lending money.
\ed

Interest is the opportunity cost of capital, or in other words the compensation for differing consumption over time.
In simple words, it is the price paid for the use of someone else's money over time.
\v

This very concept of time is what makes the interest a factor that connects the present to the future, thus creating
the concept of the ``time value of money''.

\bd[Time Value Of Money]
The \textbf{time value of money} is the idea that money available at the present time is worth more than the same
amount in the future due to its potential earning capacity through interest.
\ed

An interest is usually expressed through an ``interest rate'' on the initial amount of money, called the ``principal''.

\bd[Principal]
The \textbf{principal} is the initial amount of money that is invested or borrowed.
\ed

\bd[Interest Rate]
The \textbf{interest rate} $r$ is the amount of interest due per period, as a proportion of the principal.
\ed

Based on the period of time the interest rate is calculated, we can have different types of interest rates:
\bit
\item \textbf{Annual Interest Rate}: The interest rate is calculated on a yearly basis.
\item \textbf{Semi-Annual Interest Rate}: The interest rate is calculated on a half-yearly basis.
\item \textbf{Quarterly Interest Rate}: The interest rate is calculated on a quarterly basis.
\item \textbf{Monthly Interest Rate}: The interest rate is calculated on a monthly basis.
\item \textbf{Daily Interest Rate}: The interest rate is calculated on a daily basis.
\eit

When no period is specified, the interest rate is assumed to be an annual interest rate, which is the most frequent
one.

\section{Future \& Present Value}

Due to interest rates, money available at the present time is worth more than the same amount in the future. This is
because money deposited in a savings account will earn interest and therefore will be worth more in the future. As a
result of this, money at present carry a ``future value''.

\bd[Future Value]
The \textbf{future value} $FV$ is the value of a current amount of money at a future date given a specified interest
rate.
\ed

Similarly, money to be received in the future is worth less than the same amount today. This is because money received
in the future can be invested from today until the future date and therefore will be worth more in the future. As a
result of this, money in the future carry a ``present value''.

\bd[Present Value]
The \textbf{present value} $PV$ is the current value of a future amount of money given a specified interest rate.
\ed

In the case of present values, the interest rate is used to discount the future amount of money to its present value,
and for this reason sometimes it is called the ``discount rate''. However, it is the same concept as the interest
rate. \v

Using the interest rate one can compute the future value of an amount of money invested today, and the present value
of an amount of money to be received in the future. Before doing so, we need to distinguish between ``simple'' and
``compound'' interest rates.

\subsection{Simple Interest Rate}

\bd[Simple Interest Rate]
\textbf{Simple interest rate} is the interest rate calculated only on the principal.
\ed

The future value $FV$ of a princiapl amount $P$ invested today at a simple interest rate $r$ per period for $t$ periods
is:
\bse
FV = P \cdot (1 + rt)
\ese

Notice that the period in which the interest rate $r$ is calculated must be the same as the period $t$ in which the
future value is calculated.

\be
The future value of a principal of \$1,500,000 with an annual interest rate of 3.8\% after 45 days is:
\bse
FV = 1,500,000 \cdot (1 + \frac{0.038}{360} \cdot 45) = 1,507,125
\ese

The annual interest rate is divided by 360 to convert it to daily interest rate, and then multiplied by the number of
days to calculate the interest for the 45 days. One could equally keep the annual interest rate and divide the number
of days by 360 to convert it to years. The result would be the same.
\ee

Similarly, the present value $PV$ of a principal amount $P$ to be received in $t$ periods from now given a simple
discount rate $t$ is:
\bse
PV = \frac{P}{1 + rt}
\ese

The simple interest rate is the most basic type of interest used in finance. However, it is not the most common one,
and in fact it is rarely used in practice. The reason is that it does not take into account the compounding effect
of interest. For this reason, the concept of the ``compound interest rate'' is introduced.

\subsection{Compound Interest Rate}

\bd[Compound Interest Rate]
\textbf{Compound interest rate} is the interest rate calculated on the principal, and also on the accumulated interest
of previous periods given that the latter is reinvested.
\ed

In essense, the compounding effect is the effect of earning interest on interest, by reinvesting the interest
received. Compound interest is the most common type of interest used in finance. \v

The future value $FV$ of a principal amount $P$ invested today at a compound interest rate $r$ per period, for $t$
periods, with $n$ number of times of the interest being compounded per period is:
\bse
FV = \biggl( \Bigl(P \cdot (1 + \frac{r}{n})^n\Bigr) \cdot (1 + \frac{r}{n})^n\biggr) \cdot \underbrace{\ldots}_{t\text{-times}} = P \cdot (1 + \frac{r}{n})^{nt}
\ese

Compounding can occur on any time interval, including:
\bit
\item \textbf{Annual Compounding}: Compounding once per year: $FV_{\text{annual}} = P \cdot \left(1 + r\right)^{t}$
\item \textbf{Semi-Annual Compounding}: Compounding twice per year: $FV_{\text{semi-annual}} = P \cdot \left(1 + \frac{r}{2}\right)^{2t}$
\item \textbf{Quarterly Compounding}: Compounding four times per year: $FV_{\text{quarterly}} = P \cdot \left(1 + \frac{r}{4}\right)^{4t}$
\item \textbf{Monthly Compounding}: Compounding twelve times per year: $FV_{\text{monthly}} = P \cdot \left(1 + \frac{r}{12}\right)^{12t}$
\item \textbf{Daily Compounding}: Compounding 365 times per year: $FV_{\text{daily}} = P \cdot \left(1 + \frac{r}{365}\right)^{365t}$
\item \textbf{Continuous Compounding}: Compounding infinitely many times per year: $FV_{\text{continuous}} = P \cdot e^{rt}$
\eit

Notice that for the continuous compounding case:
\bse
FV_{\text{continuous}} = \lim_{n \to \infty} \left(P \cdot (1 + \frac{r}{n})^{nt}\right)
\overset{m=\frac{n}{r}}{=} P \cdot \lim_{m \to \infty} \left((1 + \frac{1}{m})^{mrt}\right)
= P \cdot \underbrace{\biggl(\lim_{m \to \infty} (1 + \frac{1}{m})^{m}}_{\text{definition of e}} \biggl)^{rt} = P \cdot e^{rt}
\ese

As the equations indicate, with compound interest the interest for each period is added to the principal at the start
of that period, and the interest for the next period is calculated on the new principal. What this means is that
interest is being earned on both the initial investment, and the interest earned from the previous periods. Another
way to think about the future value given a compound interest is:
\bse
FV = \text{Principal} + \text{Contract Interest} + \text{Compound Interest}
\ese

\be
Given \$10,000 of principal invested for 5 years at a 4\% annual interest rate, let's compare the impact of annual, to
semi-annual, to monthly, to continuous compounding:
\bit
\item $FV_{\text{annual}} = 10.000 \cdot \left(1 + \frac{0.04}{1}\right)^{1 \cdot 5} = 12,166.53 = \underbrace{\$10,000}_{\text{Principal}} + \underbrace{\$2,000}_{\text{Contract Interest}} + \underbrace{\$166.53}_{\text{Compound Interest}}$
\item $FV_{\text{semi-annual}} = 10.000 \cdot \left(1 + \frac{0.04}{2}\right)^{2 \cdot 5} = 12,189.94 = \underbrace{\$10,000}_{\text{Principal}} + \underbrace{\$2,000}_{\text{Contract Interest}} + \underbrace{\$189.94}_{\text{Compound Interest}}$
\item $FV_{\text{monthly}} = 10.000 \cdot \left(1 + \frac{0.04}{12}\right)^{12 \cdot 5} = 12,209.39 = \underbrace{\$10,000}_{\text{Principal}} + \underbrace{\$2,000}_{\text{Contract Interest}} + \underbrace{\$209.39}_{\text{Compound Interest}}$
\item $FV_{\text{continuous}} = 10.000 \cdot e^{0.04 \cdot 5} = 12,214.03 = \underbrace{\$10,000}_{\text{Principal}} + \underbrace{\$2,000}_{\text{Contract Interest}} + \underbrace{\$214.03}_{\text{Compound Interest}}$
\eit

As we can see, the more frequent the compounding, the higher the future value. It is important to note that in real
life, the compounding effect is usually not as simple as this example. In most cases, the interest rate is different
depending on the compounding period.
\ee

 \v

Similarly, the present value $PV$ of a principal amount $P$ to be received in $t$, assuming a compounding discount rate
$r$ and $n$ compounding periods per period is:
\bse
PV = \frac{FV}{(1 + \frac{r}{n})^{nt}}
\ese

\section{Cash Flows}

In finance, the concept of the time value of money is used to evaluate ``cash flows''.

\bd[Cash Flow]
A \textbf{cash flow} $CF_t$ at time step $t$ is a payment to be received or to be made at time $t$.
\ed

Since cash flows are payment to happen in some point in time, they carry a present and a future value. In simple words,
the present values of a cash flow, is the amount of money one needs to invest today in order to obtain the cash flow
amount to be obtained in the future, and the future value of a cash flow is the amount of money the cash flow will be
worth in the future.

\subsection{Single Cash Flow}

Let's start with the simplest case of a single cash flow $CF$ to be received at time $t=T$. In finance, it is common to
visualize cash flows using a timeline where time is represented on the horizontal axis and the cash flows on the
vertical axis. For a positive cash flow the line is above the time axis, and for a negative cash flow the line is below
the time axis.

\fig{cf1}{0.5}

In this case, the present value is simply the discounted present value of the single future amount of money. Assumming
a compounding discount rate $r$, and $n$ compounds by period, by using the present value equation we get:
\bse \label{eq:pv_single}
PV = \frac{CF}{(1 + \frac{r}{n})^{n \cdot T}}
\ese

Similarly, the future value is simply the future amount of money the cash flow will be worth at time $t=T$. Assumming a
compounding interest rate $r$, and $n$ compounds by period, by using the future value equation we get:
\bse
FV = CF \cdot (1 + \frac{r}{n})^{nT}
\ese

\subsection{Series Of Cash Flows}

Now let's consider the slighlty more complicated case of a series of cash flows $CF_1$, $CF_2$, $\ldots$, $CF_n$ to be
received at time steps $t=1, 2, \ldots, T$.

\fig{cf2}{0.5}

In this case, the present value is simply the sum of the present values of the individual cash flows, with the present
value of each individual cash flow being the discounted value of the future amount of money. \v

Assumming a compounding discount rate $r$, and $n$ compounds by period, by using the present value equation we get:
\bse \label{eq:pv_series}
PV = \frac{CF_1}{(1 + \frac{r}{n})^{n \cdot 1}} + \frac{CF_2}{(1 + \frac{r}{n})^{n \cdot 2}} + \ldots + \frac{CF_T}{(1 + \frac{r}{n})^{n \cdot T}} = \sum_{i=1}^T \frac{CF_i}{(1 + \frac{r}{n})^{n \cdot i}}
\ese

Similarly, the future value is simply the sum of the future values of the individual cash flows, with the future value
of each individual cash flow being the future amount of money the cash flow will be. Assumming a compounding
discount rate $r$, and $n$ compounds by period, by using the future value equation we get:
\bse
FV = CF_1 \cdot (1 + \frac{r}{n})^{n \cdot 1} + CF_2 \cdot (1 + \frac{r}{n})^{n \cdot 2} + \ldots + CF_T \cdot (1 + \frac{r}{n})^{n \cdot T} = \sum_{i=1}^T CF_i \cdot (1 + \frac{r}{n})^{n \cdot i}
\ese

\subsection{Annuity}

A special case of a series of cash flows is the case of an annuity.

\bd[Annunity]
An \textbf{annuity} is a stream of equal cash flows made at regular intervals.
\ed

Notice that a prerequisite for a series of cash flows to be considered an annuity is that the payments or receipts are
equal and are made at regular intervals. One can distinguish between two types of annuities: ``ordinary annuities'' and
``annuities due''.

\bd[Ordinary Annuity]
An \textbf{ordinary annuity} is an annuity in which the payments are made at the end of each period.
\ed

\fig{cf3}{0.5}

For an ordinary annuity things are simple. Since all cash flows are equal i.e.\ $CF_i = CF \: \forall \: i$ the formula for
the present value of a series of cash flows simplifies to:
\bse
PV = \sum_{i=1}^T \frac{CF}{(1 + \frac{r}{n})^{n \cdot i}} = CF \cdot \sum_{i=1}^T \frac{1}{\Big((1 + \frac{r}{n})^n \Big)^i}
\ese

A careful observer will notice that the sum in the above equation is the sum of a geometric series, and by using the
formula for the sum of a geometric series we get:
\bse
PV = CF \cdot \frac{1 - (1 + \frac{r}{n})^{-nT}}{1 - (1 + \frac{r}{n})^{-n}}
\ese

Similarly, the formula for the future value of a series of cash flows simplifies to:
\bse
FV = \sum_{i=1}^T CF \cdot (1 + \frac{r}{n})^{n \cdot i} = CF \cdot \sum_{i=1}^T \Big((1 + \frac{r}{n})^n\Big)^i
\ese

Again, the sum in the above equation is the sum of a geometric series, and by using the formula for the sum of a
geometric series we get:
\bse
FV = CF \cdot \frac{(1 + \frac{r}{n})^{nT} - 1}{\frac{r}{n}}
\ese

\bd[Annuity Due]
An \textbf{annuity due} is an annuity in which the payments are made at the beginning of each period.
\ed

\fig{cf4}{0.45}

The main difference of an annuity due from an ordinary annuity is that the first cash flow is received at the beginning
of the first period, and not at the end. This means that the first cash flow is not discounted at all, the second cash
flow is discounted once, the third cash flow is discounted twice, and so on. \v

Hence, the present value of an annuity due can be calculated by compounding for one period the present value of an
ordinary annuity:
\bse
PV = CF \cdot \frac{1 - (1 + \frac{r}{n})^{-nT}}{1 - (1 + \frac{r}{n})^{-n}} \cdot (1 + \frac{r}{n})^{n}
\ese

Similarly, the future value of an annuity due can be calculated by compounding for one period the future value of an
ordinary annuity:
\bse
FV = CF \cdot \frac{(1 + \frac{r}{n})^{nT} - 1}{\frac{r}{n}} \cdot (1 + \frac{r}{n})^{n}
\ese

\subsection{Perpetuity}

\bd[Perpetuity]
A \textbf{perpetuity} is an annuity that has no end, or a stream of cash payments that continues forever.
\ed

The present value of a perpetuity is simply the present value of an annuity with an infinite number of cash flows.
By using the formula for the present value of an annuity and by sending the number of cash flows to infinity we get:
\bse
PV = \lim_{T \to \infty} \Big(CF \cdot \frac{1 - (1 + \frac{r}{n})^{-nT}}{1 - (1 + \frac{r}{n})^{-n}}\Big) = CF \cdot \frac{1}{1 - (1 + \frac{r}{n})^{-n}}
\ese

\section{Various Rates}

In finance, the interest rate is the most important rate, but there are other rates that are used to evaluate the
time value of money. In this section we will introduce some of the most important ones.

\subsection{Effective Annual Rate}

Given that the formula for calculating the future value takes into account not only the interest rates but also the
compounding periods, a one-to-one comparison between different interest rates with different compounding periods is
not straightforward. For this reason, the concept of the ``effective annual rate'' is introduced.

\bd[Effective Annual Rate (EAR)]
The \textbf{effective annual rate} $r_{\text{eff}}$ is the actual interest rate that is paid on an investment taking
into account the compounding periods $n$:
\bse
r_{\text{eff}} = (1 + \frac{r}{n})^n - 1
\ese
\ed

\be
Given three different interest rates: $r_{\text{annual}} = 4.03\%$, $r_{\text{semi-annual}} = 4.00\%$,
$r_{\text{monthly}} = 3.96\%$, and $r_{\text{continuous}} = 3.9\%$ we can compute the effective annual rates as:
\bit
\item $r_{\text{eff, annual}} = \left(1 + \frac{0.0403}{1}\right)^1 - 1 = 4.03\%$
\item $r_{\text{eff, semi-annual}} = \left(1 + \frac{0.0400}{2}\right)^2 - 1 = 4.04\%$
\item $r_{\text{eff, monthly}} = \left(1 + \frac{0.0396}{12}\right)^{12} - 1 = 4.03\%$
\item $r_{\text{eff, continuous}} = e^{0.039} - 1 = 3.96\%$
\eit

As we can see, while the annual interest rate seems like the best option since it's the highest, the effective annual
rate of semi-annual compounding is higher due to compounding effect. The main point here is that comparing interest
rates with different compounding periods is not an apples-to-apples comparison and one needs to switch to the
effective annual rate to make a fair comparison.
\ee

\subsection{Rate Of Return}

Another important concept in finance is the ``rate of return''. The rate of return is a measure of the profitability
of an investment defined as the net gain or loss of an investment over a specified time period, expressed as a
percentage of the investment's initial cost. When calculating the rate of return, we are determining the percentage
change from the beginning of the period until the end.

\bd[Rate Of Return]
The \textbf{rate of return} $R$ is the gain or loss on an investment over a specified period, expressed as a percentage
of the investment's cost:
\bse
R = \frac{FV - PV}{PV} = \frac{FV}{PV} - 1
\ese
\ed

In a similar way we define the ``annual rate of return'' as the rate of return per year.

\bd[Annual Rate Of Return] \label{def:annual_rate_of_return}
The \textbf{annual rate of return} $R_{\text{annual}}$ is the rate of return per year:
\bse
R_{\text{annual}} = \Big(\frac{FV}{PV}\Big)^{\frac{1}{t}} - 1
\ese
\ed

\be
Given a principal of \$10,000 invested for 5 years at a 4\% annual interest rate, the future value is:
\bse
FV = 10,000 \cdot \left(1 + 0.04\right)^5 = 12,166.53
\ese

The rate of return is:
\bse
R = \frac{12,166.53 - 10,000}{10,000} = 0.216653 = 21.6653\%
\ese

The annual rate of return is:
\bse
R_{\text{annual}} = \Big(\frac{12,166.53}{10,000}\Big)^{\frac{1}{5}} - 1 = 0.04 = 4\%
\ese
\ee