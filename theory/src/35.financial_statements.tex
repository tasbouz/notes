%! suppress = Quote
%! suppress = EscapeAmpersand
\bd[Financial Statements / Financial Reports]
\textbf{Financial statements} or \textbf{financial reports} are written records that convey the business activities and
the financial performance of a company.
\ed

Financial statements are created by accountants and are often audited by government agencies, accountants, firms, etc.
to ensure accuracy and for tax, financing, or investing purposes. There are three basic financial statements:
\bit
\item Balance Sheet.
\item Income Statement.
\item Cash Flow Statement.
\eit

In what follows we will go through all of them one by one. Before that though, accountants have some basic, necessary
rules and assumptions upon which rest all their work in preparing financial statements. These accounting rules and
assumptions dictate what financial items to measure and when and how to measure them. These accounting profession's
set of rules and guiding principles are set by specific organizations of each country.

\be
Financial statements in the United States must be prepared according to the accounting profession's set of rules and
guiding principles called the Generally Accepted Accounting Principles, GAAP for short. Other countries use different
rules. GAAP is a series of conventions, rules and procedures for preparing and reporting financial statements. The
Financial Accounting Standards Board, FASB for short, lays out the GAAP conventions, rules and procedures. The FASB's
mission is ``to establish and improve standards of financial accounting and reporting for guidance and education of
the public, including issuers, auditors, and users of financial information''.
\ee

Now, let's go through the GAAP conventions.

\section{Basic Accountants Principles}

\bd[Accounting Entity]
The \textbf{accounting entity} is the business entity for which the financial statements are being prepared.
\ed

The accounting entity principle states that there is a ``business entity'' separate from its owners, a fictional
``person'' called a ``company'' for which the books are written.

\bd[Going Concern]
\textbf{Going concern} is the assumption which states that, unless there is evidence to the contrary, the life of the
business entity is infinitely long.
\ed

Obviously this assumption can not be verified and is hardly ever true. But this assumption does greatly simplify the
presentation of the financial position of the firm and aids in the preparation of financial statements. If during the
review of a corporation's books, the accountant has reason to believe that the company may go bankrupt, he must issue
a ``qualified opinion'' stating the potential of the company's demise.

\bd[Measurement]
Accounting only deals with things that can be \textbf{measured}.
\ed

Accounting deals with things that can be quantified—resources and obligations upon which there is an agreed-upon
value. Financial statements contain only the quantifiable estimates of assets (what the business owns) and
liabilities (what the business owes). The difference between the two equals owner's equity. (We will properly define
all these terms in the next section). This assumption leaves out many very valuable company ``assets''.

\be
For example, loyal customers, while necessary for company success, still cannot be quantified and assigned a value
and thus are not stated in the books.
\ee

\bd[Units Of Measure]
There is a single \textbf{unit of measure} which represent the unit of value reported in the financial statements.
\ed

\be
U.S.\ dollars are the units of value reported in the financial statements of U.S.\ companies. Results of any foreign
subsidiaries are translated into dollars for consolidated reporting of results. As exchange rates vary, so do the values
of any foreign currency denominated assets and liabilities.
\ee

\bd[Historical Cost]
What a company owns and what it owes are recorded at their original, \textbf{historical cost } with no adjustment for
inflation.
\ed

\be
A company can own a building valued at \$50 million yet carry it on the financial statements at its \$5 million
original purchase price (less accumulated depreciation), a gross understatement of value.
\ee

This assumption can greatly understate the value of some assets purchased in the past and depreciated to a very low
amount on the financial statements. However, it is the easiest thing to do since you do not have to appraise and
reap-praise all the time.

\bd[Materiality]
\textbf{Materiality} refers to the relative importance of different financial information.
\ed

Accountants don't sweat the small stuff. But all transactions must be reported if they would materially affect the
financial condition of the company.

\be
Remember, what is material for a corner drug store is not material for IBM (lost in the rounding errors). Materiality
is a straightforward judgment call.
\ee

\bd[Estimates \& Judgments]
Complexity and uncertainty make any measurement less than exact. \textbf{Estimates} and \textbf{judgments} must often
be made for financial reporting.
\ed

It is okay to guess if that is the best you can do and the expected error would not matter much anyway. But accountants
should use the same guessing method for each period. Be consistent in your guesses and do the best you can. \v

Sometimes identical transactions can be accounted for differently. You could do it this way or that way, depending upon
some preference.

\bd[Consistency]
The principle of \textbf{consistency} states that each individual enterprise must choose a single method of reporting
and use it consistently over time.
\ed

In other words you cannot switch back and forth. Measurement techniques must be consistent from any one fiscal period
to another.

\bd[Conservatism]
\textbf{Conservatism} refers to the downward measurement bias of accountants, preferring understatement to
overvaluation.
\ed

\be
For example, losses are recorded when you feel that they have a great probability of occurring, not later, when they
actually do occur. Conversely, the recording of a gain is postponed until it actually occurs, not when it is only
anticipated.
\ee

\bd[Periodicity]
\textbf{Periodicity} refers to the assumption that the life of a corporation can be divided into periods of time for
which profits and losses can be reported, usually a month, quarter or year.
\ed

What is so special about a month, quarter or year? They are just convenient periods; short enough so that management
can remember what has happened, long enough to have meaning and not just be random fluctuations. These periods are
called ``fiscal'' periods.\footnote{A fiscal period is a period that companies and governments use for financial
reporting and budgeting. Although a fiscal year can start on Jan. 1 and end on Dec. 31, not all fiscal years
correspond with the calendar year. For example, a fiscal year could extend from October 1 in one year till September
30 in the next year. Another example are universities which often begin and end their fiscal years according to the
school year.}

\bd[Substance Over Form]
Accountants report the economic \textbf{substance} of a transaction rather than just its \textbf{form}.
\ed

\be
For example, an equipment lease that is really a purchase dressed in a costume, is booked as a purchase and not as a
lease on financial statements. This substance over form rule states that if it's a duck, then you must report it as a
duck.
\ee

\bd[Accrual Basis Of Presentation]
Accountants translate into dollars of profit or loss all the money-making (or losing) activities that take place during
a fiscal period.
\ed

In accrual accounting, if a business action in a period makes money, then all its product costs and its business
expenses should be reported in that period. Otherwise, profits and losses could flop around depending on which period
entries were made. In accrual accounting, this documentation is accomplished by matching for presentation: (1) the
revenue received in selling product and (2) the costs to make that specific product sold. \v

Fiscal period expenses such as selling, legal, administrative and so forth are then subtracted. Key to accrual
accounting is determining: (1) when you may report a sale on the financial statements, (2) matching and then
reporting the appropriate costs of products sold and (3) using a systematic and rational method allocating all the
other costs of being in business for the period. We will deal with each point separately:
\bit
\item \textbf{Revenue recognition}: In accrual accounting, a sale is recorded when all the necessary activities to
provide the good or service have been completed regardless of when cash changes hands. A customer just ordering a
product has not yet generated any revenue. Revenue is recorded when the product is shipped.
\item \textbf{Matching principle}: In accrual accounting, the costs associated with making products (Cost of Goods Sold)
are recorded at the same time the matching revenue is recorded.
\item \textbf{Allocation}: Many costs are not specifically associated with a product. These costs must be allocated to
fiscal periods in a reasonable fashion. For example, each month can be charged with one-twelfth of the general
business insurance policy even though the policy was paid in full at the beginning of the year. Other expenses are
recorded when they arise (period expenses).
\eit

Those were the basic accountant principles where financial statements are created based on. Now, let's go through the
three basic financial statements starting with the balance sheet.

\section{Balance Sheet}

\bd[Balance Sheet]
A \textbf{balance sheet} (also known as statement of financial position or statement of financial condition) is a
summary of the financial balances of an organization.
\ed

The balance sheet presents what the enterprise has today i.e.\ its assets (\ref{def:asset}), how much the enterprise
owes today i.e.\ its liabilities (\ref{def:liability}), and what the enterprise is worth today, i.e.\ equity. \v

The basic equation of accounting states that ``what you're worth is what you have minus what you owe is'', or in a
simpler form:
\bse
\text{Equity} = \text{Assets} - \text{Liabilities}
\ese

or in the form that appear in the balance sheet:
\bse
\text{Assets} = \text{Liabilities} + \text{Equity}
\ese

By definition, this equation must always be in balance, with assets equaling the sum of liabilities and equity. So, if
you add an asset to the left side of the equation, you must also increase the right side by adding a liability or
increasing equity. \v

A balance sheet is often described as a ``snapshot of a company's financial condition'', since assets, liabilities
and equity are listed as of a specific date (the date it was written), such as the end of its fiscal year. Of the
three basic financial statements, the balance sheet is the only statement which applies to a single point in time of a
business's calendar year. \v

In order to fully align with the $\text{Assets} = \text{Liabilities} + \text{Equity}$ equation, a standard balance
sheet has two sides: assets on the left, and liabilities and equity on the right. The main categories of assets are
usually listed first, and typically in order of liquidity. Assets are followed by the liabilities and equity.

\subsection{Assets}

Assets are displayed in the asset section of the balance sheet in the descending order of liquidity. Based on liquidity,
assets are divided into three subclasses: current assets, other assets, and fixed assets.

\bd[Current Asset]
A \textbf{current asset} is any asset which can reasonably be expected to be sold, consumed, or exhausted through the
normal operations of a business within the current fiscal year or operating cycle or financial year (whichever period is
longer).
\ed

\bd[Other Asset]
\textbf{Other asset} is a catchall category that includes intangible assets such as the value of patents, trade names
and so forth.
\ed

\bd[Fixed Asset]
A \textbf{fixed asset} (or long-lived assets, or property, plant and equipment (PP\&E)) is any asset that may not
easily be converted into cash.
\ed

\subsubsection{Current Assets}

By definition, current assets are those assets that are the most liquid hence, they appear first in the balance sheet,
and they are only all tangible assets. Let's start exploring them one by one.

\bd[Cash]
\textbf{Cash} is money in the physical form of currency, such as banknotes and coins.
\ed

\bd[Cash Equivalents]
\textbf{Cash equivalents} are investments securities that are meant for short-term investing; they have high credit
quality and are highly liquid.
\ed

Cash and cash equivalents are the ultimate liquid assets. Like all the rest of the balance sheet, cash is denominated
in U.S.\ dollars for corporations in the United States. A U.S.\ company with foreign subsidiaries would convert the
value of any foreign currency it holds (and also other foreign assets) into dollars for financial reporting.

\bd[Short Term Investments]
\textbf{Short term investments} include securities bought and held for sale in the near future to generate income on
short-term price differences.
\ed

Short term investment can be sold to market and become cash.

\bd[Accounts Receivable]
\textbf{Accounts receivable} (AR) are legally enforceable claims for payment held by a business for goods supplied or
services rendered that customers have ordered but not paid for.
\ed

Accounts receivable upon collection becomes cash.

\bd[Inventory / Stock]
\textbf{Inventory} or \textbf{stock} refers to the goods and materials that a business holds for the ultimate goal of
resale, production or utilisation.
\ed

Inventory is both finished products for ready sale to customers and also materials to be made into products. A
manufacturer's inventory includes three groupings:
\bit
\item{Raw Material Inventory}: which is unprocessed materials that will be used in manufacturing products.
\item{Work-In-Process Inventory}: which is partially finished products in the process of being manufactured.
\item{Finished Goods Inventory}: which is completed products ready for shipment to customers when they place orders.
\eit

When finished goods inventory is sold it becomes an accounts receivable and then cash when the customer pays.

\bd[Prepaid Expenses / Prepaid Liabilities]
\textbf{Prepaid expenses} or \textbf{Prepaid liabilities} are bills the company has already paid for services not yet
received.
\ed

\be
Prepaid expenses are things like prepaid insurance premiums, prepayment of rent, deposits paid to the telephone
company, salary advances, etc.
\ee

Prepaid expenses are current assets not because they can be turned into cash, but because the enterprise will not
have to use cash to pay them in the near future. They have been paid already.

These are the current assets. If we want to put them in a formula one could write:

\bi{rCl}
\text{Current Assets} & = & \text{ Cash} + \text{ Cash Equivalents} + \text{ Short Term Investments} \\
&+& \text{ Accounts Receivable} + \text{ Inventory} + \text{ Prepaid Expenses}
\ei

\subsubsection{Other Assets}

Other assets (and fixed assets), are the so-called ``non-current assets'' that are not converted into cash during the
normal course of business, therefore they cannot be properly classified into current asset. As we already defined,
other assets is a catchall category that includes intangible assets such as the value of patents, trade names and so
forth.

\subsubsection{Fixed Assets}

A fixed asset, is a term used for assets and property that may not easily be converted into cash. Fixed assets are
different from current assets because the latter are liquid assets. In most cases, only tangible assets are referred to
as fixed. \v

Fixed assets are productive assets not intended for sale. They will be used over and over again to manufacture the
product, display it, warehouse it, transport it and so forth. Fixed assets commonly include land, buildings,
machinery, equipment, furniture, automobiles, trucks, etc. Fixed assets are reported on the balance sheet at original
purchased price, for this reason sometimes they are noted as ``fixed assets at cost''.

\bd[Fixed Assets At Cost]
\textbf{Fixed assets at cost} are the fixed assets at original purchased price.
\ed

However, fixed assets at cost are usually depreciated. Depreciating an asset means spreading the cost to acquire the
asset over the asset's whole useful life.

\bd[Depreciation]
\textbf{Depreciation} is the expense generated by using an asset. It is the wear and tear and thus diminution in the
historical value due to usage.
\ed

\bd[Accumulated Depreciation]
\textbf{Accumulated depreciation} is the sum of all the depreciation charges taken since the asset was first acquired.
\ed

By combining fixed assets at cost and accumulated depreciation (both appear in the balance sheet) one can calculate
the ``net fixed assets''.

\bd[Net Fixed Assets]
The \textbf{net fixed assets} of a company are the sum of its fixed assets at cost, minus the accumulated depreciation
charges over the years.
\bse
\text{Net Fixed Assets} = \text{Fixed assets at cost} - \text{Accumulated Depreciation}
\ese
\ed

The so-called ``book value'' of an asset—its value as reported on the books of the company, i.e.\ asset's purchase
price minus its accumulated depreciation. Note that depreciation does not necessarily relate to an actual decrease
in value. In fact, some assets appreciate in value over time.

\subsubsection{Total Assets}

\bd[Total Assets]
The \textbf{total assets} of a company is the sum of current, net fixed and other assets:
\bse
\text{Total Assets} = \text{Current Assets} + \text{Net Fixed Assets} + \text{Other Assets}
\ese
\ed

\subsection{Liabilities \& Equity}

Liabilities are categorized and grouped for presentation on the balance sheet in two main categories: current
liabilities and long-term liabilities, in order to help users assess the company's financial standing in short-term and
long-term periods.

\bd[Current Liabilities]
\textbf{Current liabilities} are often understood as all liabilities of the business that are to be settled in cash
within the fiscal year or the operating cycle of a given firm, whichever period is longer.
\ed

\bd[Long-Term Liabilities]
\textbf{Long-term liabilities} are liabilities that are due beyond a year or the normal operation period of the company.
\ed

Current liabilities inform the user of debt that the company owes in the current period while long-term liabilities
give users more information about the long-term prosperity of the company, while.

\subsubsection{Current Liabilities}

As it was the case with assets, in the same way liabilities are listed in order of liquidity on a balance sheet, so
current liabilities come before long-term liabilities. Let's start exploring them one by one.

\bd[Accounts Payable]
\textbf{Accounts payable} is money owed by a business to its suppliers shown as a liability on a company's balance
sheet.
\ed

Accounts payable are bills, generally to other companies for materials and equipment bought on credit, that the
corporation must pay soon. When it receives materials, the corporation can either pay for them immediately with cash
or wait and let what is owed become an account payable.

\bd[Accrued Expenses]
\textbf{Accrued expenses} is money owed by a business to its employees and services.
\ed

Accrued expenses are monetary obligations similar to accounts payable. The business uses one or the other
classification depending on to whom the debt is owed. Accounts payable is used for debts to regular suppliers of
merchandise or services bought on credit, while accrued expenses are salaries earned by employees but not yet
paid to them, lawyers' bills not yet paid, interest due but not yet paid on bank debt and so forth.

\bd[Current Portion Of Long-Term Debt]
\textbf{Current portion of long-term debt} is the total amount of long-term debt that must be paid within 12 months.
\ed

In simple words, the current portion of long-term debt is the portion of a long-term liability that is coming due
within the next twelve months. It is separated out on the company's balance sheet because it needs to be paid by
highly liquid assets, such as cash. It is an important tool for creditors and investors to use to identify if a
company has the ability to pay off its short-term obligations as they come due.

\bd[Income Taxes Payable]
\textbf{Income taxes payable} is the amount that an organization expects to pay in income taxes within 12 months.
\ed

Every time the company sells something and makes a profit on the sale, a percentage of the profit will be owed the
government as income taxes. Income taxes payable are income taxes that the company owes the government but that the
company has not yet paid. Every three months or so the company will send the government a check for the income taxes
owed. For the time between when the profit was made and the time when the taxes are actually paid, the company will
show the amount to be paid as income taxes payable on the balance sheet.

These are the current liabilities. If we want to put them in a formula one could write:
\bi{rCl}
\text{Current Liabilities} & = & \text{ Accounts Payable} + \text{ Accrued Expenses} \\
&+& \text{ Current Portion Of Long-Term Debt} + \text{ Income Taxes Payable}
\ei

\subsubsection{Long-Term Liabilities}

Long-term liabilities are liabilities that are due beyond a year or the normal operation period of the company. On a
classified balance sheet, liabilities are separated between current and long-term liabilities to help users assess
the company's financial standing in short-term and long-term periods. Long-term liabilities give users more
information about the long-term prosperity of the company, while current liabilities inform the user of debt that the
company owes in the current period. \v

On a balance sheet, accounts are listed in order of liquidity, so long-term liabilities come after current
liabilities. In addition, the specific long-term liability accounts are listed on the balance sheet in order of
liquidity. Therefore, an account due within eighteen months would be listed before an account due within twenty-four
months. Examples of long-term liabilities are bonds payable, long-term loans, capital leases, pension liabilities,
post-retirement healthcare liabilities, deferred compensation, deferred revenues, deferred income taxes, and
derivative liabilities. \v

More often than not, all long-term liabilities are gathered in the so-called ``long-term debt''.

\bd[Long-Term Debt]
\textbf{Long-term debt} is the total amount of long-term debt that must be paid in more than 12 months.
\ed

\be
Common types of long-term debt include mortgages for land and buildings and so-called chattel mortgages for machinery
and equipment.
\ee

\subsubsection{Total Liabilities}

\bd[Total Liabilities]
The \textbf{total liabilities} of a company is the sum of current and long-term liabilities:
\bse
\text{Total Liabilities} = \text{Current Liabilities} + \text{Long-Term Liabilities}
\ese
\ed

More often than not, there is not a separate line for total liabilities in most balance sheets.

\subsubsection{Equity}

If you subtract what the company owes (total liabilities) from what it has (total assets), you are left with the
company's value to its owners, called worth, or net worth, or equity, or owners' equity, or shareholders' equity,
or book value. From now on we will stick with the shorter word, i.e.\ equity.

\bd[Equity]
\textbf{Equity} is the amount of money left over after you subtract total liabilities from total assets:
\bse
\text{Equity} = \text{Total Assets} - \text{Total Liabilities}
\ese
\ed

For sake of completeness we can give another definition, that of net current asset, or working capital.

\bd[Net Current Assets / Working Capital]
\textbf{Net current assets} or \textbf{working capital} is the amount of money left over after you subtract current
liabilities from current assets:
\bse
\text{Net Current Assets} = \text{Current Assets} - \text{Current Liabilities}
\ese
\ed

In simple words, working capital is the amount of money the enterprise has to work with in the short-term. Sources of
working capital are ways working capital increases in the normal course of business. This increase in working capital
happens when current liabilities decrease and/or current assets increase. Uses of working capital are ways working
capital decreases during the normal course of business. For example, when current assets decrease and/or current
liabilities increase. With lots of working capital it will be easy to pay your current financial obligations. In any
case, let's go back to equity. \v

Equity is a very special kind of liability. It represents the value of the corporation that belongs to its owners.
However, this ``debt'' will never be repaid in the normal course of business. Equity has two components: capital
stock and retained earnings.

Before we move on to the next current liability, we need to provide some important definitions.

\bd[Stock / Equity]
A \textbf{stock}, also known as \textbf{equity}, is a security that represents the ownership of a fraction of the
issuing corporation.
\ed

\bd[Share]
Units of stock are called \textbf{shares} which entitles the owner to a proportion of the corporation's
assets and profits equal to how much stock they own.
\ed

Stocks are bought and sold predominantly on stock exchanges and are the foundation of many individual investors'
portfolios. Stock trades have to conform to government regulations meant to protect investors from fraudulent practices.
\v

There are two main types of stock: ``common'' and ``preferred''.

\bd[Common Stock]
A \textbf{common stock} gives the stockholder the right to share in the profits of the company, and to vote on matters
of corporate policy and the composition of the members of the board of directors.
\ed

\bd[Preferred Stock]
A \textbf{preferred stock} is a component of share capital that may have any combination of features not possessed by
common stock, including properties of both an equity and a debt instrument, and is generally considered a hybrid
instrument.
\ed

Preferred stocks are senior (i.e.\ higher ranking) to common stock but subordinate to bonds in terms of claim (or
rights to their share of the assets of the company, given that such assets are payable to the returnee stock bond)
and may have priority over common stock in the payment of dividends and upon liquidation. Terms of the preferred
stock are described in the issuing company's articles of association or articles of incorporation. \v

Now that we defined stocks we can give the definition of ``capital stock'' which is the next equity part.

\bd[Capital Stock]
\textbf{Capital stock} is the amount of common and preferred shares that a company is authorized to issue, according
to its corporate charter. Capital stock can only be issued by the company and is the maximum number of shares that
can ever be outstanding.
\ed

The original money to start and any add-on money invested in the business is represented by shares of capital stock
held by owners of the enterprise. So-called common stock is the regular ``denomination of ownership'' for all
corporations. All companies issue common stock, but they may issue other kinds of stock, too. Companies often issue
preferred stock that have certain contractual rights or ``preferences'' over the common stock. These rights may include
a specified dividend and/or a preference over common stock to receive company assets if the company is liquidated. \v

Before we move on to the next equity part, we need to provide the definition of a ``dividend''

\bd[Dividend]
A \textbf{dividend} is a distribution of profits by a corporation to its shareholders.
\ed

A dividend is allocated as a fixed amount per share, with shareholders receiving a dividend in proportion to their
shareholding. Dividends can provide stable income and may be subject to income tax called ``dividend
tax''. The tax treatment of this income varies considerably between jurisdictions. The corporation does not receive a
tax deduction for the dividends it pays. \v

So, when a corporation earns a profit or surplus, it is able to pay a portion of the profit as a dividend to
shareholders. Any amount not distributed is taken to be re-invested in the business. This amount is called ``retained
earnings'' and it is a liability reported in the balance sheet.

\bd[Retained Earnings]
\textbf{Retained\footnote{As an important concept in accounting, the word ``retained'' captures the fact that because
those earnings were not paid out to shareholders as dividends, they were instead retained by the company. For this
reason, retained earnings decrease when a company either loses money or pays dividends and increase when new profits
are created.} earnings} are the cumulative net earnings or profits of a company after accounting for dividend
payments.
\bse
\text{Retained Earnings} = \text{Sum Of All Profits} - \text{Sum Of All Dividends}
\ese
\ed

Retained earnings can be viewed as a ``pool'' of money from which future dividends could be paid. In fact, dividends
cannot be paid to shareholders unless sufficient retained earnings are on the balance sheet to cover the total amount
of the dividend checks. If the company has not made a profit but rather has sustained losses, it has ``negative
retained earnings'' that are called its accumulated deficit.

\bd[Accumulated Deficit]
Negative retained earnings are called \textbf{accumulated deficit}.
\ed

Hence:
\bse
\text{Equity} = \text{Capital Stock} + \text{Retained Earnings}
\ese

From this formula we see that equity is just the sum of the investment made in the stock of the company plus any
profits (less any losses) minus any dividends that have been paid to shareholders. The value of equity increases when
the company makes a profit thereby increasing retained earnings, or sells new stock to investors thereby increasing
capital stock. The value of equity decreases when the company has a loss thereby lowering retained earnings or pays
dividends to shareholders, thereby lowering retained earnings.

\subsection{Balance Sheet Summary}

\fig{balance_sheet}{0.5}

The balance sheet presents the financial picture of the enterprise on one particular day, an instant in time. By
definition, this equation must always be in balance with assets equaling the sum of liabilities and equity. \v

The balance sheet along with the income statement form the two major financial statements of the company. So now let's
move on to the so called ``income statement''.

\section{Income Statement}

\bd[Income Statement]
An \textbf{income statement}\footnote{Income statement can also be refered to as: profit and loss account, profit
and loss statement (P\&L), statement of profit or loss, revenue statement, statement of financial performance,
earnings statement, statement of earnings, operating statement, or statement of operations.} is one of the financial
statements of a company and shows the company's revenues and expenses during a particular period.
\ed

While the balance sheet reports on assets, liabilities and equity, the income statement gives one important
perspective on the health of a business: its profitability. It indicates how the revenues are transformed into the
net income or net profit (the result after all revenues and expenses have been accounted for), while saying nothing
about when the company receives cash or how much cash it has on hand. The sole purpose of the income statement is to
show managers and investors whether the company made money (profit) or lost money (loss) during the period being
reported. In contrast with the balance sheet, which represents a single moment in time, an income statement
represents a period of time (as does the cash flow statement which will see in the next section). \v

The income statement reports on making and selling activities of a business over a period of time: what's sold in the
period minus what it cost to make minus selling and general expenses for the period equals income for the period. The
income statement documents for a specific period the second basic equation of accounting:
\bse
\text{Income} = \text{Sales} - \text{Costs \& Expenses}
\ese

or in the form that appear in the income statement:
\bse
\text{Sales} - \text{Costs \& Expenses} = \text{Income}
\ese

\subsection{Sales}

\bd[Sales / Revenue]
\textbf{Sales} or \textbf{revenue} is the total amount of income generated by the sale of goods and services related to the
primary operations of the business.
\ed

\bd[Net Sales / Net Revenue]
\textbf{Net sales} or \textbf{net revenue} is the total amount of income generated by the sale of goods and services
related to the primary operations of the business less any discounts offered to the customer to induce purchase.
\ed

Sales (or net sales) are recorded on the income statement when the company actually ships products to customers.
Customers now have an obligation to pay for the product and the company has the right to collect. When the company
ships a product to a customer, it also sends an invoice (a bill). As we already said in the previous section, the
company's right to collect is called an account receivable and is entered on the company's balance sheet. \v

It is important to make a distinction between sales and orders. A sale is made when the company actually ships a
product to a customer. Orders, however, are something different. Orders become sales only when the products ordered
have left the company's loading dock and are en route to the customer. When a sale is made, income is generated on
the income statement. Orders only increase the ``backlog'' of products to be shipped and do not have an impact on the
income statement in any way. Simply receiving an order does not result in income.

\subsection{Costs \& Expenses}

It is important to make the distinction between the two different terms: cost and expense, since using these terms
correctly will make it easier to understand how the income statement and balance sheet work together. Both terms are
used to describe how the company spends its money. The crucial difference is that manufacturing expenditures to build
inventories are called costs while all other business expenditures are called expenses. Also note that an expenditure
can be either a cost or an expense. Expenditure simply means the use of cash to pay for an item purchased. Let's see
all this one by one.

\bd[Cost]
A \textbf{cost} is the value of money that has been used up to produce something or deliver a service, and hence, is not
available for use anymore.
\ed

Costs are expenditures for raw materials, workers' wages, manufacturing overhead and so forth. Costs are what you
spend when you buy (or make) products for inventory. To make a connection with the previous chapter, costs lower cash
and increase inventory values on the balance sheet. Only when inventory is sold does its value move from the balance
sheet to the income statement. More specifically, when this inventory is sold, that is, shipped to customers, its
total cost is taken out of inventory and entered in the income statement as a special type of expense called ``cost
of goods sold''.

\bd[Cost Of Goods Sold]
\textbf{Cost of goods sold} refers to the direct costs of producing the goods sold by a company.
\ed

When a product is shipped and a sale is booked, the company records the total cost of manufacturing the product as
cost of goods sold on the income statement. This includes the cost of the materials and labor directly used to create
the good. It excludes indirect expenses, such as distribution costs and sales force costs. When the company made the
product, it took all the product's costs and added them to the value of inventory. The costs to manufacture products
are accumulated in inventory until the products are sold. Then these costs are expensed through the income statement
as cost of goods sold.

\bd[Gross Margin / Gross Profit]
\textbf{Gross margin} or \textbf{Gross profit} is sales less the cost of goods sold.
\bse
\text{Gross Margin} = \text{Sales} - \text{Cost Of Goods Sold}
\ese
\ed

In simple words, it's the amount of money a company retains after incurring the direct costs associated with
producing the goods it sells and the services it provides. The higher the gross margin, the more capital a company
retains, which it can then use to pay other costs or satisfy debt obligations.

\bd[Expense]
An \textbf{expense} is the cost of operations that a company incurs to generate revenue.
\ed

Expense is simply defined as the cost one is required to spend on obtaining something. Expenses pay for developing
and selling products and for running the general and administrative aspects of the business.

\be
Examples of expenses are paying legal fees and a sales person's salary, buying chemicals for the R\&D laboratory and
so forth.
\ee

\bd[Operating Expense]
An \textbf{operating expense} is an expense a business incurs through its normal business operations.
\ed

Often abbreviated as OPEX, operating expenses are those expenditures that a company makes to generate income and include
rent, equipment, inventory costs, marketing, payroll, insurance, step costs, and funds allocated for research and
development. A common groupings of operating expense are:
\bit
\item Sales \& Marketing expenses.
\item Research \& Development expenses.
\item General \& Administrative expenses.
\eit

\subsection{Income}

\bd[Income / Profit / Earnings]
\textbf{Income} or \textbf{profit} or \textbf{earnings} refers to the money that a business receives in exchange for
their products, and it is equal to:
\bse
\text{Income} = \text{Sales} - \text{Costs \& Expenses}
\ese
\ed

If sales exceed costs plus expenses (as reported on the income statement), the business has earned income. If costs
plus expenses exceed sales, then a loss has occurred. \v

Income is the difference between two very large numbers: sales less costs and expenses. Slightly lower sales and/or
slightly higher costs and expenses can eliminate any expected profit and result in a loss. \v 

As usual, income can have many different parts. Let's see them.

\bd[Income From Opeartions / Operating Income]
\textbf{Income from operations} or \textbf{operating income} is the profit realized from a business' own operations.
\ed

A manufacturing company's operations are all its actions taken in making and selling products, resulting in both
expenses and costs. The term income from operations refers to what is left over after expenses and costs are
subtracted from sales. In simple words income from operations is generated from running the primary business and
excludes income from other sources.

\bd[Interest Income]
\textbf{Interest income} is the amount paid to a business for lending its money or letting another entity use its funds.
\ed

Receiving interest on cash balances in the company's bank account is ``non-operating income''. Because it is from
non-operating sources, interest income (or expense) is reported on the income statement just below the Income from
Operations line.

\bd[Taxes Income]
\textbf{Taxes income} is the amount paid from a business due to taxation.
\ed

A company's operations can be producing income, but the company as a whole can still show an overall loss. This sad
state of affairs comes about when non-operating expenses (such as very high interest expenses) exceed the total
operating income. \v

By combining income from operations, interest income and taxes income we can compute the net income as:

\bd[Net Income]
\textbf{Net income} is defined as the sum of income from operations and interest income less the taxes income:
\bse
\text{Net Income} = \text{Income From Opeartions} + \text{Interest Income} - \text{Taxes Income}
\ese
\ed

For this reason, income from operation it is also called ``earnings before interest and taxes''.

\bd[Earnings Before Interest And Taxes (EBIT)]
\textbf{Earnings before interest and taxes} or \textbf{EBIT} is the profit realized from a business' own operations
before interest and taxes.
\ed

A very important thing is that the words income and revenue are often confused. They mean very different
things. Profit and income do mean the same thing. Sales and revenue do mean the same thing. Income (also called
profits) is at the bottom of the income statement. Sales (also called revenue) is at the top of the income statement
Income is often referred to as the bottom line because it is the last line of the income statement. Sales are
often referred to as the top line because it is at the top of the income statement.

\subsection{Income Statement Summary}

\fig{income_statement}{0.5}

The income statement summarizes and displays the financial impact of movement of goods to customers (sales) minus
efforts to make and sell those goods (costs and expenses) equals any value created in the process (income). All
business activities that generate income or result in a loss for a company, that is, all transactions that change the
value of shareholders' equity, are recorded on the income statement. \v

Balance sheet and income statement are inexorably linked. If the income statement shows income, then retained
earnings are increased on the balance sheet. Then, also, either the enterprise's assets must increase or its
liabilities decrease for the balance sheet to remain in balance. Thus, the income statement shows for a period all
the actions taken by the enterprise to either increase assets or decrease liabilities on the balance sheet.

\section{Cash Flow Statement}

We have already defined cash as money in the physical form of currency, such as banknotes and coins. Based on
this definition we can define the concept of ``cash flow'' as follows.

\bd[Cash Flow]
\textbf{Cash flow} refers to the net amount of cash and cash equivalents being transferred in and out of a company.
\ed

Cash flow comes into the business in two major ways:
\bit
\item Operating activities such as receiving payment from customers.
\item Financing activities such as selling stock or borrowing money.
\eit

Cash flow goes out of the business in four major ways:
\bit
\item Operating activities such as paying suppliers and employees.
\item Financial activities such as paying interest and principal on debt or paying dividends to shareholders.
\item Making major capital investments in long-lived productive assets like machines.
\item Paying income taxes to the government.
\eit

A positive cash flow for a period means the company has more cash at the end of the period than at the beginning. A
negative cash flow for a period means that the company has less cash at the end of the period than at the beginning.
If a company has a continuing negative cash flow, it runs the risk of running out of cash and not being able to pay
its bills when due.

\bd[Cash Flow Statement]
A \textbf{cash flow statement} is a financial statement that shows how changes in balance sheet and income statement
affect cash and cash equivalents, and breaks the analysis down to operating, investing and financing activities.
\ed

Essentially, the cash flow statement is concerned with the flow of cash in and out of the business. It tracks the
movement of cash through the business over a period of time. A company's cash flow statement is just like a check
register, recording all the company's transactions that use cash (checks) or supply cash (deposits). The cash flow
statement shows: cash on hand at the start of a period plus cash received in the period minus cash spent in the
period equals cash on hand at the end of the period. As an analytical tool, the statement of cash flows is useful in
determining the short-term viability of a company, particularly its ability to pay bills.

\bd[Cash Transactions]
A \textbf{cash transaction} is the immediate payment of cash for the purchase of an asset.
\ed

Cash transactions affect cash flow.

\be
For example, paying salaries, paying for equipment and paying off a loan lowers cash. On the other hand, receiving
money borrowed from a bank, receiving money from investors for stock and receiving money from customers raises cash.
\ee

\bd[Non-Cash Transactions]
A \textbf{non-cash transaction} is a transaction where no cash moves from one entity to another.
\ed

Non-cash transactions have no effect on the cash flow statement, but they can affect the income statement and balance
sheet.

\be
Examples of non-cash transactions include: shipping product to a customer, receiving supplies from a vendor and
receiving raw materials required to make the product. For these material transfer transactions, no cash actually
changes hands during the transaction proper, only later.
\ee

Now, let's see what is included inside a cash flow statement, starting with the most obvious one ``beginning cash
balance''.

\bd[Beginning Cash Balance]
On the cash flows statement, \textbf{beginning cash balance} is the amount of cash a company has at the start of the
fiscal period.
\ed

Nothing more to add here, so let's move on!

\bd[Cash Receipt]
A \textbf{cash receipt} is a printed statement of the amount of cash received in a cash transaction.
\ed

Cash receipts (also called simply receipts) come from collecting money from customers. Cash receipts increase the
amount of cash the company has on hand. (Receiving cash from customers decreases the amount that is due the
company as accounts receivable shown on the balance sheet). Cash receipts are not profits. Profits are something
else altogether. Don't confuse the two. Profits are reported on the income statement.

\bd[Cash Disbursement]
\textbf{Cash disbursement} is the outflow of cash paid in exchange for the provision of goods or services.
\ed

A cash disbursement (also called payment or simply disbursement) is writing a check to pay for the rent, for
inventory and supplies or for a worker's salary. Cash disbursements lower the amount of cash the company has on hand.
Cash disbursements to suppliers lower the amount the company owes as reported in accounts payable on the
balance sheet. \v 

So, cash receipts are inflows of money coming from operating the business. On the other hand, cash disbursements are
outflows of money used in operating the business. Cash receipts (money in) minus cash disbursements (money out)
equals ``cash from operations\footnote{The normal day-to-day business activities of a business are called its
operations.}'' which is the first section depicted on a company's cash flow statement.

\bd[Cash From Operations]
\textbf{Cash from operations} indicates the amount of money a company brings in from its ongoing, regular business 
activities, such as manufacturing and selling goods or providing a service to customers. 
\bse
\text{Cash From Operations} = \text{Cash Receipt} - \text{Cash Disbursement}
\ese
\ed

Cash from operations reports the flow of money into and out of the business from the making and selling of products
It is a good measure of how well the enterprise is doing in its day-to-day business activities. \v

The cash flow statement shows cash from operations separately from other cash flows, however cash from operations is
just one of the important elements of cash flow. Money spent to buy property, plant and equipment (PP\&E) is an
investment in the long-term capability of the company to manufacture and sell product. Paying for PP\&E is not
considered part of operations and thus is not reported in cash disbursements from operations. Cash payments for PP\&E
are reported on a separate line on the cash flow statement under a special term called ``fixed asset purchases''.

\bd[Fixed Asset Purchases]
\textbf{Fixed asset purchases} are cash payments related to buying fixed assets.
\ed

PP\&E purchases are investments in productive assets. Needless to say, after paying for PP\&E the business has less
cash. Cash is used when the PP\&E is purchased originally. Note, however, when the enterprise depreciates a fixed
asset, it does not use any cash at that time. No check is written to anyone. \v

Borrowing money increases the amount of cash the company has on hand. Conversely, paying back a loan decreases the
company's supply of cash on hand. The difference between any new borrowings in a period and the amount paid back in
the period is called ``net borrowings''.

\bd[Net Borrowing]
\textbf{Net borrowing} represents the amount of money that the company has borrowed through various sources minus the
amount paid back for uses.
\ed

Net borrowings are reported for the period on a separate line in the cash flow statement. \v

Owing income taxes is different from paying them. The business owes some more income tax every time it sells
something for a profit. But just owing taxes does not reduce cash. Only writing a check to the government and thus
paying the taxes due actually reduces the company's cash on hand.

\bd[Income Taxes Paid]
\textbf{Income taxes paid} represents the amount of money that the company has paid for tax purposes.
\ed

Paying income taxes to the government decreases the company's supply of cash. Income taxes paid are reported on the
cash flow statement. \v

\bd[Sale Of Stock]
\textbf{Sale of stock} refers to the situation when a company sells stocks for exchange of cash.
\ed

When people invest in a company's stock, they exchange one piece of paper for another: real U.S.\ currency for a
stock certificate. When a company sells stock to investors, it receives money and increases the amount of cash it
has on hand. Selling stock is the closest thing to printing money that a company can do. \v

The beginning cash balance plus or minus all cash transactions that took place during the period equals the ending cash
balance.

\bd[Ending Cash Balance]
\textbf{Ending cash balance} is the beginning cash balance plus or minus all cash transactions:
\bi{rCl}
\text{Ending Cash Balance} & = & \text{Beginning Cash Balance} + \text{ Cash From Operations} \\
&-& \text{ Fixed Asset Purchases} + \text{ Net Borrowings} - \text{ Income Taxes Paid} \\
&+& \text{ Sale Of Stock}
\ei
\ed

Thus, beginning cash on hand plus cash received minus cash spent equals ending cash on hand. This is of course equal to
the beginning cash balance of the next cash flow statement.

\subsection{Cash Flow Statement Summary}

\fig{cash_flow}{0.5}

Cash flow statement acts as a check register reporting all the company's payments (cash outflows) and deposits (cash
inflows) for a period of time. If no actual cash changes hands in a particular transaction, then the cash flow
statement is not changed. Note, however, that the balance sheet and income statement may be changed by a non-cash
transaction. Note also that cash transactions reported on the cash flow statement usually do have some effect
on the income statement and balance sheet as well as taken by the enterprise to either increase assets or decrease
liabilities on the balance sheet. \v

To sum up, let's remember the fundamental reporting function of each of the three main financial statements:
\bit
\item The income statement shows the manufacturing and selling actions of the enterprise that results in profit or loss.
\item The cash flow statement details the movements of cash into and out of the coffers of the enterprise.
\item The Balance Sheet records what the company owns and what it owes, including the owner's stake.
\eit

The three major financial statements we introduced work in concert to give a true picture of the enterprise's
financial health. Each statement views the enterprise's financial health from a different and very necessary
perspective, and each statement relates to the other two. Income statement relates to the balance sheet and vice
versa and any changes to each can effect the cash flow statement. Fundamentally, the financial statements document
the movement of cash and goods and services into and out of the enterprise. That is all the financial statements are
about. So, one has to pay attention to the flow of cash money and to the flow of goods and services.