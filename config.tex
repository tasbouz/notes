%! suppress = EscapeHashOutsideCommand

% load packages
\usepackage[english]{babel}
\usepackage[toc, page]{appendix}
\usepackage[framemethod=TikZ]{mdframed}
\usepackage{graphicx, geometry, mathtools, amsthm, amsmath, enumitem, float, dirtytalk, textcomp, amssymb, cancel,
hyperref, IEEEtrantools, mathrsfs, xfrac, pgfplots, bbding, titlepic,cleveref, bm, stmaryrd,  graphicx, movie15,
subfig, tikz, tikz-cd, fontspec, adjustbox, lmodern, color, xcolor, listings, courier, multirow, caption, booktabs,
microtype}

% tikz config
\usetikzlibrary{matrix, calc, positioning, decorations.markings, decorations.pathmorphing, decorations.pathreplacing,
arrows, cd, shapes}
\tikzset{snake it/.style={-stealth,decoration={snake, amplitude = .4mm, segment length = 2mm, post length=0.9mm},
decorate}}

% image paths
\graphicspath{{./img/}}

% numbering and depth in table of contents 
\setcounter{secnumdepth}{5}
\setcounter{tocdepth}{5}

% layout config
\geometry{margin=1in}
\setlength{\abovedisplayskip}{10pt}
\setlength{\belowdisplayskip}{10pt}
\setlength{\abovedisplayshortskip}{5pt}
\setlength{\belowdisplayshortskip}{5pt}
\setlength\parindent{0pt}
\setlength{\jot}{5pt}

% font config
\setmainfont{Avenir}

% captions config
\captionsetup{font=footnotesize}

% extras (theorems, lemmas, etc...)
\theoremstyle{plain}
\newtheorem{definition}{Definition}[chapter]
\newtheorem{lemma}{Lemma}[chapter]
\newtheorem{theorem}{Theorem}[chapter]
\newtheorem{corollary}{Corollary}[chapter]
\newtheorem{proposition}{Proposition}[chapter]
\theoremstyle{remark}
\newtheorem{example}{Example}[chapter]
\newtheorem*{notation}{Notation}
\newtheorem{remark}{Remark}[chapter]
\newtheorem*{solution}{Solution}

% shortcuts
\def\ba{\begin{array}}
\def\ea{\end{array}}
\def\ben{\begin{enumerate}}
\def\een{\end{enumerate}}
\def\bse{\begin{equation*}}
\def\ese{\end{equation*}}
\def\bi{\begin{IEEEeqnarray*}}
\def\ei{\end{IEEEeqnarray*}}
\def\bit{\begin{itemize}}
\def\eit{\end{itemize}}
\def\btab{\begin{table}}
\def\etab{\end{table}}
\def\btb{\begin{tabular}}
\def\etb{\end{tabular}}
\def\v{\vspace{5pt}}
\def\a{\alpha}
\def\C{\mathbb{C}}
\def\cA{\mathcal{A}}
\def\cF{\mathcal{F}}
\def\cH{\mathcal{H}}
\def\cO{\mathcal{O}}
\def\cP{\mathcal{P}}
\def\cl{\colon}
\def\D{\mathrm{D}}
\def\d{\mathrm{d}}
\def\ds{\displaystyle}
\def\e{\mathrm{e}}
\def\eqv{\Leftrightarrow}
\def\F{\mathbb{F}}
\def\g{\gamma}
\def\ic{\mathrm{i}}
\def\img{\mathrm{im}}
\def\imp{\Rightarrow}
\def\iset{\cong_\mathrm{set}}
\def\l{\lambda}
\def\la{\langle}
\def\Mat{\mathrm{Mat}}
\def\m{\mathrm{m}}
\def\N{\mathbb{N}}
\def\ol{\overline}
\def\p{\partial}
\def\Q{\mathbb{Q}}
\def\R{\mathbb{R}}
\def\ra{\rangle}
\def\re{\Re\e}
\def\S{\Sigma}
\def\s{\sigma}
\def\se{\subseteq}
\def\sm{\setminus}
\def\ss{\subset}
\def\t{\text}
\def\ua{\nearrow}
\def\ve{\varepsilon}
\def\vn{\varnothing}
\def\wto{\rightharpoonup}
\def\Z{\mathbb{Z}}
\def\lacts{\vartriangleright}
\def\racts{\vartriangleleft}
\def\smallblackbox{\mathbin{\raisebox{0.6pt}{\scalebox{0.55}{$\blacksquare$}}}}
\def\Riem{\mathrm{Riem}}
\DeclareMathOperator*{\argmax}{arg\,max}
\DeclareMathOperator*{\argmin}{arg\,min}
\DeclareMathOperator{\Ad}{Ad}
\DeclareMathOperator{\Aut}{Aut}
\DeclareMathOperator{\ad}{ad}
\DeclareMathOperator{\Der}{Der}
\DeclareMathOperator{\End}{End}
\DeclareMathOperator{\ev}{ev}
\DeclareMathOperator{\Gr}{Gr}
\DeclareMathOperator{\Hol}{Hol}
\DeclareMathOperator{\Hom}{Hom}
\DeclareMathOperator{\hor}{hor}
\DeclareMathOperator{\id}{id}
\DeclareMathOperator{\im}{im}
\DeclareMathOperator{\preim}{preim}
\DeclareMathOperator{\proj}{proj}
\DeclareMathOperator{\sgn}{sgn}
\DeclareMathOperator{\lspan}{span}
\DeclareMathOperator{\tr}{tr}
\newcommand{\tvb}[3]{\left(\frac{\partial}{\partial {#1}^{#2}}\right)_{\negmedspace #3}}
\DeclareMathOperator{\ver}{ver}
\DeclareMathOperator{\vol}{vol}
\DeclareMathOperator{\GL}{GL}
\DeclareMathOperator*{\esup}{ess\,sup}
\def\gl{\mathfrak{gl}}
\DeclareMathOperator{\Ort}{O}
\def\ort{\mathfrak{o}}
\DeclareMathOperator{\SL}{SL}
\def\sl{\mathfrak{sl}}
\DeclareMathOperator{\SO}{SO}
\def\so{\mathfrak{so}}
\DeclareMathOperator{\SU}{SU}
\def\su{\mathfrak{su}}
\newcommand{\cibasis}[2][]{\frac{\partial #1}{\partial #2}}
\newcommand{\projmapto}{\stackrel{\pi}{\longrightarrow}}

\newcommand{\halfWedge}{\mathbin{
\begin{tikzpicture}
\draw[line cap=round,rounded corners=0.15,line width=0.4pt] (0,0) -- (0.65ex,1.4ex) -- (1.3ex,0ex);
\draw[line cap=round,rounded corners=0.1,line width=0.4pt] (0.3ex,0) -- (0.78ex,1.05ex);
\end{tikzpicture}}
}

\newcommand{\fig}[2]{
\begin{figure}[H]
    \includegraphics[scale=#2]{#1}
    \centering
\end{figure}
}

\newcommand{\Wedge}{\mathbin{
\begin{tikzpicture}
\draw[line cap=round,rounded corners=0.1,line width=0.4pt] (0,0) -- (0.65ex,1.4ex) -- (1.3ex,0ex);
\draw[line cap=round,rounded corners=0.25,line width=0.4pt] (0.3ex,0) -- (0.65ex,0.78ex) -- (1ex,0);
\end{tikzpicture}}
}

\DeclareFontFamily{U}{MnSymbolC}{}
\DeclareSymbolFont{MnSyC}{U}{MnSymbolC}{m}{n}
\DeclareMathSymbol{\diamondplus}{\mathbin}{MnSyC}{"7C}
\DeclareMathSymbol{\diamonddot}{\mathbin}{MnSyC}{"7E}
\DeclareFontShape{U}{MnSymbolC}{m}{n}{
<-6>  MnSymbolC5
<6-7>  MnSymbolC6
<7-8>  MnSymbolC7
<8-9>  MnSymbolC8
<9-10> MnSymbolC9
<10-12> MnSymbolC10
<12->   MnSymbolC12}{}

% Definition Box
\newcounter{defi}[section]\setcounter{defi}{0}
\renewcommand{\thedefi}{\arabic{chapter}.\arabic{section}.\arabic{defi}}
\newenvironment{defi}[2][]{
\refstepcounter{defi}
\ifstrempty{#1}
{\mdfsetup{
frametitle={
\tikz[baseline=(current bounding box.east),outer sep=0pt]
\node[anchor=east,rectangle,fill=blue!20]
{\strut Definition~\thedefi};}}
}
{\mdfsetup{
frametitle={
\tikz[baseline=(current bounding box.east),outer sep=0pt]
\node[anchor=east,rectangle,fill=blue!20]
{\strut Definition~\thedefi:~#1};}}
}
\mdfsetup{innertopmargin=10pt,linecolor=blue!20,
linewidth=2pt,topline=true,
frametitleaboveskip=\dimexpr-\ht\strutbox\relax
}
\begin{mdframed}[nobreak=true]\relax
\label{#2}}{\end{mdframed}}

\def\bd[#1]{\vspace{2pt} \begin{defi}[#1]{placeholder} \vspace{-9pt}}
\def\ed{\end{defi}}

% Theorem Box
\newcounter{theo}[section]\setcounter{theo}{0}
\renewcommand{\thetheo}{\arabic{chapter}.\arabic{section}.\arabic{theo}}
\newenvironment{theo}[2][]{
\refstepcounter{theo}
\ifstrempty{#1}
{\mdfsetup{
frametitle={
\tikz[baseline=(current bounding box.east),outer sep=0pt]
\node[anchor=east,rectangle,fill=green!20]
{\strut Theorem~\thetheo};}}
}
{\mdfsetup{
frametitle={
\tikz[baseline=(current bounding box.east),outer sep=0pt]
\node[anchor=east,rectangle,fill=green!20]
{\strut Thoerem~\thetheo:~#1};}}
}
\mdfsetup{innertopmargin=10pt,linecolor=green!20,
linewidth=2pt,topline=true,
frametitleaboveskip=\dimexpr-\ht\strutbox\relax
}
\begin{mdframed}[nobreak=true]\relax
\label{#2}}{\end{mdframed}}

\def\bt[#1]{\vspace{2pt} \begin{theo}[#1]{placeholder} \vspace{-9pt}}
\def\et{\end{theo}}

% Example Box
\newcounter{exa}[section]\setcounter{exa}{0}
\renewcommand{\theexa}{\arabic{chapter}.\arabic{section}.\arabic{exa}}
\newenvironment{exa}[2][]{
\refstepcounter{exa}
\ifstrempty{#1}
{\mdfsetup{
frametitle={
\tikz[baseline=(current bounding box.east),outer sep=0pt]
\node[anchor=east,rectangle,fill=black!20]
{\strut Example~\theexa};}}
}
{\mdfsetup{
frametitle={
\tikz[baseline=(current bounding box.east),outer sep=0pt]
\node[anchor=east,rectangle,fill=black!20]
{\strut Example~\theexa:~#1};}}
}
\mdfsetup{innertopmargin=10pt,linecolor=black!20,
linewidth=2pt,topline=true,
frametitleaboveskip=\dimexpr-\ht\strutbox\relax
}
\begin{mdframed}[nobreak=true]\relax
\label{#2}}{\end{mdframed}}

\def\be{\vspace{2pt} \begin{exa}{placeholder} \vspace{-9pt}}
\def\ee{\end{exa}}

% proof Box
\newcounter{proo}[section]\setcounter{proo}{0}
\renewcommand{\theproo}{\arabic{chapter}.\arabic{section}.\arabic{proo}}
\newenvironment{proo}[2][]{
\refstepcounter{proo}
\ifstrempty{#1}
{\mdfsetup{
frametitle={
\tikz[baseline=(current bounding box.east),outer sep=0pt]
\node[anchor=east,rectangle,fill=red!20]
{\strut Proof~#1};}}
}
{\mdfsetup{
frametitle={
\tikz[baseline=(current bounding box.east),outer sep=0pt]
\node[anchor=east,rectangle,fill=red!20]
{\strut Proof~:~#1};}}
}
\mdfsetup{innertopmargin=10pt,linecolor=red!20,
linewidth=2pt,topline=true,
frametitleaboveskip=\dimexpr-\ht\strutbox\relax
}
\begin{mdframed}[nobreak=true]\relax
\label{#2}}{\end{mdframed}}

\def\bq{\vspace{2pt} \begin{proo}{placeholder} \vspace{-9pt}}
\def\eq{\end{proo}}

% code in text
\definecolor{codegray}{gray}{0.9}
\newcommand{\code}[1]{\colorbox{codegray}{\texttt{#1}}}

% listings
\DeclareFixedFont{\ttm}{T1}{txtt}{m}{n}{9}

% bash
\newcommand\bashstyle{\lstset{
  language=bash,
  keywordstyle=\sffamily\ttm,
  basicstyle=\sffamily\ttm,
  numbersep=5pt,
  frame=tb,
  columns=fullflexible,
  backgroundcolor=\color{yellow!20},
  linewidth=1\linewidth,
  breaklines=true,
  commentstyle=\sffamily\ttm,
  mathescape=true,
  aboveskip=8pt,
  belowskip=8pt}}
\lstnewenvironment{bash}{\bashstyle \lstset{}}{}

% generic block
\newcommand\blockstyle{\lstset{
     basicstyle=\color{black}\footnotesize,
     rulecolor=\color{black},
     string=[s]{'}{'},
     stringstyle=\color{black},
     comment=[l]{:},
     commentstyle=\color{black},
     morecomment=[l]{-},
     frame=tb,
     backgroundcolor=\color{gray!20},
     linewidth=1\linewidth,
     breaklines=true,
     mathescape=true,
     aboveskip=8pt,
     belowskip=8pt}}
\lstnewenvironment{block}{\blockstyle\lstset{}}{}