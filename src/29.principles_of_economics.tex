\section{Introduction}

The word economy comes from the Greek word ``oikonomos'', which means ``one who manages a household''. Households and
societie, have much in common as they both face many decisions as to how allocate their scarce resources, goods, and
services among its various members.

\subsection{Goods \& Services}

\bd[Goods]
\textbf{Goods} are tangible entities that satisfy human wants and provide utility.
\ed

\bd[Services]
\textbf{Services} are intangible entities that satisfy human wants and provide utility.
\ed

\bd[Resource]
A \textbf{resource} is anything used to produce goods and services.
\ed

We group goods and services based on two characteristics: ``excludability'' and ``rivalry''.

\bd[Excludability]
\textbf{Excludability} is the degree to which goods and services can be prevented from usage.
\ed

\bd[Excludable Good/Service]
A good or service which caries the property of excludability is called an \textbf{excludable good} or an
\textbf{excludable service}.
\ed

\bd[Rivalry]
\textbf{Rivalry} is the degree to which the usage of goods and services by one person diminishes other people's use.
\ed

\bd[Rival Good/Service]
A good or service which caries the property of rivalry is called a \textbf{rival good} or a \textbf{rival service}.
\ed

Based on these two characteristics we can define four categories of goods and services: ``private'', ``public'',
``common'', and ``club''.

\bd[Private Good/Service]
A \textbf{private good} or \textbf{private service} is a good or service that is both excludable and rival.
\ed

Most goods and services in an economy are private goods. You don't get them unless you pay for them, and once you have
them, you are the only person who benefits.

\be
An example of a private good is an ice-cream. It is excludable because it is possible to prevent someone from eating
one (you just don't give it to her), and it is rival because if one person eats an ice-cream cone, another person
cannot eat the same cone.
\ee

\bd[Public Good/Service]
A \textbf{public good} or \textbf{public service} is a good or service that is neither excludable nor rival.
\ed

\be
An example of a public service is a tornado siren. Once the siren sounds, it is impossible to prevent any single person
from hearing it, so it is not excludable. Moreover, when one person gets the benefit of the warning, she does not
reduce the benefit to anyone else, so it is not rival.
\ee

\bd[Common Good/Service]
A \textbf{common good} or \textbf{common service} is a good or service that is not excludable but rival.
\ed

\be
An example of a common good is fishes in the oceans. They are rival because when one person catches some fish, fewer
fish are left for the next person to catch. But these fish are not an excludable good because it is hard to stop
fishermen from taking fish out of a vast ocean.
\ee

\bd[Club Good/Service]
A \textbf{club good} or \textbf{club service} is a good or service that is excludable but not rival.
\ed

\be
An example of a club service is satellite TV. If you don't pay the company offering the service, it can prevent you from
using it, making the good excludable. But your accessing the satellite signal does not diminish anyone else's ability
to access it, so the good is not rival.
\ee

\subsection{Scarcity}

The management of society's resources is important because resources are scarce.

\bd[Scarcity]
\textbf{Scarcity} refers to the gap between limited resources and (theoretically) limitless wants.
\ed

Scarcity, as an economic concept, refers to the basic fact of life that society has limited resources and therefore
cannot produce all the goods and services people wish to have, even with the best technical knowledge available. This
situation requires people to make decisions about how to allocate resources efficiently, in order to satisfy basic
needs and as many additional wants as possible. \v

Combining the concept of scarcity with the definition of goods and services we can come up with the notion of an
``economic good'' and an ``economic service''.

\bd[Economic Good/Service]
A good or service is an \textbf{economic good} or an \textbf{economic service}, if it is useful to people but scarce in
relation to its demand so that human effort is required to obtain it.
\ed

If the conditions of scarcity didn't exist and an infinite amount of every good and service could be produced, there
would be no economic goods and services. This situation of abundance is called ``post scarcity'', although it does
not mean that scarcity has been eliminated for all goods and services, but that all people can easily have their
basic survival needs met along with some significant proportion of their desires for goods and services.

\bd[Post Scarcity]
\textbf{Post scarcity} is a theoretical economic situation in which most goods can be produced in great abundance
with minimal human labor needed, so that they become available to all very cheaply or even freely.
\ed

From the economic goods and services we can define the concept of a commodity.

\bd[Commodity]
A \textbf{commodity} is an economic good or service, and more usually a resource, that has full or substantial
fungibility, i.e.\ instances of it are equivalent with no regard to who produced them.
\ed

Each economic good and service carries an economic value.

\bd[Economic Value]
\textbf{Economic value} is the value that a person places on an economic good or service based on the benefit that they
derive from the good or service, typically measured in units of currency.
\ed

\subsection{Assets \& Liabilities}

Based on the concept of economic value we can define the broader term of an ``asset''.

\bd[Asset]
An \textbf{asset} is any resource owned or controlled by an economic entity that can be used to produce positive
economic value.
\ed

In simple words assets are everything you've got, and they can be grouped into three major classes.

\bd[Real / Physical / Tangible Asset]
A \textbf{real asset}, or \textbf{physical asset}, or \textbf{tangible asset} is any asset that can be touched.
\ed

\be
Examples of tangible assets are real property like buildings or machinery like an iphone.
\ee

\bd[Intangible Asset]
An \textbf{intangible asset} is an asset that lacks physical substance.
\ed

\be
Examples of intangible asset are goodwill, copyrights, trademarks, patents or computer programs.
\ee

An intangible asset is usually very difficult to valuate since they suffer from both non-rivalry and non-excludability.
\v

Finaly, there is a third category in-between the first two, called ``financial assets''.

\bd[Financial Asset]
A \textbf{financial asset} is a non-physical asset whose value is derived from a contract.
\ed

\be
Examples of financial assets are stocks, bonds, bank deposits, and currencies.
\ee

One very important property of any asset is its ``liquidity''.

\bd[Liquidity]
\textbf{Liquidity} refers to the ease with which an asset can be converted into cash without affecting its market price.
\ed

In the opposite side of the spectrum we have the concept of a ``liability''.

\bd[Liability]
A \textbf{liability} is the future sacrifices of economic benefits that an entity is obliged to make to other entities
as a result of past events, the settlement of which may result in the transfer or use of assets, provision of services
or other yielding of economic benefits in the future.
\ed

In simple words liabilities are everything you owe. More on financial assets and liabilities in the next chapters.

\subsection{Economics}

Finally, we are in a position to properly define ``economics''.

\bd[Economics]
\textbf{Economics} is the science that studies the production, distribution, and consumption of economic goods and
services, and how societies make choices about how to allocate resources.
\ed

In most societies, resources are allocated not by an all-powerful dictator but through the combined choices of millions
of households and firms. Economists therefore study how people make decisions, how they interact with one another, and
how forces and trends affect the economy as a whole. \v

Since economics is a science, economists try to address their subject with a scientist's objectivity, i.e.\ by making
assumptions, devising theories, collecting data, and then analyzing these data to verify or refute their theories.
However, they are not allowed to manipulate a system under investigation in order to generate useful data, but they
usually have to make do with whatever data the world gives them. Subsequently, economists pay close attention to the
natural experiments offered by history episodes. Studying these episodes is valuable because they give us insight into
the economy of the past and allow us to illustrate and evaluate economic theories of the present. \v

Economics can generally be broken down into the field of ``microeconomics'' and ``macroeconomics'', that we will
discuss in more detail in the coming chapters.

\bd[Microeconomics]
\textbf{Microeconomics} is a branch of economics that studies the behavior of individuals and firms in making decisions
regarding the allocation of scarce resources and the interactions among these individuals and firms.
\ed

Microeconomics focuses on supply and demand and other forces that determine price levels in the economy. Through these
forces, microeconomics analyzes the relative prices among goods and services and the allocation of limited resources
among alternative uses. It also analyzes market failure, where markets fail to produce efficient results. Microeconomics
does not try to answer or explain what forces should take place in a market. Rather, it tries to explain what happens
when there are changes in certain conditions. \v

\bd[Macroeconomics]
\textbf{Macroeconomics} is a branch of economics dealing with performance,structure, behavior, and decision-making of an
economy as a whole.
\ed

Macroeconomics focuses on aggregates and econometric correlations, which is why governments and their agencies rely
on macroeconomics to formulate economic and fiscal policy. John Maynard Keynes is often credited as the founder of
macroeconomics, as he initiated the use of monetary aggregates to study broad phenomena. Some economists dispute his
theories, while many Keynesians disagree on how to interpret his work. \v

Another basic distinction of economics is between ``positive'' and ``normative'' economics.

\bd[Positive Economics]
\textbf{Positive economics} refers to the objective analysis in the study of economics.
\ed

Positive statements are descriptive. They make a claim about how the world is. Most economists look at what has happened
and what is currently happening in a given economy to form their basis of predictions for the future.

\bd[Normative Economics]
\textbf{Normative economics} is a perspective on economics that reflects normative, or ideologically prescriptive
judgments toward economic development, investment projects, statements, and scenarios.
\ed

Normative statements are prescriptive. They make a claim about how the world ought to be. Unlike positive economics,
which relies on objective data analysis, normative economics heavily concerns itself with value judgments and statements
of ``what ought to be'' rather than facts based on cause-and-effect statements. \v

Positive and normative statements are fundamentally different, but within a person's set of beliefs, they are often
intertwined. In particular, positive views about how the world works affect normative views about what policies are
desirable. \v

A key difference between positive and normative statements is how we judge their validity. We can, in principle,
confirm or refute positive statements by examining evidence. By contrast, normative economics involves ideological
judgments about what may result in economic activity if public policy changes are made, hence evaluating normative
statements involves values as well as facts. Deciding what is good or bad policy is not just a matter of science. It
also involves our views on ethics, religion, and political philosophy. As a result, most of the time normative
economic statements can't be verified or tested.

\section{The Ten Principles Of Economics}

The study of economics has many facets, but it is unified by several central ideas. In this chapter, we look at ``Ten
Principles of Economics''.

\subsection*{Principle 1: People Face Trade-offs}

To get something that we like, we usually have to give up something else that we also like. There is no such thing as
a free lunch (TINSTAAFL). Making decisions requires trading one goal for another.

\bd[Trade-off]
A \textbf{trade-off} is a situational decision that involves diminishing or losing one quality, quantity, or property
of a set or design in return for gains in other aspects.
\ed

In simple terms, a trade-off is where one thing increases, and another must decrease. A trade-off, then, involves a
sacrifice that must be made to obtain a certain product, service, or experience, rather than others that could be
made or obtained using the same required resources. \v

A very fundamental example of a trade-off, is the trade-off between (economic) efficiency and (economic) equality.

\bd[Economic Efficiency]
\textbf{Economic efficiency} is when all goods and services of production in an economy are distributed or allocated
to their most valuable uses and waste is eliminated or minimized.
\ed

\bd[Economic Equality]
\textbf{Economic equality} is the property of distributing economic prosperity fairly among the members of society.
\ed

\bd[Equality-Efficiency Trade-off]
\textbf{Equality-efficiency trade-off} refers to the conflict of maximizing economic efficiency while maximizing
economic equality of society at the same time.
\ed

Recognizing that people face trade-offs does not by itself tell us what decisions they will or should make. Nonetheless,
people are likely to make good decisions only if they understand the options available to them. Our study of economics,
therefore, starts by acknowledging life's trade-offs.

\subsection*{Principle 2: The Cost Of Something Is What You Give Up To Get It}

Because people face trade-offs, making decisions requires comparing the costs and benefits of alternative courses of
action. In many cases, however, the cost of an action is not as obvious as it might first appear. \v

In economics a trade-off is expressed in terms of the ``opportunity cost'' of a particular choice, which is the loss of
the most preferred alternative given up.

\bd[Opportunity Cost]
\textbf{Opportunity costs} represent the potential benefits an individual, investor, or business misses out on when
choosing one alternative over another.
\ed

The opportunity cost of an item is what you give up to get that item. When making any decision, decision makers should
take into account the opportunity costs of each possible action. \v

Because opportunity costs are, by definition, unseen, they can be easily overlooked. Understanding the potential
missed opportunities when a business or individual chooses one investment over another allows for better decision-making.

\subsection*{Principle 3: Rational People Think At The Margin}

Economics focuses on the actions of human beings, based on assumptions that humans act with rational behavior, seeking
the most optimal level of benefit or utility.

\bd[Rational Behavior]
\textbf{Rational behavior} refers to a decision-making process that is based on making choices that result in the
optimal level of benefit or utility for an individual.
\ed

The assumption of rational behavior implies that people would rather take actions that benefit them versus actions
that are neutral or harm them. Most classical economic theories are based on the assumption that all individuals
taking part in an activity are behaving rationally. \v

Rational people systematically do the best they can to achieve their objectives, given the available opportunities,
knowing that decisions in life are rarely black and white but often involve shades of gray. In economics, this
concept is being expressed through the notion of ``marginal change''.

\bd[Marginal Change]
\textbf{Marginal change} is used to describe a small incremental adjustment to an existing plan of action.
\ed

Margin means edge, so marginal changes are adjustments around the edges of what you are doing. Rational people make
decisions by comparing marginal benefits and costs.

\bd[Marginal Benefit]
A \textbf{marginal benefit} is the maximum amount a consumer is willing to pay for an additional unit of a good or a
service which will bring an additional unit of satisfaction or utility.
\ed

\bd[Marginal Cost]
A \textbf{marginal cost} is the small, but measurable, amount a producer has to pay for producing an additional unit of
a good or service.
\ed

The marginal benefit for a consumer tends to decrease as consumption of the good or service increases. If the additional
satisfaction obtained by an addition in the units of a good or a service is equal to the price a consumer is willing
to pay for that good or service, he achieves maximum satisfaction, which is the main goal of every rational consumer. A
rational decision maker takes an action if and only if the action's marginal benefit exceeds its marginal cost.

\be
Marginal decision-making can explain some otherwise puzzling phenomena, such as one of the most famous paradoxes, the
so called ``paradox of value'', also known as the ``diamond–water paradox'': why is water so cheap, while diamonds
are so expensive? \v

Humans need water to survive, while diamonds are unnecessary. Yet people are willing to pay much more for a diamond
than for a cup of water. The reason is that a person's willingness to pay for a good is based on the marginal benefit
that an extra unit of the good would yield. The marginal benefit, in turn, depends on how many units a person already
has. Water is essential, but the marginal benefit of an extra cup is small because water is plentiful. By contrast,
no one needs diamonds to survive, but because diamonds are so rare, the marginal benefit of an extra diamond is large.
\ee

As we will see in later chapters, while the principle (or assumption) that people are rational is a very powerful one,
it is also a not a perfect one. People are not always rational, and they don't always think at the margin.

\subsection*{Principle 4: People Respond To Incentives}

Because rational people make decisions by comparing marginal costs and benefits, they respond to incentives.

\bd[Incentive]
\textbf{Incentive} is a positive or negative intention that induces a person to act in a certain way.
\ed

Incentives play a central role in the study of economics, since they are key to analyzing how markets work. Public
policymakers should never forget about incentives: many policies change the costs or benefits that people face and,
as a result, alter their behavior. When policymakers fail to consider how their policies affect incentives, they often
face unintended consequences. \v

The first four principles discussed how individuals make decisions. The next three principles concern how people
interact with one another.

\subsection*{Principle 5: Trade Can Make Everyone Better Off}

There are two ways to compare the abilities of two people to produce a good. The first one is the so called ``absolute
advantage''.

\bd[Absolute Advantage]
\textbf{Absolute advantage} is the ability of an entity to produce a greater quantity of a good or a service with the
same quantity of inputs per unit of time, or to produce the same quantity of a good or a service per unit of time using
a lesser quantity of inputs, than its competitors.
\ed

The entity which can produce the good or the service with the smaller quantity of inputs is said to have an absolute
advantage in producing the good. Absolute advantage can be accomplished by creating the good or service at a lower
absolute cost per unit by using a smaller number of inputs, or by a more efficient process.

\bd[Unit Cost]
\textbf{Unit cost} is the total expenditure incurred by a company to produce, store, and sell one unit of a particular
good or service.
\ed

The second one is the so called ``comparative advantage''.

\bd[Comparative Advantage]
\textbf{Comparative advantage} is an economy's ability to produce a particular good or service at a lower opportunity
cost than its trading partners.
\ed

The entity which has the lower opportunity cost of producing the good is said to have a comparative advantage.
Comparative advantage is used to explain why entities can benefit from trading, since the gains from trade are based on
comparative advantage, not absolute advantage.

\bd[Trade]
\textbf{Trade} is a basic economic concept involving the buying and selling of goods and services, with compensation
paid by a buyer to a seller, or the exchange of goods or services between parties.
\ed

Trade takes place within an economy between producers and consumers.

\bd[Producer / Seller]
A \textbf{producer}, or \textbf{seller}, is someone who creates and supplies goods or services.
\ed

\bd[Consumer / Buyer]
A \textbf{consumer}, or \textbf{buyer}, is someone who acquires goods or services.
\ed

Trade can also take place between countries in international trade. International trade allows countries to expand
markets for both goods and services that otherwise may not have been available. As a result the market contains
greater competition and therefore, more competitive prices, which brings a cheaper product home to the consumer. \v

In general trade makes everyone better off because it allows people to specialize in those activities in which they
have a comparative advantage. In a broader scope, trade allows for specialization in products that benefits countries.
Interdependence and trade are desirable because they allow everyone to enjoy a greater quantity and variety of
goods and services. For centuries, it was widely believed that in international trade one country's gain from an
exchange must be the other country's loss. However, trade is not like a sports competition, where one side gains and
the other side loses.

\subsection*{Principle 6: Markets Are Usually A Good Way To Organize Economic Activity}

\bd[Economic System]
An \textbf{economic system} is a system of production, resource allocation, and distribution of goods and services
within a society.
\ed

An economic system is a type of social system which confront and solve the four fundamental economic problems: what
kinds and quantities of goods and services shall be produced, how shall be produced, when to be produced, and how
they will be distributed. It includes the combination of the various institutions, agencies, entities, decision-making
processes and patterns of consumption that comprise the economic structure of a given community. \v

There are many different economic systems. In what follows we will mention the most important ones.

\bd[Centrally Planned Economy / Command Economy]
A \textbf{centrally planned economy}, or \textbf{command economy}, is an economic system in which a central authority,
makes economic decisions regarding the manufacturing and the distribution of products.
\ed

More often than not, the central authority of a centrally planned economy is the government. The central planners
(bureaucrats) decide what goods and services are produced and in what quantities, and they determine the prices at
which the goods and services are sold. In a centrally planned economy, individuals have little influence over how the
basic economic questions are answered. \v

Centrally planned economies have failed because they did not allow the market to work. Most countries that once had a
centrally planned economy, have abandoned the system and instead have adopted market economies in which such decisions
are traditionally made by businesses and consumers inside the market. \v

\bd[Market]
A\textbf{market} is a place where producers and consumers can gather to facilitate the exchange of goods and services.
\ed

A market may be physical like a retail outlet, where people meet face-to-face, or virtual like an online market, where
there is no direct physical contact between buyers and sellers. More on markets in the next chapter.

\bd[Market Economy]
A \textbf{market economy} is an economic system in which economic decisions regarding the manufacturing and the
distribution of products are guided by the price signals created by the forces of the market.
\ed

In a market economy, the decisions of a central planner are replaced by the decisions of millions of individual
citizens, businesses, and households. Firms decide whom to hire and what to make. Households decide which firms to
work for and what to buy with their incomes. These firms and households interact in the market, where prices and
self-interest guide their decisions. Because a market economy rewards people for their ability to produce things that
other people are willing to pay for, there will be an unequal distribution of economic prosperity. \v

Market economies range from minimally regulated free market systems where state activity is restricted to providing
public goods and services and safeguarding private ownership, to interventionist forms where the government plays an
active role in serving special interests and promoting social welfare. \v

At first glance, the success of market economies is puzzling. In a market economy, no one is looking out for the
well-being of society as a whole. Free markets contain many buyers and sellers of numerous goods and services, and
all of them are interested primarily in their own well-being. Yet despite decentralized decision-making and
self-interested decision makers, market economies have proven remarkably successful in organizing economic activity
to promote overall prosperity. Adam Smith's 1776 work suggested that although individuals are motivated by
self-interest, an ``invisible hand'' guides this self-interest into promoting society's economic well-being.

\bd[Invisible Hand]
The \textbf{invisible hand} is a metaphor for the unseen forces that move the free market economy. Through individual
self-interest and freedom of production and consumption, the best interest of society, as a whole, are fulfilled.
\ed

\subsection*{Principle 7: Governments Can Sometimes Improve Market Outcomes}

There are two broad reasons for the government to interfere with the economy. First reason we need government is that
the invisible hand can work its magic only if the government enforces the rules and maintains the institutions that
are key to a market economy. Most important, market economies need institutions to enforce property rights so
individuals can own and control scarce resources.

\bd[Property Rights]

\textbf{Property rights} define the theoretical and legal ownership of tangible or intangible resources owned by
individuals, businesses, and governments and how they can be used.
\ed

The second reason we need government is that, although the invisible hand is powerful, it is not omnipotent. There
are two broad rationales for a government to intervene in the economy and change the allocation of resources that
people would choose on their own: to promote efficiency or to promote equality. That is, most policies aim either to
enlarge the economic pie or to change how the pie is divided. \v

Consider first the goal of efficiency. Although the invisible hand usually leads markets to allocate resources to
maximize the size of the economic pie, this is not always the case. Economists use the term market failure to refer
to a situation in which the market on its own fails to produce an efficient allocation of resources.

\bd[Market Failure]
\textbf{Market failure} is a situation defined by an inefficient distribution of goods and services in the free
market.
\ed

In market failure, the individual incentives for rational behavior do not lead to rational outcomes for the group. In
other words, in market failure, each individual makes the correct decision for themselves, but those prove to be the
wrong decisions for the group. In traditional microeconomics, this can sometimes be shown as a steady-state
disequilibrium in which the quantity supplied does not equal the quantity demanded. One possible cause of market
failure is an externality, which is the impact of one person's actions on the well-being of a bystander.

\bd[Externality]
An \textbf{externality} is a cost or benefit caused by a producer that is not financially incurred or received by that
producer.
\ed

An externality can be both positive or negative and can stem from either the production or consumption of a good or
service. The costs and benefits can be both private—to an individual or an organization—or social, meaning it can
affect society as a whole. Externalities by nature are generally environmental, such as natural resources or public
health.

\be
The classic example of an externality is pollution. When the production of a good pollutes the air and creates health
problems for those who live near the factories, the market on its own may fail to take this cost into account.
\ee

Another possible cause of market failure is market power, which refers to the ability of a single person or firm (or
a small group of them) to unduly influence market prices.

\bd[Market Power]
\textbf{Market power} refers to a single person's or company's relative ability to manipulate the price of an item in
the marketp by manipulating the level of supply, demand or both.
\ed

A company with substantial market power has the ability to manipulate the market price and thereby control its profit
margin, and possibly the ability to increase obstacles to potential new entrants into the market. Firms that have
market power are often described as ``price makers'' because they can establish or adjust the marketplace price of an
item without relinquishing market share.

\be
If everyone in town needs water but there is only one well, the owner of the well does not face the rigorous
competition with which the invisible hand normally keeps self-interest in check.
\ee

In the presence of externalities or market power, well-designed public policy can enhance economic efficiency. \v

Now consider the goal of equality. Even when the invisible hand yields efficient outcomes, it can nonetheless leave
sizable disparities in economic well-being. A market economy rewards people according to their ability to produce
things that other people are willing to pay for.

\be
The world's best basketball player earns more than the world's best chess player simply because people are willing to
pay more to watch basketball than chess.
\ee

The invisible hand does not ensure that everyone has sufficient food, decent clothing, and adequate healthcare. This
inequality may, depending on one's political philosophy, call for government intervention. In practice, many public
policies, such as the income tax and the welfare system, aim to achieve a more equal distribution of economic
well-being. \v

To say that the government can improve market outcomes does not mean that it always will. Public policy is made not
by angels but by a political process that is far from perfect. Sometimes policies are designed to reward the
politically powerful. Sometimes they are made by well-intentioned leaders who are not fully informed. As you study
economics, you will become a better judge of when a government policy is justifiable because it promotes efficiency
or equality and when it is not. \v

We started by discussing how individuals make decisions and then looked at how people interact with one another. All
these decisions and interactions together make up ``the economy''. The last three principles concern the workings of
the economy as a whole.

\subsection*{Principle 8: A Country's Standard Of Living Depends On Its Ability To Produce Goods And Services}

Differences in the standard of living from one country to another are quite large. Changes in living standards over
time are also quite large. The explanation for differences in living standards lies in differences in productivity.

\bd[Productivity]
\textbf{Productivity} is the quantity of goods and services produced by each unit of labor input.
\ed

Measurements of productivity are often expressed as a ratio of an aggregate output to a single input or an aggregate
input used in a production process. \v

High productivity implies a high standard of living. The relationship between productivity and living standards is
simple, but its implications are far-reaching. If productivity is the primary determinant of living standards, other
explanations must be less important. \v

The relationship between productivity and living standards also has profound implications for public policy. When
thinking about how any policy will affect living standards, the key question is how it will affect our ability to
produce goods and services. To boost living standards, policymakers need to raise productivity by ensuring that
workers are well-educated, have the tools they need to produce goods and services, and have access to the best
available technology.

\subsection*{Principle 9: Prices Rise When The Government Prints Too Much Money}

\bd[Purchasing Power]
\textbf{Purchasing power} is the value of a currency expressed in terms of the number of goods or services that one unit
of money can buy.
\ed

Purchasing power is important because, all else being equal, inflation decreases the number of goods or services you
would be able to purchase.

\bd[Inflation]
\textbf{Inflation} is the decline of purchasing power of a given currency over time.
\ed

A quantitative estimate of the rate at which the decline in purchasing power occurs can be reflected in the increase
of an average price level of a basket of selected goods and services in an economy over some period of time. The rise
in the general level of prices, often expressed as a percentage, means that a unit of currency effectively buys less
than it did in prior periods. \v

What causes inflation? In almost all cases of large or persistent inflation, the culprit is growth in the quantity of
money. When a government creates large quantities of the nation's money, the value of the money falls, because the
individuals have more money and the demand increases. When the demand increases, price also increases and creates
inflation of money.

\subsection*{Principle 10: Society Faces A Tradeoff Between Inflation And Unemployment}

\bd[Monetary Authority / Central Bank]
A \textbf{monetary authority}, or \textbf{central bank}, is the entity that manages a country's currency and money
supply, with the objective of controlling inflation, interest rates, and unemployment rate.
\ed

With its monetary tools, a monetary authority is able to effectively influence parameters which control the cost and
availability of money (like interest rates). In contrast to a commercial bank, a central bank possesses a monopoly on
increasing the monetary base.

\bd[Financial Institution]
A \textbf{financial institution} is a company engaged in the business of dealing with financial and monetary
transactions such as deposits, loans, investments, and currency exchange.
\ed

Financial institutions encompass a broad range of business operations within the financial services sector including
banks, trust companies, insurance companies, brokerage firms, and investment dealers. Virtually everyone living in a
developed economy has an ongoing or at least periodic need for the services of financial institutions.

\bd[Commercial Bank]
A \textbf{commercial bank} is a financial institution which accepts deposits from the public and gives loans for the
purposes of consumption and investment to make profit.
\ed

\bd[Monetary Policy]
\textbf{Monetary policy} is the policy adopted by the monetary authority of a nation to control either the interest
rate payable for very short-term borrowing (borrowing by banks from each other to meet their short-term needs) or the
money supply, often as an attempt to reduce inflation or the interest rate, to ensure price stability and general
trust of the value and stability of the nation's currency.
\ed

The goal of a monetary policy is to keep the economy humming along at a rate that is neither too hot nor too cold.
The central bank may force up interest rates on borrowing in order to discourage spending or force down interest
rates to inspire more borrowing and spending. The main weapon at its disposal is the nation's money. The central bank
sets the rates it charges to loan money to the nation's banks. When it raises or lowers its rates, all financial
institutions tweak the rates they charge all of their customers, from big businesses borrowing for major projects to
home buyers applying for mortgages. \v

While an increase in the quantity of money primarily raises prices in the long run, the short-run story is more
complex. In short run increasing the amount of money in the economy stimulates the overall level of spending and thus
the demand for goods and services. Higher demand may over time cause firms to raise their prices, but in the
meantime, it also encourages them to hire more workers and produce a larger quantity of goods and services. Finally,
more hiring means lower unemployment. This line of reasoning leads to one final economy-wide trade-off: a short-run
trade-off between inflation and unemployment. \v

Although some economists still question these ideas, most accept that society faces a short-run trade-off between
inflation and unemployment. This simply means that, over a period of a year or two, many economic policies push
inflation and unemployment in opposite directions. Policymakers face this trade-off regardless of whether inflation
and unemployment both start out at high levels (as they did in the early 1980s), at low levels (as they did in the
late 1990s), or someplace in between. This short-run trade-off plays a key role in the analysis of the business
cycle—the irregular and largely unpredictable fluctuations in economic activity, as measured by the production of
goods and services or the number of people employed.

\bd[Business Cycles]
\textbf{Business cycles} are intervals of expansion followed by recession in economic activity.
\ed

Policymakers can exploit the short-run trade-off between inflation and unemployment using various policy instruments.
By changing the amount that the government spends, the amount it taxes, and the amount of money it prints,
policymakers can influence the overall demand for goods and services. Changes in demand in turn influence the
combination of inflation and unemployment that the economy experiences in the short run. Because these instruments of
economic policy are so powerful, how policymakers should use them to control the economy, if at all, is a subject of
continuing debate.