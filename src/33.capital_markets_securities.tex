%! suppress = EscapeUnderscore
\section{Financial Markets}

\bd[Financial Asset]
A \textbf{financial asset} is a non-physical asset whose value is derived from a contract.
\ed

The contract that specifies the value of the financial asset is called a ``financial instrument''.

\bd[Financial Instrument]
A \textbf{financial instrument} is a contract that gives rise to a financial asset.
\ed

Sometimes the terms ``financial asset'' and ``financial instrument'' are used interchangeably, but it is important to
understand the difference between them. A financial asset is the result of a financial instrument, and a financial
instrument is a contract that gives rise to a financial asset. The financial instrument is the cause and the financial
asset is the effect. \v

Having defined financial instruments, we can now define a very special kind of market called a ``financial market''.

\bd[Financial Market]
A \textbf{financial market} is a market where people trade financial instruments at low transaction costs and at prices
that reflect supply and demand.
\ed

Depending on the trend of the prices of the financial instruments in the financial market, we can distinguish between
``normal`` or `bull markets'' and ``inverted'' or ``bear markets''.

\bd[Normal / Bull Market]
A \textbf{normal}, or \textbf{bull market}, is a market where the prices of financial instruments are rising.
\ed

\bd[Inverted / Bear Market]
An \textbf{inverted}, or \textbf{bear market}, is a market where the prices of financial instruments are falling.
\ed

We call people who buy and sell financial instruments in financial markets ``investors''.

\bd[Investor]
An \textbf{investor} is a person or entity that commits capital in financial markets with the expectation of receiving
financial returns generated by financial instruments.
\ed

In traditional finance investors are considered to be just households, however in modern finance an investor can be any
institution who has capital to invest including pension funds investing group retirement plans, insurance companies
investing premiums, investment companies like mutual funds and unit trusts, sovereign wealth funds, endowments like
universities and non-profit oranizations, asset managers, and hedge funds. Its worths mentioning though that all these
institutional investors are ultimately investing the savings of households, hence in a way they act in behalf of the
households. \v

At this point it is important to make a distinction between an investor and a ``trader''.

\bd[Trader]
A \textbf{trader} is a person or entity that buys and sells financial instruments with the intention of making a profit.
\ed

The main difference between an investor and a trader is that an investor buys and holds financial instruments for the
long term while a trader buys and sells securities for the short term with the intention of making a profit. \v

There are three broad categories of traders: ``hedgers'', ``speculators'', and ``arbitrageurs''.

\bd[Hedger]
A \textbf{hedger} is a trader who uses financial instruments to reduce the risk that they face from potential future
movements in a market variable.
\ed

\bd[Speculator]
A \textbf{speculator} is a trader who uses financial instruments to bet on the future direction of a market variable.
\ed

\bd[Arbitrageur]
An \textbf{arbitrageur} is a trader who takes offsetting positions in two or more financial instruments to lock in a
profit.
\ed

The fact that financial instruments can be used for hedging, speculation, and arbitrage, makes them very versatile
instruments. Sometimes traders who have a mandate to hedge risks or follow an arbitrage strategy become (consciously
or unconsciously) speculators. The results can be disastrous, so it is very important for both financial and
nonfinancial corporations to set up controls to ensure that financial instruments are being used for their intended
purpose. Risk limits should be set and the activities of traders should be monitored daily to ensure that these risk
limits are adhered to. Unfortunately, even when traders follow the risk limits that have been specified, big mistakes
can happen. \v

Financial markets can be divided into two categories: ``money markets'' and ``capital markets''. In these notes we will
not be discussing money markets at all, but we will be focusing on capital markets.

\subsection{Capital Markets}

\bd[Capital Market]
A \textbf{capital market} is a financial market in which long-term (a year or more) financial instruments are traded.
\ed

The 4 main economic functions of capital markets are:
\bit
\item \textbf{Wealth Allocation} To facilitate the raising of capital by channeling wealth and allocating savings into
investments.
\item \textbf{Liquidity}: To allow the flexibility in timing consumption by providing liquidity of buying and selling
financial instruments.
\item \textbf{Risk Management}: To provide a mechanism for risk management including risk transfer, risk reduction
through diversification, and optimal risk sharing.
\item \textbf{Pricing}: To provide a mechanism for price discovery, i.e.\ the process of determining the price of an
asset in the market.
\eit

Any time one engages in a capital market transaction they're altering the risk profile of their asset portfolios.
Subsequently, another way of thinking about capital markets is not necessarily the movement of capital, but the
alteration of those risks. \v

The most important financial instruments that are traded in capital markets are:
\bit
\item \textbf{Securities}: Financial instruments that represent ownership in a corporation or a creditor relationship
with a governmental body or a corporation.
\item \textbf{Derivatives}: Financial instruments that derive their value from the performance of an underlying asset,
index, or interest rate.
\eit

In this chapter we will be focusing on securities, and in the next chapter we will be focusing on derivatives.

\section{Securities}

\bd[Security]
A \textbf{security} is a financial instrument that represents an ownership in a corporation or a creditor relationship
with a governmental body or a corporation.
\ed

\bd[Securities Market]
A \textbf{securities market} is a submarket of capital markets where people trade securities.
\ed

The main function of security markets is wealth allocation, i.e.\ to facilitate the transfer of funds from those who
have excess funds to those who need funds, by providing a mechanism for the buying and selling of securities.

\subsection{Participants Of Capital Markets}

The 3 main participants of capital markets are: ``investors'', ``issuers'', and ``financial intermediaries''. Although
we have already provided the definition of an investor, we will provide it again in the context of capital markets for
the sake of completeness.

\bd[Investor]
An \textbf{investor} is a person or entity that commits capital in capital markets with the expectation of receiving
financial returns generated by financial instruments.
\ed

\bd[Issuer]
An \textbf{issuer} is a legal entity that develops, registers and sells financial instruments for the purpose of
financing its operations.
\ed

In traditional finance issuers are considered to be just corporations, however in modern finance an issuer can be any
entity who needs to raise capital, including governments and governments agencies, municipalities, and supranational
entities. \v

Although issuers are the one who issue financial instruments, most of the time they do not sell them directly to
investors. Instead, they hire financial intermediaries between them and the investors.

\bd[Financial Intermediary]
A \textbf{financial intermediary} is an institution that acts as the middleman between investors and issuers.
\ed

The main function of financial intermediaries is to facilitate the transfer of funds from those who have excess funds
to those who need funds, by providing a mechanism for buying and selling of financial instruments.

\vspace{5pt}
\fig{cm}{0.5}
\vspace{5pt}

The most important financial intermediaries are comercial and investment banks.

\bd[Commercial Bank]
A \textbf{commercial bank} is a financial institution that provides deposit taking and commercial lending services.
\ed

\bd[Investment Bank]
An \textbf{investment bank} is a financial institution that provides financial instruments trading, financial
instruments underwriting and corporate advisory services.
\ed

Althoug there is a clear distinction between commercial and investment banks, in practice the distinction is not always
clear. In the past, commercial banks were not allowed to engage in investment banking activities, and investment banks
were not allowed to engage in commercial banking activities. However, in recent years, the distinction between the two
types of banks has become less clear, finding the two types of banks competing for similar services, and many providing
both types of services.

\subsection{Primary Market}

Securities markets are divided into two main types of ``submarkets'': the primary market and the secondary market.

\bd[Primary Market]
A \textbf{primary market} is the part of a security market where new securities are issued and sold for the first time.
\ed

New securities are issued and sold for the first time through initial public offerings (IPOs) or private placements.

\bd[Initial Public Offering (IPO)]
An \textbf{initial public offering} (\textbf{IPO}) is the first time that a company's securities are offered to the
public.
\ed

\bd[Private Placement]
A \textbf{private placement} is a sale of securities to pre-selected investors and institutions rather than on the
open market.
\ed

\subsection{Secondary Market}

\bd[Secondary Market]
A \textbf{secondary market} is the part of a security market where existing securities are bought and sold.
\ed

Secondary market is the trading of already outstanding securities. For the most part, they are non issuer
transactions. That is, the trade takes place between two different investors that are exchanging money for securities,
but it doesn't change the number of securities outstanding, nor does it provide the issuer with any additional
capital for investment. In simple words, there's no new money being raised and no wealth being channeled into
investment in the secondary market, however secondary markets are much larger than the primary markets, with the vast
majority of securities transactions (more than 95\%) taking place in the secondary market. \v

Although there are no issuers in secondary markets, there are financial intermediaries. The two primary roles of
financial intermediaries in the secondary market are to act either as ``market makers'' (also called ``dealers'') or
as ``brokers''.

\bd[Market Maker / Dealer]
A \textbf{market maker} or \textbf{dealer} is a financial intermediary that acts as principal by standing ready to
trade securities at all times at the publicly quoted price.
\ed

In simple words, a market maker is a financial intermediary in the business of trading securities for their own
account, and not on behalf of clients. Market makers take the opposite side of a trade from the customer, i.e.\ they
buy when a customer wants to sell and sell when a customer wants to buy. As it makes sense, this makes market makers
to have some capital in risk, due to the security inventory they have, and they need to be compensated for that. In
order to do so, market makers quote their own prices on securities both for buying (called ``bid price'') and for
selling (called ``ask price'' or ``offer price'').

\bd[Bid Price]
The \textbf{bid price} is the price at which a market maker is willing to buy a security.
\ed

\bd[Ask Price / Offer Price]
The \textbf{ask price}, or \textbf{offer price}, is the price at which a market maker is willing to sell a security.
\ed

It is important to note that both bid and ask prices do not reflect the true value of a security, but rather the market
maker's opinion of the security's value. The difference between the bid and ask price is called the ``spread'' and it
is the market maker's profit. The wider the spread, the more money the market maker makes.

\bd[Spread]
The \textbf{spread} is the difference between the ask price and the bid price:
\bse
\text{Spread} = \text{Ask Price} - \text{Bid Price}
\ese
\ed

The spread of a security is a measure of the liquidity of the market for that security. The smaller the spread, the
more liquid the market for that security. \v

The middle point between bid and ask prices is called the ``market price''.

\bd[Market Price]
The \textbf{market price} is the middle point between the bid and ask prices:
\bse
\text{Market Price} = \frac{\text{Bid Price} + \text{Ask Price}}{2}
\ese
\ed

The market price is the price at which a security is currently being traded in the secondary market. However, at the
market price, no one is able to neither buy nor sell a security. The bid and ask prices are the net prices at which a
market maker is willing to trade a security and that's the price that an investor will pay to buy a security or receive
to sell a security. \v

The difference between the market price and the bid and ask prices are called the ``mark-up'' and ``mark-down''.

\bd[Mark-Up]
A \textbf{mark-up} is the amount by which the ask price exceeds the market price:
\bse
\text{Mark-Up} = \text{Ask Price} - \text{Market Price}
\ese
\ed

\bd[Mark-Down]
A \textbf{mark-down} is the amount by which the bid price is less than the market price:
\bse
\text{Mark-Down} = \text{Market Price} - \text{Bid Price}
\ese
\ed

Saying the same thing in different words, market makers are eager to sell a security to an investor at market price
plus the mark-up, or buy a security from an investor at market price minus the mark-down. Mark-ups and mark-downs are
the profits market makers from selling and buying the securities. One can easily show that the spread is equal to the
sum of the mark-up and the mark-down:
\bse
\text{Spread} = \text{Mark-Up} + \text{Mark-Down}
\ese

Moving on, the second role of financial intermediaries in the secondary market is to act as brokers.

\bd[Broker]
A \textbf{broker} is a financial intermediary that brings buyers and sellers together but does not maintain an
inventory of securities.
\ed

A broker does not have capital at risk, and does not make money from the spread. Instead, brokers make money by
charging a commission for their services including compensation for time, effort and expenses of executing trade. The
commission is a fee that is charged for executing a trade. The commission can be a fixed amount, a percentage of the
trade, or a combination of both. \v

When an individual is buying a security via a broker, they will be given the choice of using two different kinds of
orders: a ``market order'' or a ``limit order''.

\bd[Market Order]
A \textbf{market order} is an order to buy or sell a security at the best available price.
\ed

A market order tells the broker to get into the security as quickly as possible, regardless of price.

\bd[Limit Order]
A \textbf{limit order} is an order to buy or sell a security at a specific price or better.
\ed

A limit order tells the broker to get into the security only at a specific price or better. If there is never a seller
at that price, the order will never be filled. \v

When one places an order to buy or sell a security, they will have one more choice to make between a ``day order'' or
a ``GTC order''.

\bd[Day Order]
A \textbf{day order} is an order that will only be executed during regular market hours today.
\ed

A day order will only be executed during regular market hours today. If the order has not been filled by the time the
market closes for the day, it will be automatically cancelled by the broker.

\bd[GTC Order]
A \textbf{GTC order} (Good 'Til Cancelled) is an order that will be good for today's market hours, as well as the
following days and weeks.
\ed

A GTC order will be good for today's market hours, as well as the following days and weeks. If the order is not
cancelled, it will still be working. Some brokers will automatically cancel a GTC order after a month or more, if it
has not yet been filled. \v

Secondary markets and their financial intermediaries provide several really important functions with the most important
one being liquidity. The fact that market makers make sure that securities can be bought and sold in the secondary
market provides investors with the ability to sell their securities and convert them into cash. This is important
because it means that investors can invest in securities without having to hold them until they mature. \v

Another important function of secondary markets is price discovery. The price of a security in the secondary market
provides information about the value of the security through the supply and demand for the security. This is important
because primary markets price things only once, at the time of the offering, and the price is set by the issuer,
without any input from the market and without having the possibility to change it, in a continuous changing world. \v

Up to this point we have discussed some of the generic characteristics of securities and the markets in which they are
traded, including the market participants and their roles, and the distinction between the primary and secondary market.
We will now switch gears and start discussing the two main types of securities.
\bit
\item \textbf{Debt / Fixed-Income Securities}: Securities that represent a creditor relationship with a governmental
body or a corporation.
\item \textbf{Equity Securities}: Securities that represent ownership in a corporation.
\eit

Let's start with debt/fixed-income securities.

\section{Debt / Fixed-Income Securities}

\bd[Debt / Fixed-Income Security]
A \textbf{debt security}, or \textbf{fixed-income security}, is a security that represents a creditor relationship with
a governmental body or a corporation.
\ed

In simple words, fixed-income securities are financial instruments that represent a loan made by an investor to the
issuer, making the investor a creditor of the issuer. The issuer of a fixed-income security is obligated to pay the
investor interest (hence the name ``fixed-income'') and to repay the initial loan amount at a later date, as it is
specified in the terms of repayment of the security.

\bd[Debt / Fixed-Income Markets]
A \textbf{debt market}, or \textbf{fixed-income market}, is a market where people trade debt securities.
\ed

Fixed-income markets are usually double or triple the size of equity markets, and are considered to be the most
important type of securities market. There are many different types of debt securities, but by far, the most important
ones are the so-called ``bonds''.

\subsection{Bonds}

\bd[Bond]
A \textbf{bond} is a fixed-income security that represents a loan made by an investor (lender) to an issuer (borrower),
usually a goverment, a govermental agency or a corporate.
\ed

Bonds are the most important type of fixed-income securities. In fact, they are so important that the terms ``debt
security'', ``fixed-income security'', and ``bond'' are often used interchangeably, despite the fact that technically
they are not the same thing.

\subsubsection{Goverment Bonds}

Bond can be issued by a variety of entities, with the two most important ones being governments and corporations.
Starting with the first ones, government and govermental agencies bonds are usually called ``sovereign debts''.

\bd[Sovereign Debt \ Government Bond]
A \textbf{sovereign debt}, or \textbf{goverment bond}, is a bond issued by a national government in a foreign currency,
in order to finance the country's growth and development.
\ed

Government bonds are usually the largest segment in any national market, and the most liquid part of the fixed-income
market. When it comes to credit worthiness, they are considered to be the safest type of fixed-income securities. In
fact, government bonds are considered to be ``risk-free'' securities, since they are backed by the full faith and
credit of the government (or a nation), however one has to keep in mind that on one hand this risk is mainly related to
the default credit risk of a bond (more on this later), and on the other hand, the risk is not really zero, since there
are countries that have defaulted on their debt in the past. Nevertheless, the risk is considered to be so low that it
is usually ignored. As a result of this, government bonds have modest expected returns.

\bd[Goverment Bond Market]
A \textbf{government bond market} is a market where people trade government bonds.
\ed

As it makes sense, each government, or each nation, has its own government bond market, and within this marker,
different kinds of government bonds are issued. As a result, there is a wide variety of government bonds each one
with its own characteristics and risk profile. \v

Although getting into the details of each type of government bond issued by each government is beyond the scope of
these notes, it is worth mentioning the biggest government bond market in the world (that of the US), and the
government bonds they issue, the so called ``US treasuries''.

\bd[US Treasury]
A \textbf{US treasury} is a government bond issued by the United States government.
\ed

US government issues 3 different kinds of US treasuries: treasury bills, treasury notes, and treasury bonds.

\bd[Treasury Bill / T-Bill]
A \textbf{treasury bill}, or \textbf{t-bill}, is a short-term US treasury with maturity of less than one year.
\ed

Treasury bills are usually issued with maturities of 1, 3, 6, and 12 months, and are sold at a discount from their face
value.

\bd[Treasury Note / T-Note]
A \textbf{treasury note} or \textbf{t-note}, is a US treasury with maturity of one to ten years.
\ed

Treasury notes are usually issued with maturities of 2, 3, 5, 7, and 10 years, and pay interest semi-annually.

\bd[Treasury Bond / T-Bond]
A \textbf{treasury bond} or \textbf{t-bond}, is a US treasury with maturity of more than ten years.
\ed

Treasury bonds are usually issued with maturities of 20 and 30 years, and pay interest semi-annually.

\subsubsection{Corporate Bonds}

The second most important type of fixed-income securities are the so-called ``corporate bonds''.

\bd[Corporate Bond]
A \textbf{corporate bond} is a bond issued by a corporation in order to raise capital.
\ed

Corporate bonds are much more complex and difficult to categorize and analyze, however, they are usually riskier than
government bonds due to uncertainty of repayment (credit risk of both issuer and sector), uncertainty of liquidity,
and uncertainty of life (embedded options). As a result of this, they have higher expected returns. \v

There are many different types of corporate bonds when it comes to various characteristics they carry, and mentioning
them all is beyond the scope of these notes. However, one of the most important type of corporate bonds are the so
called ``convertible bonds'' and ``exchangeable bonds''.

\bd[Convertible Bond]
A \textbf{convertible bond} is a bond that can be converted into a predetermined number of shares of the issuer's
common stocks.
\ed

\bd[Exchangeable Bond]
An \textbf{exchangeable bond} is a bond that can be converted into a predetermined number of shares of the stock
of a company other than the issuer.
\ed

\subsubsection{Bond Characteristics}

When an investor buys a bond, they are lending money to the issuer in exchange for interest payments and the return of
the bond's principal when it matures.

\bd[Principal / Face Value / Par Value]
The \textbf{principal}, \textbf{face value}, or \textbf{par value} of a bond is the amount that the issuer agrees to
repay the bondholder at the maturity date.
\ed

\bd[Maturity Date]
The \textbf{maturity date} is the date on which the issuer of a bond agrees to repay to the bondholder the principal
of the bond.
\ed

The principal of the bond is usually paied out at its entirety at the end of the maturity date. In this case we refer
to the bond as a ``term'' or a ``bullet'' bond.

\bd[Term Bond / Bullet Bond]
A \textbf{term bond}, or \textbf{bullet bond}, is a bond that pays out the principal at its entirety at the end of the
maturity date.
\ed

However, there are bonds that pay out the principal in installments, called ``amortizing bonds''.

\bd[Amortizing Bond]
An \textbf{amortizing bond} is a bond that pays out the principal in installments.
\ed

The interest payments on the principal the issuer agrees to pay to the bondholder are called ``coupons''.

\bd[Coupon]
The \textbf{coupon} $C$ is the interest on the principal that the issuer agrees to pay to the bondholder.
\ed

\bd[Coupon Rate / Nominal Yield]
The \textbf{coupon rate}, or \textbf{nominal yield}, is an interest rate $r_C$ which is the amount of a coupon $C$,
expressed as a percentage of the principal $P$, including $n$ compounding periods per year:
\bse
r_C = \frac{n \cdot C}{P}
\ese
\ed

Hence, the coupon $C$ is a regular interest payment the bond-holder receives, calculated by multiplying the coupon
rate $r_C$ by the principal $P$ and dividing by the number of coupon payments per year $n$:
\bse
C = \frac{r_C \cdot P}{n}
\ese

\bd[Fixed Coupon Rate]
A \textbf{fixed coupon} is a coupon rate that remains constant over time.
\ed

\bd[Floating Coupon Rate]
A \textbf{floating coupon} is an adjustable coupon rate that can change over time, usually determined by a benchmark
index rate.
\ed

Coupons are paid either all together at the end of the maturity date together with the principal (called ``zero-coupon
bonds'' or ``strips''), or at regular time intervals, e.g.\ annually, semi-annually, etc., depending on the
fixed-income market (called ``periodic coupon bonds'').

\bd[Zero-Coupon Bond / Strip]
A \textbf{zero-coupon bond}, or \textbf{strip} is a bond in which coupons are paied at their entirety at the end of the
maturity date together with the principal.
\ed

\bd[Periodic Coupon Bond]
A \textbf{periodic coupon bond} is a bond in which coupons are paied at regular time intervals.
\ed

\subsubsection{Bond Price}

Two of the most important concepts around a bond are its ``yield to maturity'' and its ``price''. The main issue with
these two concept is that in order to define, and subsequently compute, the one, we need to know the other, and vice
versa. However, one needs to start from somewhere.

\bd[Yield To Maturity]
The \textbf{yield to maturity} (\textbf{YTM}) $r_{\text{YTM}}$ is a single discount rate equating the present value of
the bond's cash flows to its price.
\ed

\bd[Bond Price]
The \textbf{price} of a bond is the present value of the bond given the current yield to maturity that the bond is
traded.
\ed

To price\footnote{It is important to mention that each market (and sometimes each security in the market) carries its
own day count convention, and as a result of this, the calculation of the price of a bond is not always straightforward.
Some of the most common day count conventions are:
\bit
\item \textbf{Actual/Actual}: The actual number of days in the month and the actual number of days in the year.
\item \textbf{30/360}: 30 days in a month and 360 days in a year.
\item \textbf{Actual/360}: The actual number of days in the month and 360 days in a year.
\item \textbf{Actual/365}: The actual number of days in the month and 365 days in a year.
\eit

Similarly, each market carries its own compounding convention, i.e.\ annual, semi-annual, etc.} a bond, we interpret
its coupons $C$ and principal $P$ received at maturity as a series of cash flows which we discount back to the present
value using the yield to maturity $r_{\text{YTM}}$ as the discount rate.

\fig{cm3}{0.5}

Hence, by assuming $n$ compounding periods, the price of a bond is given by the formula for the present value of a
series of cash flows (\ref{eq:pv_series}):
\bse
\text{Price} = \sum_{i=1}^T \Bigg(\frac{C}{(1 + \frac{r_{\text{YTM}}}{n})^{n \cdot i}}\Bigg) +
\frac{P}{(1 + \frac{r_{\text{YTM}}}{n})^{n \cdot T}}
\ese

where:
\bse
C = \frac{r_C \cdot P}{n}
\ese

As we can see this equation combines the price of a bond with its yield to maturity. Thus, in order to calculate the
price of a bond, one needs to know its yield to maturity and by substituting it back to the equation, one can
calculate its price. Conversely, in order to calculate the yield to maturity of a bond, one needs to know its price,
and by an iterative process, since no analytical solution exists, one can find a yield to maturity that satisfies the
equation. If both are unknown at the same time, then one needs to start making assumptions. \v

\be
Let's assume a 10-year periodic coupon bond compounded semi-annually, with a principal of \$1000, a coupon rate of 7\%,
and a yield to maturity of 10\%. \v

The coupon payments are:
\bse
C = \frac{r_C \cdot P}{n} = \frac{0.07 \cdot 1000}{2} = 35
\ese

The price of the bond is:
\bse
\text{Price} = \sum_{i=1}^T \Bigg(\frac{C}{(1 + \frac{r_{\text{YTM}}}{n})^{n \cdot i}}\Bigg) + \frac{P}{(1 +
\frac{r_{\text{YTM}}}{n})^{n \cdot T}} = \sum_{i=1}^{20} \Bigg(\frac{35}{(1 + \frac{0.1}{2})^{2 \cdot i}}\Bigg) +
\frac{1000}{(1 + \frac{0.1}{2})^{2 \cdot 10}} = \ldots
\ese
\ee

Notice that so far we have made the assumption that the bond under consideration is a periodic coupon bond. To price a
zero-coupon bond, we interpret it as a single cash flow and discount it back to the present value.

\fig{cm5}{0.5}

Thus, in the case of a zero-coupon bond, the price of the bond is simply the principal $P$ received at maturity
discounted back to the present value using the yield to maturity $r_{\text{YTM}}$ as the discount rate. By
assuming $n$ compounding periods, the price of a zero-coupon bond is given by the formula for the present value of a
single cash flow (\ref{eq:pv_single}):
\bse
\text{Price} = \frac{P}{(1 + \frac{r}{n})^{n \cdot T}}
\ese

\be
The price of a 10-year zero-coupon bond compounded semi-annually, with a principal of \$100,000 and a yield to maturity
of 3.2\%  is:
\bse
\text{Price} = \frac{P}{(1 + \frac{r}{n})^{n \cdot T}} = \frac{100,000}{(1 + \frac{0.032}{2})^{2 \cdot 10}} = \ldots
\ese
\ee

Having defined a bond's price, we can now provide of ``current yield'' which differs from the nominal yield in the
sense that it is the interest on the price, and not on the principal.

\bd[Current Yield]
The \textbf{current yield}, is an interest rate $r_{\text{cy}}$ which is the amount of a coupon $C$, expressed as a
percentage of the price, including $n$ compounding periods per year:
\bse
r_{\text{cy}} = \frac{n \cdot C}{\text{Price}}
\ese
\ed

\subsubsection{Bond Valuation}

Bonds carry a certain amount of risks that makes the future return of the bondholder uncertain. Let's begin this
discussion by exploring the risks of a bond.
\bit
\item \textbf{Default / Credit Risk}: The risk that the issuer will not be able to make the interest payments or repay
the principal at the maturity date.
\eit

The credit risk is the primary risk of a bond, and it is usually measured by its credit rating.

\bd[Credit Rating]
A \textbf{credit rating} is an evaluation of the credit worthiness of a bond issuer.
\ed

In general, there are 4 main determinants of the credit risk, oftenly called the ``4 Cs of credit risk'':
\bit
\item \textbf{Capacity}: The issuer's ability to pay the interest and the principal.
\item \textbf{Covenants}: The terms of the bond that protect the bondholder.
\item \textbf{Collateral}: The assets that the issuer has pledged to secure the bond.
\item \textbf{Character}: The issuer's willingness to pay the interest and the principal.
\eit

Credit ratings are usually provided by credit rating agencies.

\bd[Credit Rating Agency]
A \textbf{credit rating agency} is a company that assigns credit ratings to fixed-income security issuers, as well ass
the fixed-income securities themselves.
\ed

The most common credit rating agencies are ``Standard \& Poor's'', ``Moody's'', and ``Fitch''. \v

The credits rating are denoted by letters and symbols, with ``AAA'' being of the highest quality rating with the
smallest degree of investment risk, and ``D'' (or ``/'' depending on the credit agency) being of the lowest quality
rating, with the largest degree of investment risk. There are more than 20 different ratings in between. \v

Depending on the credit rating, there is some jargon characterizing the riskiness of bonds.

\bd[Investment Grade Bond]
An \textbf{investment grade bond} is a bond with a credit rating of ``BBB'' or higher.
\ed

\bd[High Yield Bond / Junk Bond]
A \textbf{high yield bond}, or \textbf{junk bond}, is a bond with a credit rating of ``BB'' or lower.
\ed

Moving on the other risks associated to bonds:
\bit
\item \textbf{Interest Rate Risk}: The risk that the value of the bond will decrease due to changes in interest rates.
This risk can be spllited into two subcategories: ``price risk'' and ``reinvestment risk''.
\bit
\item \textbf{Market / Price Risk}: The risk that the value of the bond will decrease due to an increase in interest
rates.
\item \textbf{Reinvestment Risk}: The risk that the bondholder will not be able to reinvest the interest payments at
the same interest rate.
\eit
\item \textbf{Inflation Risk}: The risk that the value of the bond will decrease due to an increase in inflation.
\item \textbf{Liquidity Risk}: The risk that the bondholder will not be able to sell the bond when the want at a fair
price.
\eit

Subsequently, since bonds carry risks they also carry sources of returns:
\bit
\item \textbf{Coupons}: The interest payments that the bondholder receives from the issuer.
\item \textbf{Principal}: The principal of the bond the bondholder receives from the issuer at the maturity date.
\item \textbf{Price Changes}: The increase in the face value of the bond due to a decrease in interest rates.
\item \textbf{Compound Interest}: The interest the bondholder receives by reinvesting the coupons.
\eit

In the previous section we defined the price of the bond using the yield to maturity as the discount rate for all the
cash flows of the bond. This is convenient, since all parties to a conversation can understand what that discount rate
implies about a particular bond. \v

However, using just one, single discount rate to value a bond, is an oversimplification unable to capture all the
risks and sources of returns of a bond. As a result, the price of a bond is not the true, intrisic value of the bond.

\bd[Bond Value]
The \textbf{value} of a bond is the true, intrinsic value of a bond.
\ed

The bond value can be discovered through the process of ``bond valuation''.

\bd[Bond Valuation]
The \textbf{valuation} of a bond is the process of determining what the value of a bond is through analytic techniques.
\ed

For bond valuation we don't want to discount by a single interest rate, but with a different discount rate for each of
the cash flows that reflects the risk and timing of that cash flow:
\bse
\text{Value} = \sum_{i=1}^T \Bigg(\frac{C}{(1 + \frac{r_i}{n})^{n \cdot i}}\Bigg) +
\frac{P}{(1 + \frac{r_T}{n})^{n \cdot T}}
\ese

where:
\bse
C = \frac{r_i \cdot P}{n}
\ese

and each $r_i$ is the yield to maturity of a bond with the same risk and timing as the $i$-th cash flow of the bond
usually called ``spot rate''.

\bd[Spot Rate]
The \textbf{spot rate} $r_i$ is the yield to maturity of a bond with the same risk and timing as the $i$-th cash flow
of the bond.
\ed

Hence, while yield to maturity is a useful measure for comparing bonds with different coupon rates, maturities, and
prices, it is not a useful measure for estimating the fair, intrinsic value of a bond. For this, we need to use the
spot rates, which are the yields to maturity of bonds with the same risk and timing as the cash flows of the bond
under consideration. This is easier said than done, since spot rates are not directly observable, and need to be
estimated using statistical techniques such as bootstrapping. \v

To take this a step further, in some textbooks, yield to maturity is defined as the rate of return anticipated on a bond
if it is held until the maturity date, as given by the annual rate of return equation (\ref{def:annual_rate_of_return}).
However, due to the simplistic nature of the yield to maturity, this is far from the truth.

\be
An an extreme example, in 2008 during the financial crisis, the yield to maturity of a 10-year US Treasury bond was
around 11\%. However, the true annual rate of return of the bond to its investor was -32\%.
\ee

Coming to some more terminology, when a bond is priced at its intrinsic value, it is said to be ``at par''. When a
bond is priced above its intrinsic value, it is said to be ``at a premium'' or ``traiding rich''. When a bond is
priced below its intrinsic value, it is said to be ``at a discount'' or ``traiding cheap''.

\subsubsection{Yield Curves}

\bd[Yield Curve \ Term Structure Of Interest Rates]
A \textbf{yield curve}, or \textbf{term structure of interest rates}, is a graph that plots yields (interest rates) of
bonds with exactly the same characteristics but different times to maturity.
\ed

\fig{cm7}{0.3}

Yield curves measure the level of interest rates across a maturity spectrum. Accurate yield curves are essential for
the pricing, trading, revaluation, and hedging interest-rate sensitive products. They are also used to forecast
economic activity and to set monetary policy. \v

There are many different yield curves depending on the yield under investigation:
\bit
\item \textbf{Yield To Maturity Curve}: A yield curve that plots yields to maturity of bonds with exactly the same
characteristics but different times to maturity. Most of the time when people refer to the yield curve, they are
referring to the yield to maturity curve.
\item \textbf{Spot Rate Curve}: A yield curve that plots spot rates of bonds versus times to maturity.
\item \textbf{Forward Rate Curve}: A yield curve that plots forward rates of bonds versus times to maturity.
\eit

A yield curve can be either ``on-the-run'' or ``off-the-run''.

\bd[On-The-Run Yield Curve]
An \textbf{on-the-run yield curve} is a yield curve of bonds that are currently being issued.
\ed

\bd[Off-The-Run Yield Curve]
An \textbf{off-the-run yield curve} is a yield curve of bonds that are not currently being issued.
\ed

It is important to note that in order to draw a yield curve we need a group of homogeneous bonds, i.e.\ bonds that
have the same credit risk, the same liquidity, the same coupon rate, the same embedded optionality, as well as a few
other things, but different maturities. It's safe to say that in real life we will never find a group of bonds that
provide that degree of homogeneity, and we need to rely on analytical techniques. \v

There is a different yield curve for each different issuer sector. In other words, government yield curves would be
different from corporate yield curves, which would be different from other backed yield curves. In reality, even within
an issuer sector there are different yield curves for different types of bonds. For example, each different credit
rating category would need to have its own yield curve for us to be able to get some serious insight by assessing the
information conveyed by the yield curves. \v

Since corporate bonds are usually riskier than government bonds, the yield curve of corporate bonds is usually higher
than the yield curve of government bonds. This difference is called the ``credit spread''.

\bd[Credit Spread / Risk Premia]
The \textbf{credit spread}, or \textbf{risk premia}, is the difference between the yield to maturity of a corporate
bond and the yield to maturity of a government bond with the same maturity.
\ed

In essense the risk premia is a collection of all the risks that are not present in the government bond, and are
present in the corporate bond, such as the default risk, the liquidity risk, etc. On top of that corporate bonds also
carry the default risk (or ``risk-free') and inflation risk that government bonds carry. The following figure shows the
yield curve of a corporate bond and the yield curve of a government bond and the involved risks.

\fig{cm8}{0.32}

As it makes sense the more sensitive a bond is to interest rate changes, the more volatile its price will be. This
sensitivity is called ``duration'' and it's nothing more than the slope of the yield curve, i.e.\ the first derivative
of the yield curve.

\bd[Duration]
The \textbf{duration} of a bond is a measure of the sensitivity of the bond's price to interest rate changes:
\bse
\text{Duration} = \frac{d \text{Price}}{d r_C}
\ese
\ed

Duration depends on the bond's time to maturity, its coupon rate, and the yield to maturity. The shorter the time to
maturity, the lower the coupon rate, and the higher the yield to maturity, the lower the duration. On the other hand,
the longer the time to maturity, the higher the coupon rate, and the lower the yield to maturity, the higher the
duration. \v

Based on the duration (slope), there are four different types of yield curves:
\bit
\item \textbf{Normal Yield Curve}: A yield curve that is upward sloping. It's called normal because it's the most
common type of yield curve.
\item \textbf{Inverted Yield Curve}: A yield curve that is downward sloping. The inversion of the yield curve is created
due to rising short term rates, falling long term rates, or a combination of the two.
\item \textbf{Flat Yield Curve}: A yield curve that is flat. The flat yield curve comes about primarily due to a
combination of rising short term rates and falling long term rates. Flat yield curves are usually seen during
transitional periods.
\item \textbf{Humped Yield Curve}: A yield curve that is humped. The humped yield curve is a result of a combination
of rising short term rates and falling long term rates.
\eit

\fig{cm9}{0.52}

There are many theories explaining the shape of yield curves, with the most important ones being:
\bit
\item \textbf{Liquidity Preference Theory}: The theory that investors demand a premium for securities with longer
maturities. It can explain nothing more than the normal yield curve.
\item \textbf{Market Segmentation Theory}: The theory that the supply and demand for securities of different maturities
are not substitutes and that the shape of the yield curve is determined by the supply and demand for securities of
different maturities.
\item \textbf{Expectations Theory}: The theory that the shape of the yield curve is determined by the market's
expectations of future interest rates. It can explain all types of yield curves, and it is the most widely accepted
theory.
\eit

Under expectations theory, the intermediate and long term bond interest rates are established by the bond market
itself, primarily, through inflation expectations. It's only the short end of the yield curve that monetary authorities
are directly trying to influence or directly set the interest rates for, given that the independent monetary authority
can either raise taxes or print money to to offset their debt obligations.

\fig{cm10}{0.25}

\newpage

\section{Equity Securities}

\bd[Equity Security]
An \textbf{equity security} is a security that represents ownership in a company.
\ed

Equity securities are the second most important type of securities after debt securities. They are the most common
type of securitis and the most well-known ones. They are also the most important type of securities for the economy,
since they are the main source of financing for companies.

\bd[Equity Market]
An \textbf{equity market} is a market where people trade equity securities.
\ed

Equity markets are divided into various subsectors such as:
\bit
\item \textbf{Geographic Sector}:
\bit
\item \textbf{Domestic Market}: The market of the country where the company is based.
\item \textbf{International Market}: The market of a (developed) country other than the country where the company is
based.
\item \textbf{Emerging Market}: The market of an (emerging) country that is in the process of rapid growth and
industrialization.
\eit
\item \textbf{Industry Sector}: Information, Technology, Healthcare, Financial, Consumer, Communication, Industrial,
Energy, Utilities, Real Estate, Materials.
\eit

\subsection{Shares / Stocks}

Just like bonds are the most important type of debt securities, shares (or stocks) are the most important and the most
common type of equity securities.

\bd[Share / Stock]
A \textbf{share}, or \textbf{stock}, is an indivisible unit of equity in a company.
\ed

Hence, a company is divided into a number of shares and each share represents a piece of the company (equity). Every
share carries a specific ticker, or stock symbol, which is a unique identifier for the share.

\bd[Ticker / Stock Symbol]
A \textbf{ticker}, or \textbf{stock symbol} is a unique identifier for a share.
\ed

\be
Some of the most well-known stock symbols are the symbols of the so-called ``FAANG'' companies (which is an acronym
that refers to the stocks of five prominent American technology companies) being: Meta (META) (formerly known as
Facebook), Amazon (AMZN), Apple (AAPL), Netflix (NFLX), and Alphabet (GOOG) (formerly known as Google).
\ee

Moving on with some more terminology, shares are authorized by the company.

\bd[Authorized Shares]
The \textbf{authorized shares} are the maximum number of shares that a company is legally allowed to issue.
\ed

Authorized shares by themselves do not have any value, and they don't represent any ownership in the company. It's
after the company issues the shares to the public, hence turning them into ``issued shares'', that they represent
ownership in the company.

\bd[Issued Shares]
The \textbf{issued shares} are the number of authorized shares that a company has issued to the public.
\ed

After authorized shares are issued to the public, they are traded in the equity market. The shares that are currently
being held by the public are called ``outstanding shares'', while the shares that are currently being held by the
company are called ``treasury shares''.

\bd[Outstanding Shares]
The \textbf{outstanding shares} are the number of issued shares that are currently being held by the public.
\ed

\bd[Treasury Shares]
The \textbf{treasury shares} are the number of issued shares that are currently being held by the company.
\ed

It is of course:
\bse
\text{Issued Shares} = \text{Outstanding Shares} + \text{Treasury Shares}
\ese

It is important to mention that treasury shares do not have any voting rights. They are usually used by the company for
various purposes, such as to reissue them to the public, to use them for employee stock options, or to use them for
acquisitions. \v

Each share has a specific price, called the ``share price''.

\bd[Share Price]
The \textbf{share price} is the price of a single share of a company.
\ed

Share prices adjust to new information based on the laws of supply and demand. If a lot of people want to buy a share,
the share price will move up. If a lot of people want to sell a share, the share price will move down. \v

By multiplying the share price by the number of outstanding shares, we get the ``market capitalization'' of the
company.

\bd[Market Capitalization / Market Cap]
The \textbf{market capitalization}, or \textbf{market cap}, is the total value of a company's outstanding shares:
\bse
\text{Market Capitalization} = \text{Share Price} \times \text{Outstanding Shares}
\ese
\ed

Market capitalization is a measure of the size of a company. It is used to compare companies to each other, and to
classify companies into different categories:
\bit
\item \textbf{Large Cap}: Companies with a market capitalization of more than \$10 billion.
\item \textbf{Mid Cap}: Companies with a market capitalization of between \$2 billion and \$10 billion.
\item \textbf{Small Cap}: Companies with a market capitalization of less than \$2 billion.
\eit

Shares are usually bought and sold on what are called ``stock exchanges''.

\bd[Stock Exchange]
A \textbf{stock exchange} is a market where people buy and sell shares of companies.
\ed

There are many other stock exchanges around the world, and each country has its own stock exchange, but the most
important stock exchanges in the world are:
\bit
\item \textbf{New York Stock Exchange} (NYSE): The largest stock exchange in the world, known for its blue chip stocks
like Coca-Cola and McDonald's. NYSE shares are usually identified by a two-letter unique ticker like KO (Coca-Cola) or
MCD (McDonald's).
\item \textbf{NASDAQ Stock Market} (NASDAQ): The second-largest stock exchange in the world, known for its tech stocks
like Netflix and Apple. Depot). Nasdaq shares are usually identified by a four-letter unique ticker like NFLX (Netflix)
or AAPL (Apple).
\item \textbf{Tokyo Stock Exchange} (TSE): The largest stock exchange in Asia. TSE shares are usually identified by a
four-number unique ticker like 7203 (Toyota) or 6758 (Sony).
\eit

Stock exchanges used to involve a lot of men standing on a floor with little pads of paper and yelling buy and sell
orders. Now it's all done by computers. The computers match up buyers and sellers who want to exchange their stock for
cash (or vice versa) at a certain price. \v

As an individual, one cannot trade directly on a stock exchange. For that one will need a broker which we have already
discussed previously.

\subsection{Shareholders / Stockholders}

When someone buys a share of a company, he or she becomes a partial owner of the company. He or she becomes a
``shareholder'' or a ``stockholder''.

\bd[Shareholder / Stockholder]
A \textbf{shareholder}, or \textbf{stockholder}, is any owner of one or more (outstanding) shares of a company.
\ed

The company is legally required to provide their shareholders with certain rights and privileges including the right
to vote in the company's annual general meeting, the right to receive dividends (more on that later), the right to
receive a portion of the company's assets in case the company is liquidated, and others. \v

People who buy shares are usually divided into two categories based on the strategy they follow: ``investors'' and
``traders''.

\bd[Investor]
An \textbf{investor} is a person who buys shares with the intention of holding them for a long period of time.
\ed

\bd[Trader]
A \textbf{trader} is a person who buys shares with the intention of selling them for a profit in a short period of
time.
\ed

\subsubsection{Types Of Shares}

There are many different types of shares depending on their characteristics. In this subsection we will mention the
most important ones, starting with the distinction between ``common'' and ``preference'' shares.

\bd[Common Shares / Common Stock]
A \textbf{common share}, or \textbf{common stock}, is a type of share that represents ownership in a company and a
claim on part of the company's profits.
\ed

Common shares are the most common type of shares, hence the naming, and they are the most widely held ones. They are,
usually, the most volatile type of shares, and subsequently the most risky ones. They are also the most important type
of shares for the company, since they are the main source of financing.

\bd[Preference Shares / Preferred Stock]
A \textbf{preference share}, or \textbf{preferred stock}, is a type of stock that has a higher claim on the company's
assets and earnings than common stock.
\ed

Although a preference share is a type of share widely assumed to belong to equity securities, in reality it is more
similar to a bond than to a common share. Preference shares pay a fixed dividend, and they are usually rated by credit
rating agencies. In addition, they are usually issued with a ``par value'' and they are callable by the issuer at a
predetermined price. \v

Another important distiction is the one among ``blue chip'', ``growth'', and ``value'' shares.

\bd[Blue Chip Share]
A \textbf{blue chip share} is a share of a company that has a long history of stable earnings, and a strong balance
sheet.
\ed

\bd[Growth Share]
A \textbf{growth share} is a share of a company that is expected to grow at an above-average rate compared to other
companies.
\ed

\bd[Value Share]
A \textbf{value share} is a share of a company that is considered to be undervalued by the market.
\ed

\subsection{Return Of Equity Securities}

Unlike bonds, the return of equity securities is not fixed, and it is not guaranteed. It comes mainly from the
dividends that the company pays, and from the capital gains that the investor makes when they sell the shares at a
higher price than the price they bought them. The return of equity securities is usually higher than the return of
debt securities, but it is also more volatile and more uncertain.

\bd[Dividend]
A \textbf{dividend} is a payment made by a corporation to its shareholders, usually in the form of cash or additional
shares.
\ed

Dividends are usually paid out of the company's profits. The company's board of directors decides whether to pay a
dividend, and how much to pay. The company's board of directors can also decide to not pay a dividend at all, and
instead reinvest the profits back into the company. The decision to pay a dividend, and how much to pay, is usually
made based on the company's financial performance, the company's growth prospects, and the company's capital needs.

\bd[Capital Gain]
A \textbf{capital gain} is the increase in the value of an asset that makes it worth more than the purchase price.
\ed

\bd[Capital Loss]
A \textbf{capital loss} is the decrease in the value of an asset that makes it worth less than the purchase price.
\ed

\subsection{Valuation Of Equity Securities}

\bd[Price-Earnings Ratio / Earnigs Multiple]
The \textbf{price-earnings ratio} (\textbf{P/E ratio}), or \textbf{earnings multiple} is a measure of the price paid
for a share relative to the annual net income or profit earned by the firm per share:
\bse
\text{P/E ratio} = \frac{\text{Price per share}}{\text{Earnings per share}}
\ese
\ed

P/E ratio is usually used in order to compute the target price of a share which is the price at which an analyst
believes a stock is fairly valued:
\bse
\text{Target Price} = \text{Earnings per share} \times \text{P/E ratio}
\ese

Based on the P/E ratio, there are two different types of shares:
\bit
\item \textbf{Growth Shares}: Shares with a high P/E ratio.
\item \textbf{Value Shares}: Shares with a low P/E ratio.
\eit

The P/E ratio is a measure of how much investors are willing to pay for a share relative to the annual net income or
profit earned by the firm per share. It is a measure of the market's expectations of the company's future growth
prospects. A high P/E ratio means that investors are expecting high growth in the future, while a low P/E ratio means
that investors are expecting low growth in the future. \v

The P/E ratio is also a measure of the company's valuation. A high P/E ratio means that the company is overvalued, while
a low P/E ratio means that the company is undervalued. However, the P/E ratio is not a perfect measure of the company's
valuation, and it should be used in conjunction with other measures.

Shares are exchanged either in the exchange market or in the over-the-counter market.

\bd[Exchange Market]
An \textbf{exchange market} is a market where shares are traded in a centralized location.
\ed

\bd[Over-The-Counter Market]
An \textbf{over-the-counter market} is a market where shares are traded directly between two parties.
\ed

\subsection{Funds}

\bd[Index]
An \textbf{index} is a collection of shares that represents a particular market or a particular sector of the market.
\ed

Indices are used to measure the performance of the market or the performance of a particular sector of the market. It
is important to mention that indices are usually market-cap weighted, which means that the companies in them that
have larger market caps are given higher weightings and thus have a greater influence on the index. \v

Some of the most well-known indices are:
\bit
\item \textbf{Standard and Poor's 500} (\textbf{S\&P 500}): An index tracking the performance of 500 of the largest
companies listed on stock exchanges in the United States. To see the companies that are included in the S\&P 500 check
the following link: \href{https://www.slickcharts.com/sp500}{S\&P 500}.
\item \textbf{Dow Jones Industrial Average} (\textbf{DJIA}): An index tracking the performance of 30 prominent companies listed
on stock exchanges in the United States. To see the companies that are included in the DJIA check the following link:
\href{https://www.slickcharts.com/dowjones}{DJIA}.
\item \textbf{NASDAQ Composite} (\textbf{NASDAQ}): An index tracking the performance of all companies listed on the
NASDAQ Stock Market. To see the companies that are included in the NASDAQ Composite check the following link:
\href{https://www.slickcharts.com/nasdaq100}{NASDAQ}.
\eit

In order for someone to invest in an index, they would need to buy all the stocks of the index (in the appropriate
wieghts) at the same time. That would be difficult and costly to execute. Fortunately, there are funds that track the
performance of indices.

\bd[Fund]
A \textbf{fund} is a collection of shares that represents a particular market or a particular sector of the market.
\ed

\bd[Index Fund]
An \textbf{index fund} is a fund that tracks the performance of an index.
\ed

\bd[Exchange-Traded Fund]
An \textbf{exchange-traded fund} (ETF) is a fund that tracks the performance of an index and is traded on a stock
exchange.
\ed

Index funds and ETFs are usually used by investors who want to invest in the market or in a particular sector of the
market, but don't want to spend the time and effort to pick individual stocks. They are also used by investors who
want to diversify their portfolio, and by investors who want to reduce the risk of their portfolio. \v

Some of the most well-known ETFs are:
\bit
\item \textbf{SPDR S\&P 500 ETF} (SPY): An ETF that tracks the performance of the S\&P 500 index.
\item \textbf{DJIA ETF} (DIA): An ETF that tracks the performance of the DJIA index.
\item \textbf{Invesco QQQ Trust} (QQQ): An ETF that tracks the performance of the NASDAQ index.
\eit

\bd[Active Investing]
\textbf{Active investing} is an investment strategy that involves actively buying and selling shares in an attempt to
outperform the market.
\ed

Active investing is usually done by professional investors, and it is usually done by picking individual stocks.

\bd[Passive Investing]
\textbf{Passive investing} is an investment strategy that involves buying and holding shares in an attempt to match
the performance of the market.
\ed

Passive investing is usually done by individual investors, and it is usually done by investing in index funds and ETFs.
The most well-known passive investing strategy is ``indexing''.

\bd[Indexing]
\textbf{Indexing} is a passive investment strategy that involves investing in index funds and ETFs.
\ed

%TODO: Definitions of Quotes, Volume & Trading Volume