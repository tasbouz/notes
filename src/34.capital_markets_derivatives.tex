%! suppress = EscapeUnderscore
\section{Derivatives}

\bd[Derivative]
A \textbf{derivative} is a contract for a future transaction of an underlying asset between two parties, the buyer and
the seller, which derives its value from the performance of the underlying asset.
\ed

\bd[Underlying Asset]
The \textbf{underlying asset} is any asset that has a value which can be measured, from which a derivative derives its
value from.
\ed

\be
The underlying asset can be really anything. From a financial asset such as a stock, an index, or a rate, to a
non-financial asset such as a piece of real estate or a car, or even a non-physical asset, such as a weather condition,
or the outcome of a sports game.
\ee

In the derivatives jargon, the buyer of the derivative is referred to as the ``long'' position and the seller is
referred to as the ``short'' position.

\bd[Long / Short Position]
The party that agrees to buy the asset is said to be in a \textbf{long position}, while the party that agrees to sell
the asset is said to be in a \textbf{short position}.
\ed

\bd[Derivative Market]
A \textbf{derivative market} is a submarket of capital markets where people trade derivatives.
\ed

The derivatives markets are much bigger than the stock market when measured in terms of underlying assets. Depending
on the market being traded derivatives can be either ``exchange-traded'' or ``over-the-counter''.

\bd[Exchange-Traded Derivative]
An \textbf{exchange-traded derivative} is a standardized derivative that is traded on a regulated derivative exchange.
\ed

\bd[Derivative Exchange]
A \textbf{derivative exchange} is a central derivative market where people can trade standardized exchange-traded
derivatives defined by the exchange.
\ed

Once two traders have agreed to a trade offered by an exchange, it is handled by the clearing house.

\bd[Clearing House]
A \textbf{clearing house} is an organization that facilitates tradings between two parties and manages the risks.
\ed

The advantage of a clearing house is that traders do not have to worry about the creditworthiness of the people they
are trading with. The clearing house takes care of credit risk by requiring each of the two traders to deposit funds
to ensure that they will live up to their obligations. \v

Traditionally exchange-traded derivatives traders used to physically meet on the floor of the exchange, shouting, and
using a complicated set of hand signals to indicate the trades they would like to carry out. Exchanges have largely
replaced this system by electronic trading which involves the use of algorithms to initiate trades, often without human
intervention, and has become an important feature of derivatives markets.

\bd[Over-The-Counter Derivative]
An \textbf{over-the-counter (OTC) derivative} is a derivative that is traded off-exchange.
\ed

\bd[Over-The-Counter Market]
An \textbf{over-the-counter (OTC) market} is a decentralized derivative market where trading of OTC derivatives is done
directly between two parties without a central exchange or clearing house.
\ed

Banks, other large financial institutions, fund managers, and corporations are the main participants in OTC derivatives
markets. Credit risk has traditionally been a feature of OTC derivatives markets. In an attempt to reduce credit risk,
the OTC market has borrowed some ideas from exchange-traded markets. Once an OTC trade has been agreed, the two parties
can either present it to a central counterparty, which is like an exchange clearing house, or clear the trade
bilaterally where the two parties have usually signed an agreement covering all their transactions with each other. \v

While OTC markets were largely unregulated, following the financial crisis of 2008 we have seen the development of
many new regulations affecting the operation of OTC markets. The main objectives of the regulations are to improve
the transparency of OTC markets and reduce systemic risk. The OTC market in some respects is being forced to become
more like the exchange-traded market. \v

Both exchange-traded and OTC markets for derivatives are huge. The number of derivatives transactions per year in OTC
markets is smaller than in exchange-traded markets, but the average size of the transactions is much greater, making
the volume of business in OTC market much larger than in the exchange-traded one. \v

Derivatives markets have been outstandingly successful. The main reason is that they have attracted many different
types of traders and have a great deal of liquidity. When a trader wants to take one side of a contract, there is
usually no problem in finding someone who is prepared to take the other side. \v

In general, three broad categories of traders can be identified:
\bit
\item \textbf{Hedgers}: Use derivatives to reduce the risk that they face from potential future movements in a market
variable.
\item \textbf{Speculators}: Use derivatives to bet on the future direction of a market variable.
\item \textbf{Arbitrageurs}: Take offsetting positions in two or more instruments to lock in a profit.
\eit

\section{Futures / Forwards}

\bd[Future / Forward]
A \textbf{future} or \textbf{forward}\footnote{The main difference between a future and a forward is that a future is an
exchange-traded derivative while a forward is an OTC derivative. From now on we will be using the term ``future'' to
refer to both futures amd forwards.} is a derivative that obligates the buyer to buy and the seller to sell the
underlying asset at a specified future date $T$ for a specified price $K$.
\ed

\bd[Delivery Date / Maturity Date]
The specified future date $T$ in a future is called the \textbf{delivery} or \textbf{maturity date}.
\ed

\bd[Delivery Price / Future Price]
The specified price $K$ in a future is called the \textbf{delivery} or \textbf{future price}.
\ed

Future price, like any other price, is determined by the laws of supply and demand. If, at a particular time, more
traders wish to sell rather than buy, the price will go down. New buyers then enter the market so that a balance
between buyers and sellers is maintained. If more traders wish to buy rather than sell the price goes up. New
sellers then enter the market and a balance between buyers and sellers is maintained. \v

When developing a future, the exchange must specify in some detail the exact nature of the agreement between the two
parties. In particular, it must specify:
\bit
\item \textbf{The Asset}: Acceptable commodities or well-defined financial assets.
\item \textbf{The Contract Size}: The amount of the asset to be delivered per contract.
\item \textbf{Delivery Arrangements}: The delivery location, crucial for commodities with high transport costs.
\item \textbf{Delivery Months}: The delivery period during the month, varying by contract.
\item \textbf{Price Quotes}: How prices are quoted.
\item \textbf{Price \& Position Limits}: Daily price movement limits to prevent speculative excesses, and position
limits for speculators.
\eit

It is important to mention that the vast majority of futures contracts do not lead to the delivery of the underlying
asset. Instead, they are ``closed out'' early, prior to the delivery period specified in the contract. Nevertheless,
it is the possibility of eventual delivery that determines the futures price.

\bd[Closing Out]
\textbf{Closing out} a position means entering into the opposite trade to the original one.
\ed

Beside the future price, there is also the price of the underlying asset, which is called the ``spot price''.

\bd[Spot Price]
The \textbf{spot price} $S_t$ is the price of an asset for immediate delivery at time $t$.
\ed

Given the future price and the spot price, it is quite straightforward to determine the payoff for a long and a short
position in a future. More specifically, the payoff for a long position $P_{\text{long}}$ in a future on one unit of an
asset is:
\bse
P_{\text{long}} = S_T - K
\ese

where $S_T$ is the spot price of the asset at maturity of the contract. This is because the holder of the contract is
obligated to buy an asset worth $S_T$ for $K$. \v

Similarly, the payoff for a short position $P_{\text{short}}$ in a future on one unit of an asset is:
\bse
P_{\text{short}} = K - S_T
\ese

\vspace{5pt}
\fig{dv1}{0.56}

These payoffs can be positive or negative. Because it costs nothing to enter into a future contract, the payoff from
the contract is also the trader's total gain or loss from the contract. \v

As the delivery period for a futures contract is approached, the futures price converges to the spot price of the
underlying asset. When the delivery period is reached, the futures price equals, or is very close to, the spot price.
This is known as the ``convergence property'' of futures prices. If that was not the case, then traders could make
riskless profits by exploiting the difference between the futures price and the spot price as the delivery period
approaches.

\fig{dv2}{0.45}

\subsection{Margin Accounts}

If two traders get in touch with each other directly and agree to trade an asset in the future for a certain price,
there are obvious risks. One of the traders may regret the deal and try to back out. Alternatively, the trader
simply may not have the financial resources to honor the agreement. One of the key roles of the exchange is to
organize trading so that contract defaults are avoided. This is where ``margin accounts'' come in.

\bd[Margin Account]
A \textbf{margin account} is an account in which a trader must keep funds in to ensure that they will be able to meet
the obligations of a contract.
\ed

The whole purpose of the margining account is to ensure that funds are available to pay traders when they make a
profit. Overall the system has been very successful.

\bd[Initial Margin]
The amount that must be deposited in the margin account at the time the contract is entered into is known as the
\textbf{initial margin}.
\ed

At the end of each trading day, the margin account is adjusted to reflect the trader's gain or loss. This practice is
referred to as ``daily settlement'' or ``marking to market''.

\bd[Daily Settlement / Marking To Market]
\textbf{Daily settlement} or \textbf{marking to market} is the process of adjusting the margin account to reflect the
trader's gain or loss at the end of each trading day.
\ed

Daily settlement leads to funds flowing each day between traders with long positions and traders with short positions.
If the futures price increases from one day to the next, funds flow from traders with short positions to traders with
long positions. If the futures price decreases from one day to the next, funds flow in the opposite direction, from
traders with long positions to traders with short positions. This daily flow of funds between traders to reflect gains
and losses is known as ``variation margin''.

\bd[Variation Margin]
The amount that must be paid to the margin account to reflect the trader's gain or loss is called the \textbf{variation
margin}.
\ed

As it makes sense, variation margins change the balance in the margin account. Margin accounts have a minimum balance
that must be maintained at all times, known as the ``maintenance margin''.

\bd[Maintenance Margin]
The minimum amount that must be maintained in the margin account is called the \textbf{maintenance margin}.
\ed

The maintenance margin is usually 75\% of the initial margin, however, both initial and maintenance margins are set
by the exchange clearing house and are determined by the variability of the price of the underlying asset and are
revised when necessary. As it makes sence, the higher the variability, the higher the margin levels. If the balance
in the margin account falls below the maintenance margin, due to the variation margin being negative, the trader
will receive a ``margin call'' from the exchange.

\bd[Margin Call]
A \textbf{margin call} is a demand for additional funds to be deposited in the margin account, when the balance in it
falls below the maintenance margin, to bring the balance up to the initial margin level.
\ed

The trader who receives a margin call is expected to top up the margin account to the initial margin level within a
short period of time. If the trader does not provide this variation margin, the broker closes out the position. On
the other hand, if the trader's contract increases in value, the balance in the margin account increases. The trader
is entitled to withdraw any balance in the margin account that is in excess of the initial margin.

\be
Here is a simple example of a margin account with an initial margin of \$12,000, and maintenance margin of \$9,000.
The daily gains represent the variation margins that are either added or subtracted from the margin account.
\fig{dv3}{0.4}
\ee

Most brokers pay traders interest on the balance in a margin account wiyh interest rates being competitive with what
could be earned elsewhere. To satisfy the initial margin requirements, but not subsequent margin calls, a trader can
usually deposit securities with the broker. The market value of the securities is reduced by a certain amount to
determine their value for margin purposes. This reduction is known as a ``haircut''.

\bd[Haircut]
A \textbf{haircut} is the reduction in the market value of securities that are deposited with a broker to satisfy the
initial margin requirements.
\ed

\subsection{Hedging Using Futures}

Many of the participants in futures markets are hedgers. Their aim is to use futures markets to reduce a particular
risk that they face. When one chooses to use futures markets to hedge a risk, the objective is often to take a
position that neutralizes the risk as far as possible.

\bd[Perfect Hedge]
A \textbf{perfect hedge} is a hedge that completely eliminates the risk.
\ed

Perfect hedges are rare. For the most part, therefore, a study of hedging using futures contracts is a study of the
ways in which hedges can be constructed so that they perform as close to perfectly as possible.

There are two main types of hedges: short hedges and long hedges.

\bd[Short Hedge]
A \textbf{short hedge} is a hedge that involves a short position in futures contracts.
\ed

A short hedge is appropriate when the hedger already owns an asset and expects to sell it at some time in the
future, or when an asset is not owned right now but will be owned and ready for sale at some time in the future.

\be
Consider a company that knows it will gain \$10,000 for each 1 cent increase in the price of a commodity over the next
3 months and lose \$10,000 for each 1 cent decrease in the price during the same period. \v

Assume that it is May 15 today and that the company has just negotiated a contract to sell 1 million units of the
commodity. It has been agreed that the price that will apply in the contract is the market price on August 15.
Suppose that on May 15 the spot price is \$50 per unit and the commodity futures price for August delivery is \$49 per
unit. Because each futures contract is for the delivery of 1,000 units, the company can hedge its exposure by
shorting (i.e., selling) 1,000 futures contracts ($1,000 \text{Contrancts} \times 1,000 \text{Units} = 1,000,000
\text{Units})$. \v

Suppose that the spot price on August 15 proves to be \$45 per unit. The company realizes \$45 million for the commodity
under its sales contract, and $\$49 - \$45 = \$4$ per unit, or \$4 million from the short futures position. The total
amount realized from both the futures position and the sales contract is therefore approximately \$49 million. \v

For an alternative outcome, suppose that the price of the commodity on August 15 proves to be \$55 per unit. The company
realizes \$55 million for the commodity under its sales contract, and loses $\$55 - \$49 = \$6$ per unit, or \$6 million
from the short futures position. The total amount realized from both the futures position and the sales contract is
therefore approximately \$49 million. \v

It is easy to see that in all cases the company ends up with approximately \$49 million.
\ee

\bd[Long Hedge]
A \textbf{long hedge} is a hedge that involves a long position in futures contracts.
\ed

A long hedge is appropriate when a company knows it will have to purchase a certain asset in the future and wants to
lock in a price now.

\be
Suppose that it is now January 15. A company knows it will require 100,000 units of a commodity on May 15 to meet a
certain contract. The spot price of commodity is \$3.40 per unit, and the futures price for May delivery is \$3.20 per
unit. The company can hedge its position by taking a long position in four futures contracts of 25,000 units of the
commodity. The strategy has the effect of locking in the price of the required commodity at close to \$3.20 per pound.
\v

Suppose that the spot price of the commodity on May 15 proves to be \$3.25 per unit. The company therefore gains
approximately $100,000 * (\$3.25 - \$3.20) = \$5,000$ on the futures contracts. It pays $100,000 * \$3.25 = \$325,000$
for the commodity, making the net cost approximately $\$325,000 - \$5,000 = \$320,000$. \v

For an alternative outcome,suppose that the spot price is \$3.05 per unit on May 15. The company then loses
approximately $100,000 * (\$3.20 - \$3.05) = \$15,000$ on the futures contracts, and pays $100,000 * \$3.05 = \$305,000$
for the commodity, making the net cost approximately $\$305,000 + \$15,000 = \$320,000$. \v

Again, it is easy to see that in all cases the company ends up paying approximately \$320 million.
\ee

The arguments in favor of hedging are so obvious that they hardly need to be stated. Most nonfinancial companies are
in the business of manufacturing, or retailing or wholesaling, or providing a service. They have no particular
skills or expertise in predicting variables such as interest rates, exchange rates, and commodity prices. It makes
sense for them to hedge the risks associated with these variables as they become aware of them. The companies can
then focus on their main activities. By hedging, they avoid unpleasant surprises such as sharp rises in the price of
a commodity that is being purchased. \v

However, if hedging is not the norm in a certain industry, it may not make sense for one particular company to
choose to be different from all others. Competitive pressures within the industry may be such that the prices of the
goods and services produced by the industry fluctuate to reflect raw material costs, interest rates, exchange rates,
and so on. In other words, all the implications of price changes on a company’s profitability should be taken into
account in the design of a hedging strategy to protect against the price changes. A company that does not hedge can
expect its profit margins to be roughly constant. However, a company that does hedge can expect its profit margins
to fluctuate! It is important to realize thus, that a hedge using futures contracts can result in a decrease or an
increase in a company’s profits relative to the position it would be in with no hedging. It is important for
executives within the organization to fully understand the nature of hedging before a hedging program is put in
place. Ideally, hedging strategies are set by a company’s board of directors and are clearly communicated to both
the company’s management and the shareholders. \v

In real life, hedging is often not quite as straightforward as in the examples above. One of the main reasons, among
others, is that the asset whose price is to be hedged may not be exactly the same as the asset underlying the futures
contract. This is known as ``cross hedging''.

\bd[Cross Hedging]
\textbf{Cross hedging} is a hedging strategy that involves taking a position in a futures contract that is different
from the asset being hedged.
\ed

\be
Consider, for example, an airline that is concerned about the future price of jet fuel. Because jet fuel futures are
not actively traded, it might choose to use heating oil futures contracts to hedge its exposure.
\ee

Let us explore the concept of cross hedging in more detail. We begin by defining the ``hedge ratio''.

\bd[Hedge Ratio]
The \textbf{hedge ratio} is the ratio of the size of the position taken in futures contracts to the size of the
exposure.
\ed

When the asset underlying the futures contract is the same as the asset being hedged, it is natural to use a hedge
ratio of 1. This is the hedge ratio we have used in the examples considered so far. When cross hedging is used,
setting the hedge ratio equal to 1 is not always optimal. The hedger should choose a value for the hedge ratio that
minimizes the variance of the value of the hedged position. We now consider how the hedger can do this.

\section{Options}

\bd[Option]
An \textbf{option} is a contract that gives the holder the right, but not the obligation, to buy or sell an underlying
asset at a specified price $K$ at a specified future date $T$.
\ed

Based on if the holder has the right to buy or sell the underlying asset, we distinguish between ``call'' and ``put''
options.

\bd[Call Option]
A \textbf{call option} is an option that gives the holder the right to buy an underlying asset at a specified price $K$
at a specified future date $T$.
\ed

\bd[Put Option]
A \textbf{put option} is an option that gives the holder the right to sell an underlying asset at a specified price $K$
at a specified future date $T$.
\ed

It should be emphasized that an option gives the holder the right to do something. The holder does not have to
exercise this right. This is what distinguishes options from forwards and futures, where the holder is obligated to
buy or sell the underlying asset.

\bd[Strike Price / Exercise Price]
The specified price $K$ in an option is called the \textbf{strike price} or \textbf{exercise price}.
\ed

\bd[Expiration Date / Maturity]
The specified future date $T$ in an option is called the \textbf{expiration date} or \textbf{maturity}.
\ed

Options are traded both on exchanges and in the over-the-counter market. Based on if the option can be exercised only
on the expiration date or at any time up to the expiration date, we distinguish between ``European'' and ``American''
options.

\bd[European Option]
A \textbf{European option} is an option that can be exercised only on the expiration date.
\ed

\bd[American Option]
An \textbf{American option} is an option that can be exercised at any time up to the expiration date.
\ed

Note that the terms American and European do not refer to the location of the option or the exchange. Some options
trading on North American exchanges are European. Most of the options that are traded on exchanges are American.
European options are generally easier to analyze than American options, and some of the properties of an American
option are frequently deduced from those of its European counterpart. \v

At this stage we note that there are four types of participants in options markets:
\bit
\item Buyers of calls.
\item Sellers of calls.
\item Buyers of puts.
\item Sellers of puts.
\eit

Whereas it costs nothing to enter into a forward or futures contract, there is a cost to acquiring an option.





Another crucial
difference is that forwards are settled at the end of their life, while futures are, as settled daily.


The purpose of hedging is to reduce risk. There is no guarantee that the outcome with hedging will be better than the
outcome without hedging.

There is a fundamental difference between the use of forward contracts and options
for hedging. Forward contracts are designed to neutralize risk by fixing the price that
the hedger will pay or receive for the underlying asset. Option contracts, by contrast,
provide insurance. They offer a way for investors to protect themselves against adverse
price movements in the future while still allowing them to benefit from favorable price
movements. Unlike forwards, options involve the payment of an up-front fee.

Futures and options are similar instruments for speculators in that they both provide a
way in which a type of leverage can be obtained. However, there is an important
difference between the two. When a speculator uses futures, the potential loss as well as
the potential gain is very large. When options are purchased, no matter how bad things
get, the speculator's loss is limited to the amount paid for the options.


Derivatives are very versatile instruments.



It is this very versatility that can cause problems.