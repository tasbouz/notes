%! suppress = EscapeUnderscore
In this chapter we will introduce a very important concept called ``connection'' (or ``covariant derivative'') on a
differentiable topological manifold. However, we need to make something very clear. ``Connection'' and ``covariant
derivative'' are not actually the same thing. In elementary courses on differential geometry the notions of
connection and covariant derivative are often confused with one another and, sometimes, the terms are even used as
synonyms. This happens because one is not in a position to properly define these concepts before introducing the
so-called ``principal fiber bundles'' that we will do in a later chapter. \v

So for now, in this section, we will make a kind of ``naive'' introduction to the concept, which however, most of the
time, this is how it is introduced in many courses and on top of that it will give us a very first understanding of
the topic. Later in the notes, once we introduce ``principal fiber bundles'' we will revisit these terms in a more
systematic way, and we will gain a better understanding of what they are exactly and how they generalize. \v

Having said that, for now we will introduce ``connection'' and ``covariant derivative'' as identical concepts which,
again, even though they are different, for the purposes of this section we shall not distinguish the two, and we will
also introduce ``parallel transport'' within this framework (later we will also revisit parallel transport).

\section{Connection / Covariant Derivative}

So far, we saw that a vector field $X$ can be used to provide a directional derivative of a function $f \in
C^{\infty}(M)$ in the direction $X$ as $Xf(p) = X_p f$. Recall that once we introduced the $(p,q)$ tensor field, we
saw that a function $f \in C^{\infty}(M)$ is nothing else but a $(0,0)$ tensor field. Hence, the equation $Xf$ can be
seen as the directional derivative acting on a $(0,0)$ tensor field, which leads to the natural question: can we
apply the directional derivative to any $(p,q)$ tensor field? \v

The short answer is no, and the reason is that we do not know how the field $X$ can act on anything else other than a
function (this is how we defined it). However, one could try to generalize the action of $X$ to any $(p,q)$ tensor
field by defining a new object $\nabla_X$, called ``connection'' or ``covariant derivative'', able to act on any $(p,
q) $ tensor field (for this section we will stick with the term ``connection''). \v

\begin{center}
\begin{tikzpicture}
\matrix (m) [matrix of nodes, row sep=3em, column sep=3em, minimum width=1em]
{ $X : C^{\infty}(M)$ & $C^{\infty}(M)$ \\ $\nabla_X : (p,q)$-tensor field & $(p,q)$-tensor field \\ };
\path[->] (m-1-1) edge (m-1-2);
\path[->] (m-2-1) edge (m-2-2);
\path[->] (m-1-1) edge[snake it] node[left] {$\vdots$} (m-2-1);
\path[->] (m-1-2) edge[snake it] node[left] {$\vdots$} (m-2-2);
\end{tikzpicture}
\end{center}

\v

The question is how to define such an object? Since it is closely related to $X$ it must be consistent with it, so we
will define $\nabla_X$ through some ``consistency'' conditions that it must satisfy.

\bd [Connection / Covariant Derivative]
A \textbf{connection} (or \textbf{covariant derivative}) $\nabla_X$ on a smooth manifold $M$ is a map that takes a
pair consisting of a vector field $X$ and a $(p,q)$-tensor field $T$ and sends them to a $(p,q)$-tensor field
$\nabla_X T$ satisfying:
\bit
\item[1.] $\nabla_X f = Xf, \, \forall f \in C^{\infty}M.$ (in order to be consistent with the already defined action
of $Xf$).
\item[2.] $\nabla_X (T + S) = \nabla_X T + \nabla_X S.$ (due to linearity of $X$).
\item[3.] Since it's a notion of derivative, for any $(p,q)$ tensor field $T$ the Leibnitz rule must be satisfied:
\bi{rCl}
\nabla_X T(\omega_1,\dotsc,\omega_p,Y_1,\dotsc,Y_q) &=& (\nabla_X T) (\omega_1,\dotsc,\omega_p,Y_1,\dotsc,Y_q) \\
& + & T(\nabla_X \omega_1,\dotsc,\omega_p,Y_1,\dotsc,Y_q) + \dotsb + T(\omega_1,\dotsc,\nabla_X \omega_p,Y_1,\dotsc, Y_q) \\
& + & T(\omega_1,\dotsc,\omega_p,\nabla_X Y_1,\dotsc,Y_q) + \dotsb + T(\omega_1,\dotsc,\omega_p,Y_1,\dotsc,\nabla_X Y_q)
\ei
\item[4.] $\nabla_{fX+Z} T = f\nabla_X T + \nabla_Z T, \, \forall f \in C^{\infty}(M).$ (This is
$C^{\infty}$-linearity, which means that no matter how the function $f$ scales the vectors at different points of the
manifold, the effect of the scaling at any point is independent of scaling in the neighbourhood and depends only on
how the scaling happens at that point)
\eit
\ed

After formulating this ``wish list'' of properties which $\nabla_X$ acting on a tensor field should have, in general
there may be many structures that satisfy these conditions. Any remaining freedom in choosing such a $\nabla_X$, will
need to be provided as additional structure beyond the structure we already have, as we did for example with the
topology and the atlases.

\bd [Topological Manifold With A Connection]
A \textbf{topological manifold with a connection} $\nabla_X$ is a quadruple $(M, \mathcal{O}, \mathscr{A}, \nabla_X)
$, where $M$ is a set, $\mathcal{O}$ is a chosen topology on the set, $\mathscr{A}$ is a chosen smooth atlas and
$\nabla_X$ is a chosen connection.
\ed

The question now is, how much freedom do we have in choosing a connection? In order to answer that we will let it act
in ``all'' possible $(p,q)$ tensor fields and see what we get. \v

Starting with the simplest case of a $(0,0)$ tensor field, i.e.\ a function $f$, by the first condition we simply have:
\bse
\nabla_X f = Xf
\ese

\v

So far so good. Let as consider now the next simplest case of an $(1,0)$ tensor field, i.e.\ a vector field $Y$:
\bi{rCl"s}
\nabla_X Y & = & \nabla_{\left(X^i \cibasis{x^i}\right)} \left(Y^m \cibasis{x^m}\right) & (expressed in chart (U,x)) \\[5pt]
& =& X^i \cdot \nabla_{\left(\cibasis{x^i}\right)} \left(Y^m \cibasis{x^m}\right) & (fourth condition)\\[5pt]
& =& X^i \left(\nabla_{\left(\cibasis{x^i}\right)} Y^m\right) \cibasis{x^m}
+ X^i \cdot Y^m \cdot \left(\nabla_{\left (\cibasis{x^i}\right)} \cibasis{x^m}\right) & (third condition) \\[5pt]
& = & X^i \cibasis{x^i} Y^m \cibasis{x^m}
+ X^i \cdot Y^m \cdot \left (\nabla_{\left(\cibasis{x^i}\right)} \cibasis{x^m}\right)& (first condition)
\ei

\v

Regarding the last term $\nabla_{\left(\cibasis{x^i}\right)} \cibasis{x^m}$, note that it simply is of the form
$\nabla_X X$ where $X$ is the coordinate induced basis vector. There isn't something more we can do on that, however
since $\cibasis{x^m}$ is a basis vector, i.e.\ a $(1,0)$ tensor field, the result of the action of $\nabla_{\left
(\cibasis{x^i}\right)}$ on it must by definition, yield again a $(1,0)$ tensor field i.e.\ a vector that we can express
in the coordinate induced basis also as:
\bse
\nabla_{\left(\cibasis{x^i}\right)} \cibasis{x^m} = \Gamma^{q}_{im} \cibasis{x^q}
\ese

Hence:
\bse
\nabla_X Y = X^i \left(\cibasis{x^i} Y^m\right) \cibasis{x^m} + X^i \cdot Y^m \cdot \Gamma^{q}_{im} \cibasis{x^q}
\ese

\v

Thus, by rearranging the indices and the terms and by discarding the basis vectors:
\bse
\left(\nabla_X Y\right)^i = X^m \left(\cibasis{x^m} Y^i\right) + \Gamma^{i}_{mn} \cdot X^m \cdot Y^n = X (Y^i) +
\Gamma^{i}_{mn} \cdot X^m \cdot Y^n
\ese

This last equation tells us the freedom we have left with on choosing a structure, after imposing the conditions on
$\nabla_X$ acting on an $(1,0)$ tensor field, i.e.\ a vector field. Namely, we need to define all the components of
$\Gamma^{i}_{nm}$ i.e.\ $(\dim\,M)^3$-many functions in order to define a directional derivative of a vector field. \v

Now we move on to consider the action of $\nabla_X$ in the case of a $(0,1)$ tensor field, i.e.\ a covector field $\omega$:
\bi{rCl"s}
\nabla_X \omega & = & \nabla_{\left(X^i \cibasis{x^i}\right)} \left(\omega_m dx^m\right) & (expressed in chart (U,x))\\[5pt]
& =& X^i \cdot \nabla_{\left(\cibasis{x^i}\right)} \left(\omega_m dx^m\right) & (fourth condition)\\[5pt]
& =& X^i \left(\nabla_{\left(\cibasis{x^i}\right)} \omega_m\right) dx^m + X^i \cdot \omega_m \cdot \left
(\nabla_{\left(\cibasis{x^i}\right)} dx^m\right) & (third condition) \\[5pt]
& = & X^i \cibasis{x^i} \omega_m dx^m + X^i \cdot \omega_m \cdot \left (\nabla_{\left(\cibasis{x^i}\right)}
dx^m\right) & (first condition)
\ei

\v

Once again we have left with the term $\nabla_{\cibasis{x^i}}\left (dx^m\right)$ which with similar reasoning as
before, it must be again a covector field that can be expressed in terms of the dual coordinate induced basis i.e.\ :
\bse
\nabla_{\cibasis{x^i}}\left(dx^m\right) = \Sigma^{m}_{ij} dx^j
\ese


The question now is, are these $\Sigma$'s independent of $\Gamma$'s? We just showed that, on a chart domain $U$, the
choice of the $(\dim\,M)^3$-many functions $\Gamma$'s suffices to fix the action of $\nabla_X$ on a vector field. Do
we need another $(\dim\,M)^3$-many functions $\Sigma$'s to fix the action on a covector field? And if it so, will we
have to provide more and more coefficients for all individual cases of $(p, q)$ tensor field? \v

Fortunately the answer is no! The same $(\dim\,M)^3$-many functions $\Gamma$'s fix the action of $\nabla$ on any
tensor field. Let's first show that the $\Sigma$'s are actually related to $\Gamma$'s. \v

Consider the following:
\bse
\nabla_{\cibasis{x^i}} \left(dx^m \left(\cibasis{x^j}\right)\right) dx^j =
\left( \nabla_{\cibasis{x^i}} \delta^m_j \right) dx^j =\left( \cibasis{x^i}(\delta^m_j) \right) dx^j = 0
\ese

However, it also is:
\begingroup
\allowdisplaybreaks
\bi{rCl}
\nabla_{\cibasis{x^i}} \left(dx^m \left(\cibasis{x^j}\right)\right) dx^j &
= & \left(\nabla_{\cibasis{x^i}} dx^m \right)\left(\cibasis{x^j}\right) dx^j + dx^m \left(\nabla_{\cibasis{x^i}}
\cibasis{x^j}\right) dx^j \\ [5pt]
& = & \left(\nabla_{\cibasis{x^i}} dx^m \right)\left(\cibasis{x^j}\right) dx^j + dx^m \left( \Gamma^{q}_{ij}
\cibasis{x^q} \right) dx^j \\ [5pt]
& = & \left(\nabla_{\cibasis{x^i}} dx^m \right) \left(\cibasis{x^j} dx^j \right) + \Gamma^{q}_{ij} \left( dx^m
\cibasis{x^q} \right) dx^j \\ [5pt]
& = & \left(\nabla_{\cibasis{x^i}} dx^m \right) \delta^j_j + \Gamma^{q}_{ij} \cdot \delta^m_q \cdot dx^j \\ [5pt]
& = & \left(\nabla_{\cibasis{x^i}} dx^m \right) + \Gamma^{m}_{ij} \cdot dx^j \\ [5pt]
& = & \Sigma^{m}_{ij} dx^j + \Gamma^{m}_{ij} \cdot dx^j \\ [5pt]
& = & \left( \Sigma^{m}_{ij} + \Gamma^{m}_{ij} \right) dx^j
\ei
\endgroup

\v

But since, as we showed, this is equal to zero we end up with:
\bse
\left( \Sigma^{m}_{ij} + \Gamma^{m}_{ij} \right) = 0 \implies \Sigma^{m}_{ij} + \Gamma^{m}_{ij}= 0 \implies
\Sigma^{m}_{ij} = - \Gamma^{m}_{ij}
\ese

Hence:
\bse
\nabla_x \omega = X^i \left(\cibasis{x^i} \omega_m \right) dx^m - X^i \cdot \omega_m \cdot \Gamma^{m}_{iq} \cdot dx^q
\ese

\v

Thus, by rearranging the indices and the terms and by discarding the basis vectors:
\bse
(\nabla_x \omega)_i = X^m\left(\cibasis{x^m} \omega_i \right) - \Gamma^{n}_{mi} \cdot X^m \cdot \omega_n = X
(\omega_i) -\Gamma^{n}_{mi} \cdot X^m \cdot \omega_n
\ese

To sum up:
\bi{rCl}
\left(\nabla_X Y\right)^i = X (Y^i) + \Gamma^{i}_{mn} \cdot X^m \cdot Y^n \\[5pt]
(\nabla_x \omega)_i = X (\omega_i) -\Gamma^{n}_{mi} \cdot X^m \cdot \omega_n
\ei

On top of that, by making use of the Leibnitz rule we can find the action of $\nabla_X$ on any $(p,q)$-tensor field.
For example for a $(1,2)$-tensor field $T$:
\bi{rCl}
\left(\nabla_X T\right)^{i}_{jk} = X\left(T^{i}_{jk} \right) +
\Gamma^{i}_{ms} T^{s}_{jk} X^m - \Gamma^{s}_{mj} T^{i}_{sk} X^m - \Gamma^{s}_{mk} T^{i}_{js} X^m
\ei

Hence, we showed that these $(\dim\,M)^3$-many functions $\Gamma$'s suffices to fix the action of $\nabla_X$ on any $
(p,q)$-vector field and thus this is all we need to chose in order to completely define the structure of the connection.

\bd [Connection Coefficient Functions]
Given a manifold with a connection $(M, \mathcal{O}, \mathscr{A}, \nabla_X)$ and a chart $(U,x) \in \mathscr{A}$,
then the \textbf{connection coefficient functions} $\Gamma$'s on $M$ of $\nabla_X$ w.r.t $(U,x)$ are $(\dim\,M)
^3$-many functions given by:
\bi{rCl}
\Gamma_{i}^{jk} : \quad U & \to & \R \\ p & \mapsto &\Gamma^{i}_{jk}(p) \coloneqq
\left(dx^i \left(\nabla_{\left (\cibasis{x^j}\right)} \cibasis{x^k}\right)\right)(p)
\ei
\ed

One important aspect of the connection coefficient functions is their transformation law under a change of
coordinates. Namely, let $(U,x)$, $(V,y) \in \mathscr{A}$ and $U \cap V \neq \emptyset$. Then for the connection
coefficient functions in $(V,y)$ holds:
\bi{rCl}
{\Gamma^{i}_{jk}}_{(y)} & \coloneqq & dy^i \left(\nabla_{\cibasis{y^j}} \cibasis{y^k} \right) \\[5pt]
& = & \cibasis[y^i]{x^q} dx^q \left(\nabla_{\cibasis[x^p]{y^j} \cibasis{x^p}} \cibasis[x^s]{y^k} \cibasis{x^s} \right)\\[5pt]
& = & \cibasis[y^i]{x^q} dx^q \left(\cibasis[x^p]{y^j} \left[ \left (\nabla_{\cibasis{x^p}} \cibasis[x^s]{y^k} \right)
\cibasis{x^s} + \cibasis[x^s]{y^k} \left(\nabla_{\cibasis{x^p}} \cibasis{x^s} \right) \right] \right) \\[5pt]
& = & \cibasis[y^i]{x^q} \cibasis[x^p]{y^j} \cibasis{x^p} \cibasis[x^s]{y^k} \delta^q_s + \cibasis[y^i]{x^q}
\cibasis[x^p]{y^j} \cibasis[x^s]{y^k} {\Gamma^{q}_{sp}}_{(x)} \\[5pt]
&=& \cibasis[y^i]{x^q} \cibasis[x^p]{y^j} \cibasis[x^s]{y^k} {\Gamma^{q}_{sp}}_{(x)} + \cibasis[y^i]{x^q}
\cibasis{y^j} \cibasis[x^q]{y^k}\\[5pt]
&=& \cibasis[y^i]{x^q} \cibasis[x^p]{y^j} \cibasis[x^s]{y^k} {\Gamma^{q}_{sp}}_{(x)} +
\cibasis[y^i]{x^q}{\frac{\partial^2 x^q}{\partial y^j \partial y^k}}
\ei

\v

Notice that the change of connection coefficient function under the change of chart:
\bse
(U\cap V,x) \to (U\cap V,y)
\ese

do not follow the usual transformation law of a tensor due to the second term of the equation. If that part was
missing then $\Gamma$'s would be the components of a tensor. Notice also that this extra term carries a second
derivative hence, for linear transformation between coordinates in two charts, the term:
\bse
\frac{\partial^2 x^q}{\partial y^j \partial y^k}
\ese

always vanishes and then the transformation law resembles the one of a tensor (without meaning that $Gamma$'s became
the components of a tensor suddenly). \v

$\Gamma$'s transformation law has another great impact. Since the second term, the one destroying the tensor
components transformation law, does not carry and $\Gamma$'s, that means that even if $\Gamma$'s are zero in one
chart, they might be non-zero in the other charts (unless the chart transition maps are all linear, however, there is
no reason not to select a coordinate which is not a linear transformation of another one). In other words just the
change of coordinate system can suddenly give non-vanishing $\Gamma$'s even if we didn't have them in the fist
coordinate system. \v

That leads us to the following definition/theorem that we will not prove.

\bd [Normal Coordinates]
Let $(M, \mathcal{O}, \mathscr{A}, \nabla_X)$ be a manifold with a connection. Having chosen a point $p$, one can
construct a chart $(U,x)$ with $p \in U$ such that the symmetric part of $\Gamma$'s vanish at the point $p$ (not
necessarily in any neighbourhood). That is:
\bse
\forall \, p \in M, \, \exists \, (U,x) \in \mathscr{A} \, : \, p \in U : {\Gamma^{i}_{(jk)}}_{(x)}(p) = 0
\ese

Such $(U,x)$ is called a \textbf{normal coordinate chart} of $\nabla_X$ at $p \in M$.
\ed

\subsection{Parallel Transport}

Parallel transport is a way of transporting geometrical data along smooth curves in a manifold. If the manifold is
equipped with a connection (a covariant derivative or connection on the tangent bundle), then this connection allows
one to transport vectors of the manifold along curves so that they stay parallel with respect to the connection. \v

The parallel transport for a connection thus supplies a way of, in some sense, moving the local geometry of a
manifold along a curve: that is, of connecting the geometries of nearby points. There may be many notions of parallel
transport available, but a specification of one, one way of connecting up the geometries of points on a curve, is
tantamount to providing a connection. In fact, the usual notion of connection is the infinitesimal analogue of
parallel transport. Or, vice versa, parallel transport is the local realization of a connection. \v

As parallel transport supplies a local realization of the connection, it also supplies a local realization of the
curvature that we will see in a while.

\bd [Parallel Transport]
Let $(M, \mathcal{O}, \mathscr{A}, \nabla_X)$ be a smooth manifold with connection. A vector field $Y$ on $M$ is said
to be \textbf{parallely transported} along a smooth curve $\gamma: \mathbb{R} \to M$ if :
\bse
\nabla_X Y = 0
\ese

where $X$ is a vector field along this curve $\gamma$, i.e.\ $X(p) = X_{\gamma,p}, \, \forall p \in M$.
\ed

Even though parallely transported sounds like an action, it is really a property. \v

A slightly weaker condition is the notion of ``parallel''.

\bd [Parallel]
Let $(M, \mathcal{O}, \mathscr{A}, \nabla_X)$ be a smooth manifold with connection. A vector field $Y$ on $M$ is said
to be \textbf{parallel} along a smooth curve $\gamma: \mathbb{R} \to M$ if:
\bse
\nabla_X Y = \mu Y
\ese

where $\mu$ is a function $\mu : \mathbb{R} \to \mathbb{R}$.
\ed

Finally, we also introduce the crucial concept of autoparallel transport.

\bd [Autoparallel Transport]
Let $(M, \mathcal{O}, \mathscr{A}, \nabla_X)$ be a smooth manifold with connection. A vector field $X$ on $M$ is said
to be \textbf{autoparallely transported} along a smooth curve $\gamma: \mathbb{R} \to M$ if:
\bse
\nabla_X X = 0
\ese
\ed

We can manipulate the autoparallel transport equation by invoking a coordinate induced basis and get one of the most
heavily used equation in physics as follows:
\bi{rCl}
\left(\nabla_X X\right)^i &=& X (X^i) + \Gamma^{i}_{mn} \cdot X^m \cdot X^n \\[5pt]
&=& X^m \left(\cibasis{x^m} X^i\right) + \Gamma^{i}_{mn} \cdot X^m \cdot X^n \\[5pt]
&=& \ddot{\gamma}^i + \Gamma^{i}_{mn} \cdot \dot{\gamma}^m \cdot \dot{\gamma}^n
\ei

where in the last line we used a different notation for the directional derivative, namely:
\bi{rrCl}
X^m \cl & \mathcal{C}^\infty(M) & \xrightarrow{\sim} & \mathcal{C}^\infty(M) \\
& f & \mapsto & ((f\circ\gamma)'(0))^m \coloneqq \dot{\gamma}^m
\ei

and we also used a fact that we can show (but we won't):
\bse
X^m \left(\cibasis{x^m} X^i\right) = \ddot{\gamma}^i
\ese

Hence, the autoparallel transport equations reads:
\bse
\ddot{\gamma}^i + \Gamma^{i}_{mn} \cdot \dot{\gamma}^m \cdot \dot{\gamma}^n = 0
\ese

which is the chart expression of the condition that $\gamma$ be autoparallely transported. \v

This equation is very important in physics since makes the connection between Newton's theory and Einstein's
relativity theory. \v

Let's see an example.

\be
In Euclidean plane having a chart:
\bse
(U = \mathbb{R}^2, x = id_{\mathbb{R}^2}), \Gamma^{i}_{jk}{(x)} = 0
\ese

leads to the following autoparallely transported equation:
\bse
\ddot{\gamma}_{(x)}^m = 0 \implies \gamma_{(x)}^m (\lambda) = a^m \lambda + b^m
\ese

where $a,b \in \mathbb{R}^d$, which is the uniform motion.
\ee

Let's see a slightly more complicated example.

\be
Consider the round sphere $(S^2, \mathcal{O}, \mathscr{A}, \nabla_{round}$). Consider the chart $x(p) = (\theta,
\phi)$ where $\theta \in (0,\pi)$ and $\phi \in (0, 2\pi)$. In this chart $\nabla_{round}$ is given by:
\bi{rCl}
\Gamma^{1}_{22}{(x)}\left(x^{-1}(\theta,\phi)\right) & \coloneqq & - \sin\theta \cos\theta \\[5pt]
\Gamma^{2}_{12}{(x)}\left(x^{-1}(\theta,\phi)\right) = \Gamma^{2}_{21}{(x)}\left(x^{-1}(\theta,\phi)\right)
& \coloneqq & \cot\theta
\ei

and all other $\Gamma$'s vanish. \v

Then, by using the sloppy notation (familiar to us from classical mechanics) i.e.\ $x^1(p) = \theta(p)$ and $x^2(p) =
\phi(p)$, the autoparallel transport equation reads:
\bi{rCl}
\ddot{\theta} + \Gamma^{1}_{22} \dot{\phi}\dot{\phi} &= 0 \\[5pt]
\ddot{\phi} + 2 \Gamma^{2}_{12} \dot{\theta}\dot{\phi} &= 0
\ei

or:
\bi{rCl}
\ddot{\theta} - \sin\theta \cos\theta \dot{\phi}\dot{\phi} &= 0 \\[5pt]
\ddot{\phi} + 2 \cot\theta \dot{\theta}\dot{\phi} &= 0
\ei

which is exactly what one would obtain by the Lagrange equations. It can be seen that the above equations are
satisfied at the equator where $\theta(\lambda) = \pi/2$, and $\phi(\lambda) = \omega\lambda + \phi_0$ (running
around the equator at constant speed $\omega$). Thus, this curve is autoparallel. However, $\phi(\lambda) =
\omega\lambda^2 + \phi_0$ wouldn't be autoparallel.
\ee

\subsection{Torsion}

We say that the connection coefficients are not a tensor. The question now is, is any tensorial information in the
connection? In other words can we use $\nabla_{X}$ to define tensors on $(M, \mathcal{O}, \mathscr{A},\nabla_{X})$?

\bd [Torsion]
The \textbf{torsion} of a connection $\nabla_X$ is the $(1,2)$-tensor field:
\bse
T(\omega,X,Y) \coloneqq \omega(\nabla_X Y - \nabla_Y X - [X,Y])
\ese
where $[X,Y]$, called the commutator of $X$ and $Y$ is a vector field defined by $[X,Y]f \coloneqq X(Yf) - Y(Xf)$.
\ed

The definition of torsion is actually more like a theorem since we need to show that what we defined is actually a
tensor. What we need to do is to check that $T$ is $C^{\infty}$-linear in each entry:
\bit
\item $T(f\omega, X, Y) = f\omega(\nabla_{X} Y - \nabla_Y (X) - [X,Y]) = fT(\omega, X, Y)$.
\item $T(\omega + \psi, X, Y) = (\omega + \psi)(\nabla_{X} Y - \nabla_Y (X) - [X,Y]) = T(\omega, X, Y) + T(\psi, X, Y)$.
\item It is:
\begin{align*}
T(\omega, X+Z, Y) & = \omega(\nabla_{X+Z} Y - \nabla_Y (X+Z) - [(X+Z), Y]) \\
& = \omega(\nabla_{X} Y + \nabla_{Z} Y - \nabla_{Y} X - \nabla_{Y} Z - (X+Z) Y + Y (X+Z)) \\
& = \omega(\nabla_{X} Y + \nabla_{Z} Y - \nabla_{Y} X + \nabla_{Y} Z - X (Y) - Z(Y) + Y(X) + Y(Z)) \\
& = \omega(\nabla_{X} Y - \nabla_{Y} X - X(Y) + Y(X)) + \omega (\nabla_{Z} Y - \nabla_{Y} Z - Z(Y) + Y(Z)) \\
& = \omega(\nabla_{X} Y - \nabla_{Y} X - [X,Y]) + \omega(\nabla_{Z} Y - \nabla_{Y} Z - [Z,Y]) \\
& = T(\omega, X, Y) + T(\omega, Z, Y)
\end{align*}

Since $T(\omega,X,Y) = - T(\omega,Y,X)$, scaling in the last factor need not be checked separately.

\item Finally, it is:
\begin{align*}
T(\omega, fX, Y) & = \omega(\nabla_{fX} Y - \nabla_Y (fX) - [fX,Y]) \\
& = \omega(f\nabla_{X} Y - (\nabla_Y (f))X - f(\nabla_Y X) - [fX,Y]) \\
& = \omega(f\nabla_{X} Y - (Yf)X - f(\nabla_Y X) - [fX,Y]) \\
& = \omega(f\nabla_{X} Y - (Yf)X - f(\nabla_Y X) - fX(Y) + Y(fX)) \\
& = \omega(f\nabla_{X} Y - (Yf)X - f(\nabla_Y X) - fX(Y) + Y(f)X + fY (X))\\
& = \omega(f\nabla_{X} Y - (Yf)X - f(\nabla_Y X) - (fX(Y) - fY(X))+ Y (f)X )\\
& = \omega(f\nabla_{X} Y - (Yf)X - f(\nabla_Y X) - f[X,Y] + (Yf)X) \\
& = \omega(f\nabla_{X} Y - f(\nabla_Y X) - f[X,Y]) \\
& = f\omega(\nabla_{X} Y - (\nabla_Y X) - [X,Y]) \\
&= fT(\omega,X,Y)
\end{align*}

Since $T(\omega,X,Y) = - T(\omega,Y,X)$, additivity in the last factor need not be checked separately.
\eit

Now the question is what is exactly the information that torsion provides about the connection. The answer for that
can be seen by invoking a chart, since the tensor components of torsion (or coefficient function if you like) are:
\begin{align*}
T^{i}_{ab} & \coloneqq T\left(dx^i, \cibasis{x^a}, \cibasis{x^b}\right) \\[5pt]
&= dx^i \left(\nabla_{\cibasis{x^a}} \left(\cibasis{x^b}\right) - \nabla_{\cibasis{x^b}} \left( \cibasis{x^a}\right)
- \left[ \cibasis{x^a},\cibasis{x^b} \right] \right) \\[5pt]
&= dx^i \left( \Gamma^{c}_{ab} \cibasis{x^c} -\Gamma^{c}_{ba} \cibasis{x^c} \right)\\[5pt]
&= \Gamma^{c}_{ab} dx^i \left( \cibasis{x^c} \right) -\Gamma^{c}_{ba} dx^i \left(\cibasis{x^c})\right)\\[5pt]
&= \Gamma^{c}_{ab} \delta^{i}_{c} -\Gamma^{c}_{ba} \delta^{i}_{c} = \Gamma^{i}_{ab} - \Gamma^{i}_{ba} = 2 \Gamma^{i}_{[ab]}
\end{align*}

\v

Hence, the torsion is simply the antisymmetric part of connection coefficients which is also the part that one cannot
get rid of with coordinate transformation, which of course make sense since if it is not zero in one chart it will
not be zero everywhere.

\bd [Torsion-Free]
A manifold with connection $(M,\mathcal{O}, \mathscr{A},\nabla_X)$ is called \textbf{torsion-free} if the torsion of
its connection is zero. That is, $T = 0$.
\ed

\subsection{Curvature}

In this section we will show how the connection is related to the curvature of the underlying manifold. Let's start
with an intuitive picture. In general, if we parallel transport a vector from a point $p$ to a point $q$ along two
different paths (path 1) and (path 2), the resulting vectors at point $q$ are different. \v

We said, in general, because, for example, if we parallel transport a vector in a Euclidean space, where the parallel
transport is defined in our usual sense, the resulting vector does not depend on the path along which it has been
parallel transported. This non-integrability of parallel transport characterizes the intrinsic notion of curvature,
which does not depend on the special coordinates chosen. \v

Since the notion of parallel transport contains the connection, one expects that the connection has some information
regarding the curvature of the manifold. And indeed this is the case. (Recall that the connection is just a
generalization of the directional derivative). In order to formalize this mathematically, we ask what is the
difference in applying two connections (here maybe the term covariant derivatives is more intuitive) in different
order, i.e.\ what is $\nabla_X \nabla_Y Z - \nabla_Y \nabla_X Z$ (where this can be seen as the difference between path
1 and path 2). In other words we ask by how much the covariant derivatives fail to commute. The so called ``Riemann
curvature'' measures exactly this failure.

\vspace{-15pt}

\fig{riem}{0.35}

\vspace{-15pt}

\bd [Riemann Curvature]
The \textbf{Riemann curvature} of a connection $\nabla_X$ is the $(1,3)$-tensor field:
\bse
Riem(\omega,Z,X,Y) \coloneqq \omega(\nabla_X \nabla_Y Z - \nabla_Y \nabla_X Z - \nabla_{[X,Y]} Z)
\ese

\v

where it can be shown that it is $C^{\infty}$-linear in each slot.
\ed

So indeed, as we can see:
\bse
\nabla_X \nabla_Y Z - \nabla_Y \nabla_X Z = Riem(\cdot,Z,X,Y) + \nabla_{[X,Y]} Z
\ese

In other word the difference we are interested in is the Riemann tensor plus a correction term. For this correction
term note that by invoking a chart $(U,x)$, and by using the coordinate induced basis:
\bse
\left(\nabla_{\cibasis{x^a}} \nabla_{\cibasis{x^b}} Z \right)^m - \left (\nabla_{\cibasis{x^b}} \nabla_{\cibasis{x^a}}
Z \right)^m = R^{m}_{nab} \, Z^n + \nabla_{\left[ \cibasis{x^a}, \cibasis{x^b} \right]} Z
\ese

However, basis vectors always commute so the last term vanishes, and now we can see how the $Riem$ tensor components
$R^{m}_{nab}$ contain all the information about how the $\nabla_{\cibasis{x^a}}$ and $\nabla_{\cibasis{x^b}}$ fail to
commute if they act on a vector field. Similarly, if they act on a tensor field, there are several terms on the right
hand side of the equation like the one term above and if they act on a function, of course they commute. Observe
that, being a tensor, $Riem$ vanishes in all coordinate systems if it vanishes in one coordinate system. \v

Now let's try to calculate the components of Riemann tensor:
\begin{align*}
R^{\rho}_{\sigma \mu \nu} & = Riem(dx^{\rho},\cibasis{x^\mu}, \cibasis{x^\nu},\cibasis{x^\sigma}) \\[5pt]
&= dx^{\rho} \left(\nabla_{\cibasis{x^\mu}} \nabla_{\cibasis{x^ \nu}} \left(\cibasis{x^ \sigma}\right) -
\nabla_{\cibasis{x^ \nu}} \nabla_{\cibasis{x^ \mu}} \left (\cibasis{x^\sigma}\right) - \nabla_{\left[
\cibasis{x^\mu},\cibasis{x^\nu} \right]} \left(\cibasis{x^\sigma}\right) \right) \\[5pt]
&= dx^{\rho} \left(\nabla_{\cibasis{x^\mu}} \left( \Gamma^{\tau}_{\nu \sigma} \cibasis{x^\tau} \right) -
\nabla_{\cibasis{x^\nu}} \left( \Gamma^{\tau}_{\mu \sigma} \cibasis{x^\tau} \right) \right) \\[5pt]
&= dx^{\rho} \left( \nabla_{\cibasis{x^\mu}} \left( \Gamma^{\tau}_{\nu\sigma} \right) \cdot \cibasis{x^\tau} +
\Gamma^{\tau}_{\nu \sigma} \nabla_{\cibasis{x^\mu}} \left(\cibasis{x^\tau} \right) - \nabla_{\cibasis{x^\nu}}
\left( \Gamma^{\tau}_{\mu \sigma} \right) \cdot \cibasis{x^\tau} - \Gamma^{\tau}_{\mu \sigma}
\nabla_{\cibasis{x^\nu}} \left( \cibasis{x^\tau} \right) \right) \\[5pt]
&= dx^{\rho} \left( \cibasis{x^\mu} \left( \Gamma^{\tau}_{\nu \sigma}\right) \cdot \cibasis{x^\tau} +
\Gamma^{\tau}_{\nu \sigma} \Gamma^{\lambda}_{\mu \tau} \cibasis{x^\lambda} - \cibasis{x^\nu}
\left(\Gamma^{\tau}_{\mu \sigma} \right) \cdot \cibasis{x^\tau} - \Gamma^{\tau}_{\mu \sigma} \Gamma^{\lambda}_{\nu \tau}
\cibasis{x^\lambda} \right)\\[5pt]
&= \cibasis{x^\mu} \left( \Gamma^{\tau}_{\nu \sigma} \right) dx^{\rho} \left( \cibasis{x^\tau} \right)
+ \Gamma^{\tau}_{\nu \sigma} \Gamma^{\lambda}_{\mu \tau} dx^{\rho} \left (\cibasis{x^\lambda} \right) -
\cibasis{x^\nu} \left( \Gamma^{\tau}_{\mu \sigma} \right) dx^{\rho} \left( \cibasis{x^\tau} \right) -
\Gamma^{\tau}_{\mu \sigma} \Gamma^{\lambda}_{\nu \tau} dx^{\rho} \left(\cibasis{x^\lambda} \right)\\[5pt]
&= \cibasis{x^\mu} \left( \Gamma^{\tau}_{\nu \sigma} \right) \delta^{\rho}_{\tau} +\Gamma^{\tau}_{\nu \sigma}
\Gamma^{\lambda}_{\mu \tau} \delta^{\rho}_{\lambda}- \cibasis{x^\nu} \left( \Gamma^{\tau}_{\mu \sigma} \right)
\delta^{\rho}_{\tau} - \Gamma^{\tau}_{\mu \sigma} \Gamma^{\lambda}_{\nu \tau} \delta^{\rho}_{\lambda}\\[5pt]
&= \cibasis{x^\mu} \left( \Gamma^{\rho}_{\nu \sigma} \right) + \Gamma^{\tau}_{\nu \sigma} \Gamma^{\rho}_{\mu \tau}
- \cibasis{x^\nu} \left( \Gamma^{\rho}_{\mu \sigma} \right) - \Gamma^{\tau}_{\mu \sigma} \Gamma^{\rho}_{\nu \tau}
\end{align*}

Hence:
\bse
\cibasis{x^\mu} \left( \Gamma^{\rho}_{\nu \sigma} \right) - \cibasis{x^\nu} \left( \Gamma^{\rho}_{\mu \sigma} \right)
+ \Gamma^{\rho}_{\mu \tau} \Gamma^{\tau}_{\nu \sigma} - \Gamma^{\rho}_{\nu \tau} \Gamma^{\tau}_{\mu \sigma}
\ese

\v

Coming back to the Euclidean space given that the connection is 0, it is straightforward to see that $Riem = 0$ hence
there is no curvature and this is why parallel transport does not depend on the path along which is performed.

\section{Metric Manifolds}
In this chapter we will establish yet another structure on a smooth manifold that allows one to assign vectors in
each tangent space a length (and an angle between vectors in the same tangent space). From this structure, one can
then define a notion of length of a curve. Then we can look at shortest curves (which will be called geodesics). \v

Requiring then that the shortest curves coincide with the straight curves (w.r.t. $\nabla_X$) will result in
$\nabla_X$ being determined by the metric structure $g$. $\nabla_X$, in turn determines the curvature given by $Riem$.

\bd [Metric Tensor]
A \textbf{metric} $g$ on a smooth manifold $M$ is a $(0,2)$-tensor field:
\begin{align*}
g : \Gamma(TM) \times \Gamma(TM) &\xrightarrow{ ~ } C^{\infty}(M) \\ (X, Y) & \mapsto g(X,Y)
\end{align*}

satisfying the following properties:
\bit
\item Symmetry: $g(X,Y) = g(Y,X) \quad \forall \, X, Y \in \Gamma(TM)$.
\item Non-degeneracy: The so called ``musical map'' $\flat$ defined as:
\begin{align*}
\flat : \Gamma(TM) & \to \Gamma(T^*M) \\ X & \mapsto \flat(X)
\end{align*}

where:
\bse
\flat(X)(Y) \coloneqq g(X,Y)
\ese

\v

is a $C^{\infty}$-isomorphism in other words, it is invertible.
\eit
\ed

\bd [Inverse Metric Tensor]
The \textbf{inverse metric} $(2,0)$-tensor field $g^{-1}$ with respect to a metric $g$ is the symmetric map:
\begin{align*}
g^{-1} : \Gamma(T^*M) \times \Gamma(T^*M) &\xrightarrow{ ~ } C^{\infty} (M) \\
(\omega, \sigma) & \mapsto g^{-1}(\omega, \sigma) \coloneqq \omega (\flat^{-1}(\sigma))
\end{align*}

with:
\begin{align*}
\flat^{-1} : \Gamma(T^*M) & \to \Gamma(TM) \\ \omega & \mapsto \flat^{-1} (\omega)
\end{align*}

where:
\bse
\flat^{-1} (\omega) (\sigma) \coloneqq g^{-1}(\omega,\sigma)
\ese
\ed

\v

In components: $g_{\mu \nu} = g_{\nu \mu}$ and $(g^{-1})^{\mu \nu} g_{\nu \rho} = \delta^\mu_\rho$. \v

The musical map turns a vector $X$ to a covector $(\flat(X))$. Given that $ (\flat(X))$ is a covector it carries, by
the predefined convention, a ``lower index'' $(\flat(X))_\mu$. Hence, in tensor components:
\bse
(\flat(X))_\mu \coloneqq g_{\mu \nu} X^\nu
\ese

Similarly, the the inverse musical map turns a covector $\omega$ to a vector $(\flat^{-1}(\omega))$. Given that $
(\flat^{-1}(\omega))$ is a vector it carries, by the predefined convention, an ``upper index'' $(\flat^{-1}(\omega))
^\mu$. Hence, in tensor components:
\bse
(\flat^{-1}(\omega))^\mu \coloneqq g^{\mu \nu} \omega_\nu
\ese

\v

Most of the time people drop the $\flat$ and $\flat^{-1}$ symbol and simply write $(\flat(X))_\mu \rightarrow X_\mu$
and $(\flat^{-1}(\omega))^\mu \rightarrow \omega^\mu$, hence, the equations read:
\bse
X_\mu \coloneqq g_{\mu \nu} X^m \qquad \text{and} \qquad \omega^\mu \coloneqq g^{\mu \nu} \omega_\nu
\ese

\v

which is the familiar ``lower the index'' and ``raise the index'' formula. However, the problem with these formulae is
that when we see $X_\mu$ (or $\omega^\mu$) we don't really know if it is actually a vector (or a covector) or if it's
the musical map (or inverse musical map) applied to a covector (or a vector). For this reason we will stick to the
musical map (and the inverse musical map) notation.

\be
Consider the sphere $(S^2, \mathcal{O}, \mathscr{A})$ and the chart $(U,x)$:
\begin{align*}
x^1 = \varphi \in (0,2\pi), & \quad \quad x^2 = \theta \in (0,\pi)
\end{align*}

Define the metric of the round sphere of radius $R$:
\bse
g_{ij}(x^{-1}(\theta,\varphi)) = \left[ \begin{matrix} R^2 & 0 \\ 0 & R^2\sin^2{\theta} \end{matrix} \right]
\ese

where the matrix is just a way of collecting this information.
\ee

As we mentioned in the ``vector space'' chapter, bilinear forms carry a signature. Subsequently, the metric tensor as
a bilinear form carries a signature. The signature $(v, p, r)$ of a metric tensor $g$ is the number (counted with
multiplicity) of positive, negative and zero eigenvalues of the real symmetric matrix $g_{\mu \nu}$ of the metric
tensor with respect to a basis. Alternatively, it can be defined as the dimensions of a maximal positive and null
subspace. \v

The signature completely classifies the metric up to a choice of basis. The signature is often denoted by a pair of
integers $(v, p)$ (implying $r=0$), or as an explicit list of signs of eigenvalues such as $(+, -, -, -)$ or $(-, +,
+, +)$ for the signatures $(1, 3, 0)$ and $(3, 1, 0)$, respectively.

\bd [Riemannian Metric]
A metric is called \textbf{Riemannian} if it is a metric with a positive definite signature $(v, 0)$, i.e.\ :
$(+,+, \dots,+)$.
\ed

\bd [Lorentzian Metric]
A metric is called \textbf{Lorentzian} if it is a metric with signature$(v,1)$ or $(1, p)$, i.e.\ : $(+, -, \dots, -)$.
\ed

\subsection{Geodesics}

\bd [Speed Of A Curve]
On a Riemannian metric manifold $(M, \mathcal{O}, \mathscr{A}, g)$, the \textbf{speed} of a curve at the point $p=
\gamma(\lambda)$ is the number:
\bse
s(\lambda) = \sqrt{g(X_{\gamma, \gamma(\lambda)}, X_{\gamma,\gamma(\lambda)})}
\ese
\ed

\bd [Length Of A Curve]
Let $\gamma:(0,1) \to M$ be a smooth curve. Then the \textbf{length} of $\gamma$ denoted by $L[\gamma] \in \R$ is the
number:
\bse
L[\gamma] \coloneqq \int_0^1 d\lambda \, s(\lambda) = \int_0^1 d\lambda \sqrt{g(X_{\gamma, \gamma(\lambda)},
X_{\gamma,\gamma(\lambda)})}
\ese
\ed

Note that in contrast with the usual way of teaching, velocity is more fundamental than speed and speed is more
fundamental than length! \v

\be
Reconsider the round sphere of radius $R$ with metric:
\bse
g_{ij} = \left[ \begin{matrix} R^2 & 0 \\ 0 & R^2 \sin^2{\theta} \end{matrix} \right]
\ese

Consider its equator:
\begin{align*}
\theta(\lambda) & \coloneqq (x^1 \circ \gamma)(\lambda) = \frac{\pi}{2} \implies \theta'(\lambda) = 0 \\
\varphi(\lambda) & \coloneqq (x^2 \circ \gamma)(\lambda) = 2\pi \lambda^3 \implies \varphi'(\lambda = 6\pi\lambda^2
\end{align*}

In this chart the length is:
\begin{align*}
L[\gamma] & = \int_0^1 d\lambda \sqrt{g_{ij}(x^{-1}(\theta(\lambda), \varphi(\lambda)))(x^i \circ \gamma)'
(\lambda)(x^j \circ \gamma)'(\lambda)} \\[5pt]
& = \int_0^1 d\lambda \sqrt{R^2 \cdot 0 + R^2\sin^2{(\theta(\lambda))} 36 \pi^2 \lambda^4} \\[5pt]
& = 6\pi R \int_0^1 d\lambda \lambda^2 = 6\pi R [\frac{1}{3} \lambda^3]^1_0 = 2\pi R
\end{align*}
\ee

Observe that for any reparametrization of $\gamma$, the factors from the reparametrization will cancel with the
corresponding factors coming from the metric and the result will be independent of the reparametrization (as it should
be since the length of the curve should not depend on the parametrization). This actually is a theorem that can be
easily proved (we skip the proof).

\bt[]
For any curve $\gamma : (0,1) \to M$ and any smooth, bijective and increasing reparametrization $\sigma :(0,1) \to
(0,1)$ of $\gamma$:
\bse
L[\gamma] = L[\gamma \circ \sigma]
\ese
\et

\bd [Geodesic]
A curve $\gamma : (0,1) \to M$ on a Riemannian manifold $(M, \mathcal{O}, \mathscr{A}, g)$ is called a
\textbf{geodesic} if it is a stationary curve with respect to a length functional $L$.
\ed

As before let's right the directional derivative using the ``curve notation'':
\bi{rrCl}
X^m \cl & \mathcal{C}^\infty(M) & \xrightarrow{\sim} & \mathcal{C}^\infty(M) \\
& f & \mapsto & ((f\circ\gamma)'(0))^m \coloneqq \dot{\gamma}^m
\ei

Thus, the length reads:
\bse
L[\gamma] \coloneqq \int_0^1 d\lambda \, s(\lambda) = \int_0^1 d\lambda \sqrt{g(\dot{\gamma}, \dot{\gamma})} =
\int_0^1 d\lambda \sqrt{g_{\mu \nu}\dot{\gamma^{\mu}} \dot{\gamma^{\nu}}}
\ese

\bse
0
= \delta L[\gamma] = \int_0^1 d\lambda \cdot \delta \left( \sqrt{g_{\mu \nu}\dot{\gamma^{\mu}} \dot{\gamma^{\nu}}} \right)
= \int_0^1 d\lambda \cdot \frac{\delta \left(g_{\mu \nu } \dot{\gamma^{\mu}} \dot{\gamma^{\nu}} \right)}
{2{\sqrt {g_{\mu \nu } \dot{\gamma^{\mu}} \dot{\gamma^{\nu}}}}}
\ese

Using the product rule we get:
\bse
0
= \int_0^1 d\lambda \cdot \left(\dot{\gamma^{\mu}}\dot{\gamma^{\nu}}\delta g_{\mu \nu }+g_{\mu \nu}
\dot{\delta\gamma^{\mu}}\dot{\gamma^{\nu}}+g_{\mu \nu} \dot{\gamma^{\mu}}\dot{\delta\gamma^{\nu}}\right) \\
= \int_0^1 d\lambda \cdot \left(\dot{\gamma^{\mu}}\dot{\gamma^{\nu}}\partial_{\alpha }g_{\mu \nu }\delta
x^{\alpha}+2g_{\mu \nu }\dot{\delta\gamma^{\mu}}\dot{\gamma^{\nu}}\right)
\ese

\v

Integrating by-parts the last term and dropping the total derivative (which equals to zero at the boundaries) we get
that:
\begin{align*}
0 = & \int_0^1 d\lambda \cdot \left (\dot{\gamma^{\mu}}\dot{\gamma^{\nu}}\partial _{\alpha }g_{\mu \nu }\delta
x^{\alpha }-2\delta x^{\mu }{\frac {d}{d\tau }}\left(g_{\mu \nu}\dot{\gamma^{\nu}}\right)\right)\\[5pt]
=& \int_0^1 d\lambda \cdot \left (\dot{\gamma^{\mu}}\dot{\gamma^{\nu}}\partial _{\alpha }g_{\mu \nu }\delta
x^{\alpha }- 2\delta x^{\mu } \partial _{\alpha }g_{\mu \nu } \dot{\gamma^{\alpha}} \dot{\gamma^{\nu}}-2\delta
x^{\mu }g_{\mu \nu } \ddot{\gamma^{\nu}}\right)\\[5pt]
=& \int_0^1 d\lambda \cdot \delta x^{\mu } \cdot \left(-2 g_{\mu \nu } \ddot{\gamma^{\nu}} +
\dot{\gamma^{\alpha}} \dot{\gamma^{\nu}} \partial _{\mu }g_{\alpha \nu } - 2 \partial _{\alpha } g_{\mu \nu }
\dot{\gamma^{\alpha}} \dot{\gamma^{\nu}} \right)\\[5pt]
=& \int_0^1 d\lambda \cdot \delta x^{\mu } \cdot \left(-2 g_{\mu \nu } \ddot{\gamma^{\nu}} + \dot{\gamma^{\alpha}}
\dot{\gamma^{\nu}} \partial _{\mu }g_{\alpha \nu } - \partial _{\alpha } g_{\mu \nu } \dot{\gamma^{\alpha}}
\dot{\gamma^{\nu}} - \partial _{\nu } g_{\mu \alpha } \dot{\gamma^{\nu}} \dot{\gamma^{\alpha}} \right)\\[5pt]
=& \int_0^1 d\lambda \cdot \delta x^{\mu } \cdot \left(g_{\mu \nu } \ddot{\gamma^{\nu}} + \frac{1}{2}
\dot{\gamma^{\alpha}} \dot{\gamma^{\nu}} \left( \partial _{\mu } g_{\alpha \nu } + \partial _{\nu } g_{\mu \alpha}
- \partial _{\alpha } g_{\mu \nu } \right) \right)
\end{align*}

\v

Hence, by Hamilton's principle we find that the Euler-Lagrange equation is:
\bse
g_{\mu \nu } \ddot{\gamma^{\nu}} + \frac{1}{2} \dot{\gamma^{\alpha}} \dot{\gamma^{\nu}} \left( \partial _{\mu }
g_{\alpha \nu } + \partial _{\nu } g_{\mu \alpha } - \partial _{\alpha } g_{\mu \nu } \right) = 0
\ese

Multiplying by the inverse metric tensor $g^{\mu \beta}$ we get that:
\bse
\ddot{\gamma^{\beta}} + \frac{1}{2} g^{\mu \beta} \left( \partial _{\mu } g_{\alpha \nu }
+ \partial _{\nu } g_{\mu \alpha } - \partial _{\alpha } g_{\mu \nu } \right) \dot{\gamma^{\alpha}}
\dot{\gamma^{\nu}} = 0
\ese

Thus, we get the geodesic equation:
\bse
\ddot{\gamma^{\beta}} + \Gamma^{\beta}_{\alpha \nu} \dot{\gamma^{\alpha}} \dot{\gamma^{\nu}} = 0
\ese

\v
where $\Gamma^{\beta}_{\alpha \nu}$ are called ``Christoffel symbols'' defined in terms of the metric tensor as:

\bse
\Gamma^{\beta}_{\alpha \nu} = \frac{1}{2} g^{\mu \beta} \left( \partial_{\mu } g_{\alpha \nu }
+ \partial _{\nu } g_{\mu \alpha } - \partial _{\alpha } g_{\mu \nu } \right)
\ese

\subsection{Ricci Curvature \& Ricci Scalar}

Having introduced a metric on a manifold we can introduce the following two important objects.

\bd [Ricci Curvature Tensor]
The \textbf{Ricci curvature tensor}, is a geometric object which is determined by a choice of Riemannian or
pseudo-Riemannian metric on a manifold, and it is given by:
\bse
R_{\mu \nu} = R^{\alpha}_{\mu \alpha \nu}
\ese
\ed

The Ricci curvature tensor can be considered, broadly, as a measure of the degree to which the geometry of a given
metric tensor differs locally from that of ordinary Euclidean space or pseudo-Euclidean space, and it can be
characterized by measurement of how a shape is deformed as one moves along geodesics in the space. Like the metric
tensor, the Ricci tensor assigns to each tangent space of the manifold a symmetric bilinear form. Broadly, one could
analogize the role of the Ricci curvature in Riemannian geometry to that of the Laplacian in the analysis of
functions. In this analogy, the Riemann curvature tensor, of which the Ricci curvature is a natural by-product, would
correspond to the full matrix of second derivatives of a function. However, there are other ways to draw the same
analogy. In three-dimensional topology, the Ricci tensor contains all the information which in higher dimensions
is encoded by the more complicated Riemann curvature tensor. \v

In differential geometry, lower bounds on the Ricci tensor on a Riemannian manifold allow one to extract global
geometric and topological information by comparison with the geometry of a constant curvature space form. This is
since lower bounds on the Ricci tensor can be successfully used in studying the length functional in Riemannian
geometry, as first shown in 1941 via Myers's theorem.

\bd [Ricci Scalar]
The \textbf{Ricci scalar}, is a geometric object which is determined by a choice of Riemannian or pseudo-Riemannian
metric on a manifold, and it is given by:
\bse
R = g^{\mu \nu} R_{\mu \nu}
\ese
\ed

The Ricci scalar is the simplest curvature invariant of a Riemannian manifold. To each point on a Riemannian
manifold, it assigns a single real number determined by the intrinsic geometry of the manifold near that point.
Specifically, the scalar curvature represents the amount by which the volume of a small geodesic ball in a Riemannian
manifold deviates from that of the standard ball in Euclidean space. In two dimensions, the scalar curvature is twice
the Gaussian curvature, and completely characterizes the curvature of a surface. In more than two dimensions,
however, the curvature of Riemannian manifolds involves more than one functionally independent quantity.