Microeconomics is a branch of mainstream economics that studies the behavior of individuals and firms in making
decisions regarding the allocation of scarce resources and the interactions among these individuals and firms.
Microeconomics focuses on the study of individual markets, sectors, or industries as opposed to the national economy
as whole, which is studied in macroeconomics. \v

One goal of microeconomics is to analyze the market mechanisms that establish relative prices among goods and
services and allocate limited resources among alternative uses. Microeconomics shows conditions under which free
markets lead to desirable allocations. It also analyzes market failure, where markets fail to produce efficient results.

\section{The Market Forces Of Supply And Demand}

As we have already defined, a market is a group of buyers and sellers of a particular good or service, where the
buyers as a group determine the demand for the product, and the sellers as a group determine the supply of the product.

\bd[Market]
A \textbf{market} is a place where parties can gather to facilitate the exchange of goods and services. The parties
involved are usually buyers (consumers) and sellers (producers).
\ed

Markets take many forms. Some markets are highly organized, such as the markets for agricultural commodities like
wheat and corn. In these markets, buyers and sellers meet at a specific time and place. Buyers come knowing how much
they are willing to buy at various prices, and sellers come knowing how much they are willing to sell at various
prices. An auctioneer facilitates the process by keeping order, arranging sales, and (most importantly) finding the
price that brings the actions of buyers and sellers into balance. However, more often than not, markets are less
organized.

\be
For example, consider the market for ice cream in a particular town. Buyers of ice cream do not meet together at any
one time or at any one place. The sellers of ice cream are in different locations and offer somewhat different
products. There is no auctioneer calling out the price of ice cream. Each seller posts a price for an ice cream cone,
and each buyer decides how many cones to buy at each store. \v

Nonetheless, these consumers and producers of ice cream are closely connected. The ice cream buyers are choosing from
the various ice-cream sellers to satisfy their cravings, and the ice cream sellers are all trying to appeal to the
same ice cream buyers to make their businesses successful. \v

Even though it is not as organized, the group of ice cream buyers and ice-cream sellers forms a market.
\ee

Most markets in the economy, are highly competitive. Each buyer knows that there are several sellers from which to
choose, and each seller is aware that his product is similar to that offered by other sellers.

\bd[Competition]
\textbf{Competition} is a scenario where different economic firms are in contention to obtain goods that are limited
by varying the elements of the marketing mix: price, product, promotion and place.
\ed

As a result of competition, the price and quantity of goods sold are not determined by any single buyer or seller.
Rather, price and quantity are determined by all buyers and sellers as they interact in the marketplace. Economists
use the term competitive market to describe a market in which there are so many buyers and so many sellers that each
has a negligible impact on the market price.

\bd[Competitive Market]
A \textbf{competitive market} is one where there are numerous producers that compete with one another in hopes to
provide goods and services consumers want and need. In other words, not one single producer can dictate the market.
Also, like producers, not one consumer can dictate the market either.
\ed

Each seller of a good has limited control over the price because other sellers are offering similar products. A
seller has little reason to charge less than the going price, and if he charges more, buyers will make their
purchases elsewhere. Similarly, no single buyer of a good can influence the price of the good because each buyer
purchases only a small amount. \v

Usually we distinguish between ``perfect'' and ``imperfect'' competition. Let's start with the first one.

\bd[Perfect Competition]
\textbf{Perfect competition} refers to a theoretical market structure with the following key characteristics:
\bit
\item The product is a commodity.
\item All firms are price takers.
\item Market share has no influence on prices.
\item Buyers have perfect information.
\item Capital resources and labor are perfectly mobile.
\item Firms can enter or exit the market without cost.
\eit
\ed

Because buyers and sellers in perfectly competitive markets must accept the price the market determines, they are
said to be ``price takers''. At the market price, buyers can buy all they want, and sellers can sell all they want.

\bd[Price Taker]
A \textbf{price taker} is an individual or company that must accept prevailing prices in a market, lacking the market
share to influence market price on its own.
\ed

All economic participants are considered to be price takers in a market of perfect competition or one in which all
companies sell an identical product, there are no barriers to entry or exit, every company has a relatively small
market share, and all buyers have full information of the market. This holds true for producers and consumers of
goods and services and for buyers and sellers in debt and equity markets.

\be
There are some markets in which the assumption of perfect competition applies perfectly. In the wheat market, for
example, there are thousands of farmers who sell wheat and millions of consumers who use wheat and wheat products.
Because no single buyer or seller can influence the price of wheat, each takes the market price as given.
\ee

Not all goods and services, however, are sold in perfectly competitive markets. Now let's examine imperfect competition.

\bd[Imperfect Competition]
\textbf{Imperfect competition} exists whenever a market, hypothetical or real, violates the abstract tenets of
neoclassical perfect competition.
\ed

In imperfect competition, companies sell different products and services, set their own individual prices, fight for
market share, and are often protected by barriers to entry and exit. \v

In an extreme case of an imperfect competitive market, some markets have only one seller, and this seller sets the
price. Such a market is called a ``monopolistic'' or more simply a ``monopoly''.

\bd[Monopoly]
A \textbf{monopoly} is a dominant position of an industry or a sector by one company, to the point of excluding all
other viable competitors.
\ed

\bd[Monopolistic Market]
A \textbf{monopolistic market} is a theoretical condition that describes a market where only one company may offer
products and services to the public.
\ed

A monopolistic market is the opposite of a perfectly competitive market in which an infinite number of firms operate
and there are no monopolies. In a purely monopolistic model, the monopoly firm can restrict output, raise prices, and
enjoy super-normal profits in the long run. Because of the fact that monopolies can set their price (instead of the
market decide for it) we usually call monopolies ``price makers''.

\bd[Price Maker]
A \textbf{price maker} is a company that can dictate the price it charges for its goods because there are no perfect
substitutes. These are generally monopolies or companies that produce goods or services that differ from what
competitors offer.
\ed

In reality markets fall somewhere between the extremes of perfect competition and monopoly. Despite the diversity of
market types we find in the world, assuming perfect competition is a useful simplification and a natural place to
start. \v

Perfectly competitive markets are the easiest to analyze because everyone participating in them takes the price as
given by market conditions. Moreover, because some degree of competition is present in most markets, many of the
lessons that we learn by studying supply and demand under perfect competition apply to more complex markets as well.
In what follows, we assume that markets are perfectly competitive.

\subsection{Demand}

The quantity demanded of any good is the amount of the good that buyers are willing and able to purchase.

\bd[Quantity Demanded]
\textbf{Quantity demanded} describes the total amount of a good or service that consumers demand over a given interval
of time.
\ed

As we will see, many things determine the quantity demanded of a good, but in our analysis of how markets work, one
determinant plays a central role: the good's market price, or more simply its price.

\bd[Market Price]
The \textbf{market price} is the current price at which a good or service can be bought or sold.
\ed

If the price of a good rose, you would buy less of it, or you might buy a different but similar good instead. If the
price of a good fell you would buy more. This relationship between price and quantity demanded is true for most goods
in the economy and, in fact, is so pervasive that economists call it the ``law of demand'', which simply states that,
other things being equal, when the price of a good rises, the quantity demanded of the good falls, and when the price
falls, the quantity demanded rises.

\bd[Law Of Demand]
The \textbf{law of demand} states that the quantity purchased varies inversely with price.
\ed

The law of demand occurs because of diminishing marginal utility. That is, consumers use the first units of an
economic good they purchase to serve their most urgent needs first, and then they use each additional unit of the
good to serve successively lower-valued ends.

\bd[Utility]
\textbf{Utility} is a term in economics that refers to the total satisfaction received from consuming a good or service.
\ed

Economic theories based on rational choice usually assume that consumers will strive to maximize their utility. The
economic utility of a good or service is important to understand, because it directly influences the demand, and
therefore price,of that good or service. In practice, a consumer's utility is impossible to measure and quantify.
However, some economists believe that they can indirectly estimate what is the utility for an economic good or
service by employing various models.

\bd[Marginal Utility]
\textbf{Marginal utility} is the incremental increase in utility that results from the consumption of one additional
unit.
\ed

\bd[Law Of Diminishing Marginal Utility]
The \textbf{law of diminishing marginal utility} states that all else equal, as consumption increases, the marginal
utility derived from each additional unit declines.
\ed

Now let's go back to the law of demand. A demand schedule tabulates the quantity of goods that consumers will
purchase at given prices.

\bd[Demand Schedule]
A \textbf{demand schedule} is a table that shows the quantity demanded of a good or service at different price levels.
\ed

A demand schedule can be graphed as a continuous demand curve on a chart where the Y-axis represents price and the
X-axis represents quantity.

\bd[Demand Curve]
The \textbf{demand curve} is a graphical representation of the demand schedule, i.e.\ the relationship between the
price of a good or service and the quantity demanded for a given period of time. In a typical representation, the
price will appear on the left vertical axis and the quantity demanded on the horizontal axis.
\ed

\be
\fig{e1}{1}

The table in Figure 1 shows how many ice cream cones Catherine would buy each month at different prices. If ice cream
cones are free, Catherine buys 12 cones per month. At \$1 per cone, Catherine buys 10 cones each month. As the price
rises further, she buys fewer and fewer cones. When the price reaches \$6, Catherine doesn't buy any cones at all.
This table is a demand schedule, a table that shows the relationship between the price of a good and the quantity
demanded, holding constant everything else that influences how much of the good consumers want to buy. \v

The graph in Figure 1 uses the numbers from the table to illustrate the law of demand. By convention, the price of
ice cream is on the vertical axis, and the quantity of ice cream demanded is on the horizontal axis. The line
relating price and quantity demanded is called the demand curve. The demand curve slopes downward because, other
things being equal, a lower price means a greater quantity demanded.
\ee

The demand curve in the previous example shows an individual's demand for a product. To analyze how markets work, we
need to determine the market demand, the sum of all the individual demands for a particular good or service. \v

Based on market demand we can define the notions of ``market demand schedule'' and ``market demand curve''.

\bd[Market Demand Schedule]
The \textbf{market demand schedule} is the summation of all the individual demand schedules in a given market.
\ed

\bd[Market Demand Curve]
The \textbf{market demand curve} is the summation of all the individual demand curves in a given market.
\ed

Because we are interested in analyzing how markets function, we work most often with the market demand curve. The
market demand curve shows how the total quantity demanded of a good varies as the price of the good varies, while all
other factors that affect how much consumers want to buy are held constant.

\be
\fig{e2}{0.84}

The table in Figure 2 shows the demand schedules for ice cream of the two individuals in this market—Catherine and
Nicholas. At any price, Catherine's demand schedule tells us how many cones she buys, and Nicholas's demand schedule
tells us how many cones he buys. The market demand at each price is the sum of the two individual demands. \v

The graph in Figure 2 shows the demand curves that correspond to these demand schedules. Notice that we sum the
individual demand curves horizontally to obtain the market demand curve. That is, to find the total quantity demanded
at any price, we add the individual quantities demanded, which are found on the horizontal axis of the individual
demand curves.
\ee

Because the market demand curve holds other things constant, it need not be stable over time. If something happens to
alter the quantity demanded at any given price, the demand curve shifts.

\bd[Shifts In Demand]
\textbf{Shifts in demand} is the condition where the position of the demand curve shifts shift to the left or right
following a change in an underlying determinant of demand.
\ed

The shift in demand can be either to the right (an increase) or to the left (a decrease).

\bd[Increase In Demand]
Any change that increases the quantity demanded at every price shifts the demand curve to the right and is called an
\textbf{increase in demand}.
\ed

\bd[Decrease In Demand]
Any change that reduces the quantity demanded at every price shifts the demand curve to the left and is called a
\textbf{decrease in demand}.
\ed

\be
\fig{e3}{0.81}

For example, suppose we discover that people who regularly eat ice cream live longer, healthier lives. The discovery
would raise the demand for ice cream. At any given price, buyers would now want to purchase a larger quantity of ice
cream, and the demand curve for ice cream would shift. Figure 3 illustrates shifts in demand.
\ee

Be careful! The demand curve shows what happens to the quantity demanded of a good as its price varies, holding
constant all the other variables that influence buyers. When one of these other variables changes, the quantity
demanded at each price changes, and the demand curve shifts. In other words, a curve shifts when there is a change in
a relevant variable that is not measured on either axis. Because the price is on the vertical axis, a change in price
represents a movement along the demand curve. By contrast, income, the prices of related goods, tastes, expectations,
and the number of buyers are not measured on either axis, so a change in one of these variables shifts the demand curve.

\fig{e5}{0.76}

Changes in many variables can shift the demand curve. Let's consider the most important.

\subsubsection*{Income}

What would happen to your demand if you lost your job one summer? Most likely, your demand would fall. A lower income
means that you have less to spend in total, so you would have to spend less on some—and probably most—goods. If the
demand for a good falls when income falls, the good is called a normal good.

\bd[Normal Good]
A \textbf{normal good} is a good that experiences an increase in its demand due to a rise in consumers' income and a
decrease in its demand due to a fall in consumers' income.
\ed

Normal goods are the norm, but not all goods are normal goods. If the demand for a good rises when income falls, the
good is called an inferior good.

\bd[Inferior Good]
An \textbf{inferior good} is a good whose demand drops when people's incomes rise and demand rises when people's incomes
fall.
\ed

\be
An example of an inferior good might be bus rides. As your income falls, you are less likely to buy a car or take a cab
and more likely to ride a bus.
\ee

Inferior goods fall out of favor as incomes and the economy improve as consumers begin buying more costly substitutes
instead.

\subsubsection*{Prices Of Related Goods}

Suppose that the price of a good falls. The law of demand says that you will buy more of this good. At the same time,
you will probably buy less of a related good. When a fall in the price of one good reduces the demand for another
good, the two goods are called substitutes.

\bd[Substitute]
A \textbf{substitute} refers
to a product or service that consumers see as essentially the same or similar-enough to another product. Put simply, a
substitute is a good that can be used in place of another.
\ed

Substitutes play an important part in the marketplace and are considered a benefit for consumers. They provide more
choices for consumers, who are then better able to satisfy their needs. \v

Now suppose that the price of a good falls. According to the law of demand, you will buy more of this good. Yet in
this case, you will likely buy more a good that is often consumed together with the first one. When a fall in the
price of one good raises the demand for another good, the two goods are called complements.

\bd[Complement]
A \textbf{complement} refers
to a product or service which demands increases when the price of another good falls.
\ed

Complements are often pairs of goods that are used together.

\be
Examples of complements are gasoline and automobiles, computers and software, and peanut butter and jelly.
\ee

\subsubsection*{Tastes}

Perhaps the most obvious determinant of your demand for any good or service is your tastes. If you like a product,
you buy more of it. Economists normally do not try to explain people's tastes because tastes are based on historical
and psychological forces that are beyond the realm of economics. Economists do, however, examine what happens when
tastes change.

\subsubsection*{Expectations}

Your expectations about the future may affect your demand for a good or service today. If you expect to earn a higher
income next month, you may choose to save less now and spend more of your current income on ice cream. If you expect
the price of a good to fall tomorrow, you may be less willing to buy this good at today's price.

\subsubsection*{Number Of Buyers}

In addition to the preceding factors, which influence the behavior of individual buyers, market demand depends on the
number of these buyers.

\subsection{Supply}

The quantity supplied of any good or service is the amount that sellers are willing and able to sell.

\bd[Quantity Supplied]
\textbf{Quantity supplied} describes the total amount of a good or service that producers supply over a given interval
of time.
\ed

There are many determinants of quantity supplied, but once again, price plays a special role in our analysis. When
the price of a good is high, selling the good is quite profitable, and so the quantity supplied is large. Sellers of
the good work long hours, buy many resources of the good, and hire many workers. By contrast, when the price of the
good is low, the business is less profitable, so sellers produce less of the good. At a low price, some sellers may
even shut down, reducing their quantity supplied to zero. This relationship between price and quantity supplied is
called the ``law of supply'', which simply states that, other things being equal, when the price of a good rises,the
quantity supplied of the good also rises, and when the price falls, the quantity supplied falls as well.

\bd[Law Of Supply]
The \textbf{law of supply} states that, all other factors being equal, as the price of a good or service increases,
the quantity of goods or services that suppliers offer will increase, and vice versa.
\ed

The law of supply says that as the price of an item goes up, suppliers will attempt to maximize their profits by
increasing the number of items for sale. \v

A supply schedule tabulates the quantity of goods that producers will supply at given prices.

\bd[Supply Schedule]
A \textbf{supply schedule} is a table that shows the quantity supplied of a good or service at different price levels.
\ed

A supply schedule can be graphed as a continuous supply curve on a chart where the Y-axis represents price and the
X-axis represents quantity.

\bd[Supply Curve]
The \textbf{supply curve} is a graphical representation of the supply schedule, i.e.\ the relationship between the
price of a good or service and the quantity supplied for a given period of time. In a typical representation, the
price will appear on the left vertical axis and the quantity demanded on the horizontal axis.
\ed

\be
\fig{e6}{0.71}

The table in Figure 5 shows the quantity of ice-cream cones supplied each month by Ben, an ice-cream seller, at
various prices of ice cream. At a price below \$2, Ben does not supply any ice cream at all. As the price rises, he
supplies a greater and greater quantity. This is the supply schedule, a table that shows the relationship between the
price of a good and the quantity supplied, holding constant everything else that influences how much of the good
producers want to sell. \v

The graph in Figure 5 uses the numbers from the table to illustrate the law of supply. The curve relating price and
quantity supplied is called the supply curve. The supply curve slopes upward because, other things being equal, a
higher price means a greater quantity supplied.
\ee

The supply curve in the previous example shows an individual's supply for a product. Just as market demand is the sum
of the demands of all buyers, market supply is the sum of the supplies of all sellers. To analyze how markets work,
we need to determine the market supply, the sum of all the individual supplies for a particular good or service. \v

Based on market supply we can define the notions of ``market supply schedule'' and ``market supply curve''.

\bd[Market Supply Schedule]
The \textbf{market supply schedule} is the summation of all the individual supply schedules in a given market.
\ed

\bd[Market Supply Curve]
The \textbf{market supply curve} is the summation of all the individual supply curves in a given market.
\ed

Because we are interested in analyzing how markets function, we work most often with the market supply curve. The
market supply curve shows how the total quantity supply of a good varies as the price of the good varies, while all
other factors that affect how much producers want to supply are held constant.

\be
\fig{e8}{0.65}

The table in Figure 6 shows the supply schedules for the two ice-cream producers in the market—Ben and Jerry At any
price, Ben's supply schedule tells us the quantity of ice cream that Ben supplies, and Jerry's supply schedule tells
us the quantity of ice cream that Jerry supplies. The market supply is the sum of the two individual supplies. \v

The graph in Figure 6 shows the supply curves that correspond to the supply schedules. As with demand curves, we
sum the individual supply curves horizontally to obtain the market supply curve. That is, to find the total quantity
supplied at any price, we add the individual quantities, which are found on the horizontal axis of the individual
supply curves. The market supply curve shows how the total quantity supplied varies as the price of the good varies,
holding constant all other factors that influence producers' decisions about how much to sell.
\ee

Because the market supply curve holds other things constant, it need not be stable over time. If something happens to
alter the quantity demanded at any given price, the supply curve shifts.

\bd[Shifts In Supply]
\textbf{Shifts in supply} is the condition where the position of the supply curve shifts shift to the left or right
following a change in an underlying determinant of demand.
\ed

The shift in supply can be either to the right (an increase) or to the left (a decrease).

\bd[Increase In Supply]
Any change that increases the quantity supplied at every price shifts the supply curve to the right and is called an
\textbf{increase in supply}.
\ed

\bd[Decrease In Supply]
Any change that reduces the quantity supplied at every price shifts the supply curve to the left and is called a
\textbf{decrease in supply}.
\ed

\be
\fig{e9}{0.64}

For example, suppose the price of sugar falls. Sugar is an input in the production of ice cream, so the lower price
of sugar makes selling ice cream more profitable. This raises the supply of ice cream: At any given price, sellers
are now willing to produce a larger quantity. As a result, the supply curve for ice cream shifts to the right. Figure
7 illustrates shifts in supply.
\ee

Be careful! The supply curve shows what happens to the quantity supplied of a good when its price varies, holding
constant all the other variables that influence sellers. When one of these other variables changes, the quantity
supplied at each price changes, and the supply curve shifts. Once again, to remember whether you need to shift or
move along the supply curve, keep in mind that a curve shifts only when there is a change in a relevant variable that
is not named on either axis. The price is on the vertical axis, so a change in price represents a movement along the
supply curve. By contrast, because input prices, technology, expectations, and the number of sellers are not measured
on either axis, a change in one of these variables shifts the supply curve. \v

There are many variables that can shift the supply curve. Let's consider the most important ones.

\subsubsection*{Input Prices}

To produce their output of ice cream, sellers use various inputs: cream, sugar, flavoring, ice-cream machines, the
buildings in which the ice cream is made, and the labor of workers who mix the ingredients and operate the machines.
When the price of one or more of these inputs rises, producing ice cream becomes less profitable, and firms supply
less ice cream. If input prices rise substantially, a firm might shut down and supply no ice cream at all. Thus, the
supply of a good is negatively related to the prices of the inputs used to make the good.

\subsubsection*{Technology}

The technology for turning inputs into ice cream is another determinant of supply. The invention of the mechanized
ice-cream machine, for example, reduced the amount of labor necessary to make ice cream. By reducing firms' costs,
the advance in technology raised the supply of ice cream.

\subsubsection*{Expectations}

The amount of ice cream a firm supplies today may depend on its expectations about the future. For example, if a firm
expects the price of ice cream to rise in the future, it will put some of its current production into storage and
supply less to the market today.

\subsubsection*{Number Of Sellers}

In addition to the preceding factors, which influence the behavior of individual sellers, market supply depends on
the number of these sellers. If Ben or Jerry were to retire from the ice-cream business, the supply in the market
would fall.

\subsection{Supply And Demand}

Having analyzed supply and demand separately, we now combine them to see how they determine the price and quantity of
a good sold in a market, by putting both the demand and the supply curve in the same diagram.

\bd[Market Equilibrium]
\textbf{Market equilibrium} is the state in which market supply and demand balance each other, and as a result prices
become stable.
\ed

Generally, an over-supply of goods or services causes prices to go down, which results in higher demand—while an
under-supply or shortage causes prices to go up resulting in less demand. The balancing effect of supply and demand
results in a state of equilibrium. \v

The price at this intersection is called the equilibrium price, and the quantity is called the equilibrium quantity.

\bd[Equilibrium Price / Market Clearing Price]
The market price at market equilibrium, that balances quantity supplied and quantity demanded is called
\textbf{equilibrium price} or \textbf{market equilibrium price}.
\ed

\bd[Equilibrium Quantity]
The the quantity supplied and the quantity demanded at the equilibrium price at market equilibrium is called
\textbf{equilibrium quantity}.
\ed

The dictionary defines the word equilibrium as a situation in which various forces are in balance. This definition
applies to a market's equilibrium as well. At the equilibrium price, the quantity of the good that buyers are willing
and able to buy exactly balances the quantity that sellers are willing and able to sell. The equilibrium price is
sometimes called the market-clearing price because, at this price, everyone in the market has been satisfied: Buyers
have bought all they want to buy, and sellers have sold all they want to sell.

\be
\fig{e11}{0.74}

Figure 8 shows the market supply curve and market demand curve together. Notice that there is one point at which the
supply and demand curves intersect. This point is called the market's equilibrium. Here the equilibrium price is \$4
per cone, and the equilibrium quantity is 7 ice-cream cones.
\ee

The actions of buyers and sellers naturally move markets toward the equilibrium of supply and demand. To see why,
consider what happens when the market price is not equal to the equilibrium price in an example.

\fig{e12}{0.75}

Suppose first that the market price is above the equilibrium price, as in panel (a) of Figure 9. At a price of \$5
per cone, the quantity of the good supplied (10 cones) exceeds the quantity demanded (4 cones). There is a surplus of
the good: Producers are unable to sell all they want at the going price. A surplus is sometimes called a situation of
excess supply.

\bd[Suplus / Excess Supply / Oversupply]
\textbf{Surplus} or \textbf{excess supply} or \textbf{oversupply} is an excessive amount of a product that is the
result of when demand is lower than supply, resulting in a surplus.
\ed

When there is a surplus in the ice-cream market, sellers of ice cream find their freezers increasingly full of ice
cream they would like to sell but cannot. They respond to the surplus by cutting their prices. Falling prices, in
turn, increase the quantity demanded and decrease the quantity supplied. These changes represent movements along the
supply and demand curves, not shifts in the curves. Prices continue to fall until the market reaches the equilibrium. \v

Suppose now that the market price is below the equilibrium price, as in panel (b) of Figure 9. In this case, the
price is \$3 per cone, and the quantity of the good demanded exceeds the quantity supplied. There is a shortage of
the good: Consumers are unable to buy all they want at the going price. A shortage is sometimes called a situation of
excess demand.

\bd[Shortage / Excess Demand / Overdemand]
\textbf{Shortage} or \textbf{excess demand} or \textbf{over-demand} is an excessive amount of demand of a product,
greater than quantity supplied.
\ed

When a shortage occurs in the ice-cream market, buyers have to wait in long lines for a chance to buy one of the few
cones available. With too many buyers chasing too few goods, sellers can respond to the shortage by raising their
prices without losing sales. These price increases cause the quantity demanded to fall and the quantity supplied to
rise. Once again, these changes represent movements along the supply and demand curves, and they move the market
toward the equilibrium. \v

Thus, regardless of whether the price starts off too high or too low, the activities of the many buyers and sellers
automatically push the market price toward the equilibrium price. Once the market reaches its equilibrium, all buyers
and sellers are satisfied, and there is no upward or downward pressure on the price. How quickly equilibrium is
reached varies from market to market depending on how quickly prices adjust. In most free markets, surpluses and
shortages are only temporary because prices eventually move toward their equilibrium levels. Indeed, this phenomenon
is so pervasive that it is called the law of supply and demand:

\bd[Law Of Supply And Demand]
The \textbf{law of supply and demand} states that the price of any good adjusts to bring the quantity supplied and
quantity demanded of that good into balance.
\ed

The law of supply and demand is actually based on the two separate laws of ``law of demand'' and ``law of supply''.
The two laws interact to determine the actual market price and volume of goods on the market. The law of supply and
demand in practice a theory that explains the interaction between the sellers of a resource and the buyers for that
resource. \v

So far, we have seen how supply and demand together determine a market's equilibrium, which in turn determines the
price and quantity of the good that buyers purchase and sellers produce. The equilibrium price and quantity depend on
the positions of the supply and demand curves. When some event shifts one of these curves, the equilibrium in the
market changes, resulting in a new price and a new quantity exchanged between buyers and sellers.

\subsection{Application: Changes In Market Equilibrium Due To Shifts}

When analyzing how some event affects the equilibrium in a market, we proceed in three steps.

\ben
\item Decide whether the event shifts the supply curve, the demand curve, or, in some cases, both.
\item We decide whether the curve shifts to the right or to the left.
\item We use the supply-and-demand diagram to compare the initial equilibrium with the new one, which shows how the
shift affects the equilibrium price and quantity.
\een

To see how this recipe is used, let's consider various events that might affect the market for ice cream.

\subsubsection*{A Change in Market Equilibrium Due To A Shift In Demand}

\fig{e13}{0.85}

Suppose that one summer the weather is very hot. How does this event affect the market for ice cream? To answer this
question, let's follow our three steps.

\ben
\item The hot weather affects the demand curve by changing people's taste for ice cream. That is, the weather changes
the amount of ice cream that people want to buy at any given price. The supply curve is unchanged because the weather
does not directly affect the firms that sell ice cream.
\item Because hot weather makes people want to eat more ice cream, the demand curve shifts to the right. Figure 10
shows this increase in demand as a shift in the demand curve from $D_1$ to $D_2$. This shift indicates that the
quantity of ice cream demanded is higher at every price.
\item At the old price of \$4, there is now an excess demand for ice cream, and this shortage induces firms to raise
the price. As Figure 10 shows, the increase in demand raises the equilibrium price from \$4 to \$5 and the
equilibrium quantity from 7 to 10 cones. In other words, the hot weather increases both the price of ice cream and
the quantity of ice cream sold.
\een

Notice that when hot weather increases the demand for ice cream and drives up the price, the quantity of ice cream
that firms supply rises, even though the supply curve remains the same. In this case, economists say there has been
an increase in ``quantity supplied'' but no change in ``supply''. Supply refers to the position of the supply curve,
whereas the quantity supplied refers to the amount producers wish to sell. In this example, supply does not change
because the weather does not alter firms' desire to sell at any given price. Instead, the hot weather alters
consumers' desire to buy at any given price and thereby shifts the demand curve to the right. The increase in demand
causes the equilibrium price to rise. When the price rises, the quantity supplied rises. This increase in quantity
supplied is represented by the movement along the supply curve. \v

\subsubsection*{A Change In Market Equilibrium Due To A Shift In Supply}

\fig{e14}{0.85}

Suppose that during another summer, a hurricane destroys part of the sugarcane crop and drives up the price of sugar.
How does this event affect the market for ice cream? Once again, to answer this question, we follow our three steps.

\ben
\item The change in the price of sugar, an input for making ice cream, affects the supply curve. By raising the costs
of production, it reduces the amount of ice cream that firms produce and sell at any given price. The demand curve
does not change because the higher cost of inputs does not directly affect the amount of ice cream consumers wish to
buy.
\item The supply curve shifts to the left because, at every price, the total amount that firms are willing and able
to sell is reduced. Figure 11 illustrates this decrease in supply as a shift in the supply curve from $S_1$ to $S_2$.
\item At the old price of \$4, there is now an excess demand for ice cream, and this shortage causes firms to raise
the price. As Figure 11 shows, the shift in the supply curve raises the equilibrium price from \$4 to \$5 and lowers
the equilibrium quantity from 7 to 4 cones. As a result of the sugar price increase, the price of ice cream rises,
and the quantity of ice cream sold falls.
\een

\subsubsection*{A Change In Market Equilibrium Due To Shifts In Both Supply And Demand}

\fig{e15}{0.85}

Now suppose that the heat wave and the hurricane occur during the same summer. To analyze this combination of events,
we again follow our three steps.

\ben
\item We determine that both curves must shift. The hot weather affects the demand curve because it alters the amount
of ice cream that consumers want to buy at any given price. At the same time, when the hurricane drives up sugar
prices, it alters the supply curve for ice cream because it changes the amount of ice cream that firms want to sell
at any given price.
\item The curves shift in the same directions as they did in our previous analysis: The demand curve shifts to the
right, and the supply curve shifts to the left. Figure 12 illustrates these shifts.
\item As Figure 12 shows, two possible outcomes might result depending on the relative size of the demand and supply
shifts. In both cases, the equilibrium price rises. In panel (a), where demand increases substantially while supply
falls just a little, the equilibrium quantity also rises. By contrast, in panel (b), where supply falls substantially
while demand rises just a little, the equilibrium quantity falls. Thus, these events certainly raise the price of ice
cream, but their impact on the amount of ice cream sold is ambiguous (that is, it could go either way).
\een

We have just seen three examples of how to use supply and demand curves to analyze a change in equilibrium. Whenever
an event shifts the supply curve, the demand curve, or perhaps both curves, you can use these tools to predict how
the event will alter the price and quantity sold in equilibrium. \v

This section has analyzed supply and demand in a single market. Our examples has centered on the market for ice
cream, but the lessons learned here apply to most other markets as well. Whenever you go to a store to buy something,
you are contributing to the demand for that item. Whenever you look for a job, you are contributing to the supply of
labor services. Because supply and demand are such pervasive economic phenomena, the model of supply and demand is a
powerful tool for analysis. We use this model repeatedly in the following chapters.

\section{Elasticity}

\bd[Elasticity]
\textbf{Elasticity} is a measure of a variable's sensitivity to a change in another variable, most commonly this
sensitivity is the change in quantity demanded relative to changes in other factors, such as price.
\ed

Elasticity is a measure of how much buyers and sellers respond to changes in market conditions. When studying how
some event or policy affects a market, we can discuss not only the direction of the effects but also their magnitude.
Elasticity is useful in many applications, as we see toward the end of this section.

\subsection{The Elasticity Of Demand}

When we introduced demand in the previous section, we noted that consumers usually buy more of a good when its price
is lower, when their incomes are higher, when the prices of its substitutes are higher, or when the prices of its
complements are lower. Our discussion of demand was qualitative, not quantitative. That is, we discussed the
direction in which quantity demanded moves but not the size of the change. To measure how much consumers respond to
changes in these variables, economists use the concept of elasticity. \v

The law of demand states that a fall in the price of a good raises the quantity demanded. The price elasticity of
demand measures how much the quantity demanded responds to a change in price.

\bd[Price Elasticity Of Demand]
\textbf{Price elasticity of demand} is a measurement of the change in the quantity demanded of a product in relation
to a change in its price.

\bse
\text{Price Elasticity Of Demand} = \frac{\text{Percentage Change In Quantity Demanded}}
{\text{Percentage Change In Price}}
\ese
\ed

Because the quantity demanded of a good is negatively related to its price, the percentage change in quantity will
always have the opposite sign as the percentage change in price. For this reason, price elasticities of demand are
sometimes reported as negative numbers. In this notes, we follow the common practice of using absolute values, hence
dropping the minus sign, and reporting all price elasticities of demand as positive numbers. With this convention, a
larger price elasticity implies a greater responsiveness of quantity demanded to changes in price. \v

In addition to the price elasticity of demand, economists use other elasticities to describe the behavior of buyers
in a market like the ``income elasticity of demand'' and the ``cross-price elasticity of demand'' that we will define
now.

\bd[Income Elasticity Of Demand]
The \textbf{income elasticity of demand} measures how the quantity demanded changes as consumer income changes.

\bse
\text{Income Elasticity Of Demand} = \frac{\text{Percentage Change In Quantity Demanded}}
{\text{Percentage Change In Income}}
\ese
\ed

\bd[Cross-Price Elasticity Of Demand]
The \textbf{cross-price elasticity of demand} measures how the quantity demanded of one good responds to a change in
the price of another good.

\bse
\text{Cross-Price Elasticity Of Demand} = \frac{\text{Percentage Change In Quantity Demanded Of Good A}}
{\text{Percentage Change In Price Of Good B}}
\ese
\ed

For now we will stick to the price elasticity of demand. \v

Economists classify demand curves according to their elasticity. Demand is considered ``elastic'' when the elasticity
is greater than one, which means the quantity moves proportionately more than the price.

\bd[Elastic Demand]
Demand for a good is said to be \textbf{elastic} if the quantity demanded responds substantially to changes in the
price.

\bse
\text{Price Elasticity Of Demand} > 1
\ese
\ed

In the extreme case where the price elasticity of demand approaches infinity and the demand curve becomes horizontal,
the demand is considered ``perfectly elastic''.

\bd[Perfectly Elastic Demand]
Demand for a good is said to be \textbf{perfectly elastic} if very small changes in the price lead to huge changes in
the quantity demanded.

\bse
\text{Price Elasticity Of Demand} \to \infty
\ese
\ed

Demand is considered inelastic when the elasticity is less than one, which means the quantity moves proportionately
less than the price.

\bd[Inelastic Demand]
Demand for a good is said to be \textbf{inelastic} if the quantity demanded responds only slightly to changes in the
price.

\bse
\text{Price Elasticity Of Demand} < 1
\ese
\ed

In the extreme case of a zero elasticity, demand said to be ``perfectly inelastic'' and the demand curve is vertical.

\bd[Perfectly Inelastic Demand]
Demand for a good is said to be \textbf{perfectly inelastic} if regardless of the price, the quantity demanded stays
the same.

\bse
\text{Price Elasticity Of Demand} = 0
\ese
\ed

If the elasticity is exactly one, the percentage change in quantity equals the percentage change in price, and demand
is said to have unit elasticity.

\bd[Unit Elastic Demand]
Demand for a good is said to be \textbf{unit elastic} if the percentage change in quantity equals the percentage
change in price.

\bse
\text{Price Elasticity Of Demand} = 1
\ese
\ed

Figure 1 shows five cases. In the extreme case of a zero elasticity, shown in panel (a), demand is perfectly
inelastic, and the demand curve is vertical. In this case, regardless of the price, the quantity demanded stays the
same. As the elasticity rises, the demand curve gets flatter and flatter, as shown in panels (b), (c), and (d). At
the opposite extreme, shown in panel (e), demand is perfectly elastic. This occurs as the price elasticity of demand
approaches infinity and the demand curve becomes horizontal, reflecting the fact that very small changes in the price
lead to huge changes in the quantity demanded.

\fig{e20}{0.67}

Because the price elasticity of demand measures how much quantity demanded responds to changes in the price, it is
closely related to the slope of the demand curve. The following rule of thumb is a useful guide: The flatter the
demand curve passing through a given point, the greater the price elasticity of demand. The steeper the demand curve
passing through a given point, the smaller the price elasticity of demand. \v

The price elasticity of demand for any good measures how willing consumers are to buy less of the good as its price
rises. Because a demand curve reflects the many economic, social, and psychological forces that shape consumer
preferences, there is no simple, universal rule for what determines a demand curve's elasticity. Based on experience,
however, we can state some rules of thumb about what influences the price elasticity of demand.

\subsubsection*{Availability Of Close Substitutes}

A good with close substitutes tends to have more elastic demand because it is easier for consumers to switch from
that good to others.

\be
For example, butter and margarine are easily substitutable. A small increase in the price of butter, assuming the
price of margarine is held fixed, causes the quantity of butter sold to fall by a large amount. By contrast, because
eggs are a food without a close substitute, the demand for eggs is less elastic than the demand for butter. A small
increase in the price of eggs does not cause a sizable drop in the quantity of eggs sold.
\ee

\subsubsection*{Necessities VS Luxuries}

Necessities tend to have inelastic demands, whereas luxuries have elastic demands. (Whether a good is a necessity or
a luxury depends not on the good's intrinsic properties but on the buyer's preferences)

\be
When the price of a doctor's visit rises, people do not dramatically reduce the number of times they go to the
doctor, although they might go somewhat less often. By contrast, when the price of sailboats rises, the quantity of
sailboats demanded falls substantially. The reason is that most people view doctor visits as a necessity and
sailboats as a luxury.
\ee

\subsubsection*{Definition Of The Market}

The elasticity of demand in any market depends on how we draw the boundaries of the market. Narrowly defined markets
tend to have more elastic demand than broadly defined markets because it is easier to find close substitutes for
narrowly defined goods.

\be
For example, food, a broad category, has a fairly inelastic demand because there are no good substitutes for food.
Ice cream, a narrow category, has a more elastic demand because it is easy to substitute other desserts for ice cream.
Vanilla ice cream, an even narrower category, has a very elastic demand because other flavors of ice cream are almost
perfect substitutes for vanilla.
\ee

\subsubsection*{Time Horizon}

Goods tend to have more elastic demand over longer time horizons.

\be
When the price of gasoline rises, the quantity of gasoline demanded falls only slightly in the first few months. Over
time, however, people buy more fuel-efficient cars, switch to public transportation, and move closer to where they
work. Within several years, the quantity of gasoline demanded falls more substantially.
\ee

\subsection{The Elasticity Of Supply}

When we introduced supply in the previous section, we noted that producers of a good offer to sell more of it when
the price of the good rises. To turn from qualitative to quantitative statements about quantity supplied, we once
again use the concept of elasticity. \v

The law of supply states that higher prices raise the quantity supplied. The price elasticity of supply measures how
much the quantity supplied responds to changes in the price.

\bd[Price Elasticity Of Supply]
\textbf{Price elasticity of supply} is a measurement of the change in the quantity supplied of a product in relation to
a change in its price.

\bse
\text{Price Elasticity Of Supply} = \frac{\text{Percentage Change In Quantity Supplied}}
{\text{Percentage Change In Price}}
\ese
\ed

The price elasticity of supply depends on the flexibility of sellers to change the amount of the good they produce.
In most markets, a key determinant of the price elasticity of supply is the time period being considered. Supply is
usually more elastic in the long run than in the short run. Over short periods of time, firms cannot easily change
the size of their factories to make more or less of a good. Thus, in the short run, the quantity supplied is not very
responsive to changes in the price. Over longer periods of time, firms can build new factories or close old ones. In
addition, new firms can enter a market, and old firms can exit. Thus, in the long run, the quantity supplied can
respond substantially to price changes. \v

In addition to the price elasticity of supply, economists use other elasticities to describe the behavior of sellers
in a market like the ``income elasticity of supply'' and the ``cross-price elasticity of supply'' that we will define
now.

\bd[Income Elasticity Of Supply]
The \textbf{income elasticity of supply} measures how the quantity supplied changes as consumer income changes.

\bse
\text{Income Elasticity Of Supply} = \frac{\text{Percentage Change In Quantity Supplied}}
{\text{Percentage Change In Income}}
\ese
\ed

\bd[Cross-Price Elasticity Of Supply]
The \textbf{cross-price elasticity of supply} measures how the quantity supplied of one good responds to a change in the
price of another good.

\bse
\text{Cross-Price Elasticity Of Demand} = \frac{\text{Percentage Change In Quantity Supplied Of Good A}}
{\text{Percentage Change In Price Of Good B}}
\ese
\ed

For now we will stick to the price elasticity of supply. \v

Economists classify supply curves according to their elasticity. Supply is considered ``elastic'' when the elasticity
is greater than one, which means the quantity moves proportionately more than the price.

\bd[Elastic Supply]
Supply for a good is said to be \textbf{elastic} if the quantity supplied responds substantially to changes in the price.

\bse
\text{Price Elasticity Of Supply} > 1
\ese
\ed

In the extreme case where the price elasticity of supply approaches infinity and the demand curve becomes horizontal,
the supply is considered ``perfectly elastic''.

\bd[Perfectly Elastic Supply]
Supply for a good is said to be \textbf{perfectly elastic} if very small changes in the price lead to huge changes in
the quantity demanded.

\bse
\text{Price Elasticity Of Supply} \to \infty
\ese
\ed

Supply is considered inelastic when the elasticity is less than one, which means the quantity moves proportionately
less than the price.

\bd[Inelastic Supply]
Supply for a good is said to be \textbf{inelastic} if the quantity demanded responds only slightly to changes in the
price.

\bse
\text{Price Elasticity Of Supply} < 1
\ese
\ed

In the extreme case of a zero elasticity, supply said to be ``perfectly inelastic'' and the supply curve is vertical.

\bd[Perfectly Inelastic Supply]
Supply for a good is said to be \textbf{perfectly inelastic} if regardless of the price, the quantity demanded stays the
same.

\bse
\text{Price Elasticity Of Supply} = 0
\ese
\ed

If the elasticity is exactly one, the percentage change in quantity equals the percentage change in price, and supply
is said to have unit elasticity.

\bd[Unit Elastic Supply]
Supply for a good is said to be \textbf{unit elastic} if the percentage change in quantity equals the percentage change
in price.

\bse
\text{Price Elasticity Of Supply} = 1
\ese
\ed

Figure 5 shows these five cases. In the extreme case of zero elasticity, as shown in panel (a), supply is perfectly
inelastic and the supply curve is vertical. In this case, the quantity supplied is the same regardless of the price.
As the elasticity rises, the supply curve gets flatter, which shows that the quantity supplied responds more to
changes in the price. At the opposite extreme, shown in panel (e), supply is perfectly elastic. This occurs as the
price elasticity of supply approaches infinity and the supply curve becomes horizontal, meaning that very small
changes in the price lead to very large changes in the quantity supplied.

\fig{e26}{0.55}

Because the price elasticity of supply measures the responsiveness of quantity supplied to changes in price, it is
reflected in the appearance of the supply curve. The following rule of thumb is a useful guide: The flatter the
supply curve passing through a given point, the greater the price elasticity of supply. The steeper the supply curve
passing through a given point, the smaller the price elasticity of supply.

\section{Applications Of Supply, Demand, And Elasticity}

\subsection{Various}

\subsubsection*{Elasticity And Total Revenue}

When studying changes in supply or demand in a market, one variable we often want to study is total revenue, the
amount paid by buyers and received by sellers of a good.

\bd[Total Revenue]
\textbf{Total revenue} is the amount paid by buyers and received by sellers of a good, computed as the price of the good
times the quantity sold.

\bse
\text{Total Revenue} = Price \times Quantity
\ese
\ed

We can show total revenue graphically, as in Figure 2. The height of the box under the demand curve is $P$, and the
width is $Q$. The area of this box, $P \times Q$ equals the total revenue in this market.

\fig{e22}{0.9}

How does total revenue change as one moves along the demand curve? The answer depends on the price elasticity of
demand. If demand is inelastic, as in panel (a) of Figure 3, then an increase in the price causes an increase in
total revenue, because the extra revenue from selling units at a higher price (represented by area A in the figure)
more than offsets the decline in revenue from selling fewer units (represented by area B). We obtain the opposite
result if demand is elastic: An increase in the price causes a decrease in total revenue, because the extra revenue
from selling units at a higher price (area A) is smaller than the decline in revenue from selling fewer units (area
B). \v

The examples in Figure 3 illustrate some general rules:
\bit
\item When demand is inelastic (a price elasticity less than one), price and total revenue move in the same direction:
If the price increases, total revenue also increases.
\item When demand is elastic (a price elasticity greater than one), price and total revenue move in opposite directions:
If the price increases, total revenue decreases.
\item If demand is unit elastic (a price elasticity exactly equal to one), total revenue remains constant when the price
changes.
\eit

\fig{e23}{0.9}

Now let's be more specific and let's examine how elasticity varies along a linear demand curve,as shown in Figure 4. \v

We know that a straight line has a constant slope. Slope here is the ratio of the change in price to the change in
quantity. This particular demand curve's slope is constant because each \$1 increase in price causes the same
two-unit decrease in the quantity demanded. \v

Even though the slope of a linear demand curve is constant, the elasticity is not. This is true because the slope is
the ratio of changes in the two variables, whereas the elasticity is the ratio of percentage changes in the two
variables. You can see this by looking at the table in Figure 4, which shows the demand schedule for the linear
demand curve in the graph. The table uses the midpoint method to calculate the price elasticity of demand. The table
illustrates the following: At points with a low price and high quantity, the demand curve is inelastic. At points
with a high price and low quantity, the demand curve is elastic. \v

The explanation for this fact comes from the arithmetic of percentage changes. When the price is low and consumers
are buying a lot, a \$1 price increase and two-unit reduction in quantity demanded constitute a large percentage
increase in the price and a small percentage decrease in quantity demanded, resulting in a small elasticity. When the
price is high and consumers are not buying much, the same \$1 price increase and two-unit reduction in quantity
demanded constitute a small percentage increase in the price and a large percentage decrease in quantity demanded,
resulting in a large elasticity. \v

The table also presents total revenue at each point on the demand curve. These numbers illustrate the relationship
between total revenue and elasticity. When the price is \$1, for instance, demand is inelastic and a price increase
to \$2 raises total revenue. When the price is \$5, demand is elastic and a price increase to \$6 reduces total
revenue. Between \$3 and \$4, demand is exactly unit elastic and total revenue is the same at these two prices. \v

The linear demand curve illustrates that the price elasticity of demand need not be the same at all points on a
demand curve. A constant elasticity is possible, but it is not always the case, and it is never the case for a linear
demand curve.

\fig{e25}{0.9}


\subsubsection*{Can Good News For Farming Be Bad News For Farmers?}

Imagine you're a Kansas wheat farmer. Because you earn all your income from selling wheat, you devote much effort to
making your land as productive as possible. You monitor weather and soil conditions, check your fields for pests and
disease, and study the latest advances in farm technology. You know that the more wheat you grow, the more you will
have to sell after the harvest, and the higher your income and standard of living will be. \v

One day, Kansas State University announces a major discovery. Researchers in its agronomy department have devised a
new hybrid of wheat that raises the amount farmers can produce from each acre of land by 20 percent. How should you
react to this news? Does this discovery make you better off or worse off than you were before?

\fig{e30}{0.8}

Recall from previous section that we answer such questions in three steps. First, we examine whether the supply or
demand curve shifts. Second, we consider the direction in which the curve shifts. Third, we use the supply-and-demand
diagram to see how the market equilibrium changes. In this case, the discovery of the new hybrid affects the supply
curve. Because the hybrid increases the amount of wheat that can be produced on each acre of land, farmers are now
willing to supply more wheat at any given price. In other words, the supply curve shifts to the right. The demand
curve remains the same because consumers' desire to buy wheat products at any given price is not affected by the
introduction of a new hybrid. Figure 7 shows an example of such a change. When the supply curve shifts from $S_1$ to
$S_2$, the quantity of wheat sold increases from 100 to 110 and the price of wheat falls from \$3 to \$2.

Does this discovery make farmers better off? Let's consider what happens to the total revenue they receive. Farmers'
total revenue is $P \times Q$, the price of the wheat times the quantity sold. The discovery affects farmers in two
conflicting ways. The hybrid allows farmers to produce more wheat (Q rises), but now each bushel of wheat sells for
less (P falls). \v

The price elasticity of demand determines whether total revenue rises or falls. In practice, the demand for basic
foodstuffs such as wheat is usually inelastic because these items are relatively inexpensive and have few good
substitutes. When the demand curve is inelastic, as it is in Figure 7, a decrease in price causes total revenue to
fall. You can see this in the figure: The price of wheat falls substantially, whereas the quantity of wheat sold
rises only slightly. Total revenue falls from \$300 to \$220. Thus, the discovery of the new hybrid lowers the total
revenue that farmers receive from the sale of their crops. \v

If farmers are made worse off by the discovery of this new hybrid, one might wonder why they adopt it. The answer
goes to the heart of how competitive markets work. Because each farmer represents only a small part of the market for
wheat, she takes the price of wheat as given. For any given price of wheat, it is better to use the new hybrid to
produce and sell more wheat. Yet when all farmers do this, the supply of wheat increases, the price falls, and
farmers are worse off. \v

This example may at first seem hypothetical, but it helps explain a major change in the U.S economy over the past
century. Two hundred years ago, most Americans lived on farms. Knowledge about farm methods was sufficiently
primitive that most Americans had to be farmers to produce enough food to feed the nation's population. But over
time, advances in farm technology increased the amount of food that each farmer could produce. This increase in food
supply, together with the inelastic demand for food, caused farm revenues to fall, which in turn encouraged people to
leave farming. A few numbers show the magnitude of this historic change. As recently as 1950, 10 million people
worked on farms in the United States, representing 17 percent of the labor force. Today, fewer than 3 million people
work on farms, representing less than 2 percent of the labor force. This change coincided with great advances in farm
productivity: Despite the large drop in the number of farmers, U.S farms now produce about five times as much output
as they did in 1950. \v

When analyzing the effects of farm technology or farm policy, it is important to keep in mind that what is good for
farmers is not necessarily good for society as a whole. Improvement in farm technology can be bad for farmers because
it makes farmers increasingly unnecessary, but it is surely good for consumers who pay less for food. Similarly, a
policy aimed at reducing the supply of farm products may raise the incomes of farmers, but it does so at the expense
of consumers.

\subsubsection*{Why Did OPEC Fail To Keep The Price Of Oil High?}

Many of the most disruptive events for the world's economies over the past several decades have originated in the
world market for oil. In the 1970s, members of the Organization of Petroleum Exporting Countries (OPEC) decided to
raise the world price of oil to increase their incomes. These countries accomplished this goal by agreeing to jointly
reduce the amount of oil they supplied. As a result, the price of oil (adjusted for overall inflation) rose more than
50 percent from 1973 to 1974. Then, a few years later, OPEC did the same thing again. From 1979 to 1981, the price of
oil approximately doubled. \v

Yet OPEC found it difficult to maintain such a high price. From 1982 to 1985, the price of oil steadily declined
about 10 percent per year. Dissatisfaction and disarray soon prevailed among the OPEC countries. In 1986, cooperation
among OPEC members completely broke down, and the price of oil plunged 45 percent. In 1990, the price of oil
(adjusted for overall inflation) was back to where it began in 1970, and it stayed at that low level throughout most
of the 1990s. \v

The OPEC episodes of the 1970s and 1980s show how supply and demand can behave differently in the short run and in
the long run. In the short run, both the supply and demand for oil are relatively inelastic. Supply is inelastic
because the quantity of known oil reserves and the capacity for oil extraction cannot be changed quickly. Demand is
inelastic because buying habits do not respond immediately to changes in price. Thus, as panel (a) of Figure 8 shows,
the short-run supply and demand curves are steep. When the supply of oil shifts from $S_1$ to $S_2$, the price
increase from $P_1$ to $P_2$ is large.

\fig{e31}{0.79}

The situation is very different in the long run. Over long periods of time, producers of oil outside OPEC respond to
high prices by increasing oil exploration and by building new extraction capacity. Consumers respond with greater
conservation, such as by replacing old inefficient cars with newer efficient ones. Thus, as panel (b) of Figure 8
shows, the long-run supply and demand curves are more elastic. In the long run, the shift in the supply curve from
$S_1$ to $S_2$ causes a much smaller increase in the price. \v

This analysis shows why OPEC succeeded in maintaining a high price of oil only in the short run. When OPEC countries
agreed to reduce their production of oil, they shifted the supply curve to the left. Even though each OPEC member
sold less oil, the price rose by so much in the short run that OPEC incomes rose. In the long run, however, supply
and demand are more elastic. As a result, the same reduction in supply, measured by the horizontal shift in the
supply curve, caused a smaller increase in the price. Thus, OPEC's coordinated reduction in supply proved less
profitable in the long run. The cartel learned that raising prices is easier in the short run than in the long run.

\subsubsection*{Does Drug Interdiction Increase Or Decrease Drug-Related Crime?}

A persistent problem facing our society is the use of illegal drugs, such as heroin, cocaine, ecstasy, and
methamphetamine. Drug use has several adverse effects. One is that drug dependence can ruin the lives of drug users
and their families. Another is that drug addicts often turn to robbery and other violent crimes to obtain the money
needed to support their habit. To discourage the use of illegal drugs, the U.S government devotes billions of dollars
each year to reducing the flow of drugs into the country. Let's use the tools of supply and demand to examine this
policy of drug interdiction. \v

Suppose the government increases the number of federal agents devoted to the war on drugs. What happens in the market
for illegal drugs? As usual, we answer this question in three steps. First, we consider whether the supply or demand
curve shifts. Second, we consider the direction of the shift. Third, we see how the shift affects the equilibrium
price and quantity. Although the purpose of drug interdiction is to reduce drug use, its direct impact is on the
sellers of drugs rather than on the buyers. When the government stops some drugs from entering the country and
arrests more smugglers, it raises the cost of selling drugs and, therefore, reduces the quantity of drugs supplied at
any given price. The demand for drugs—the amount buyers want at any given price—remains the same. As panel (a) of
Figure 9 shows, interdiction shifts the supply curve to the left from $S_1$ to $S_2$ without changing the demand
curve. The equilibrium price of drugs rises from $P_1$ to $P_2$, and the equilibrium quantity falls from $Q_1$ to
$Q_2$. The fall in the equilibrium quantity shows that drug interdiction does reduce drug use.

\fig{e32}{0.79}

But what about the amount of drug-related crime? To answer this question, consider the total amount that drug users
pay for the drugs they buy. Because few drug addicts are likely to break their destructive habits in response to a
higher price, it is likely that the demand for drugs is inelastic, as it is drawn in the figure. If demand is
inelastic, then an increase in price raises total revenue in the drug market. That is, because drug interdiction
raises the price of drugs proportionately more than it reduces drug use, it raises the total amount of money that
drug users pay for drugs. Addicts who already had to steal to support their habits would now have an even greater
need for quick cash. Thus, drug interdiction could increase drug-related crime. \v

Because of this adverse effect of drug interdiction, some analysts argue for alternative approaches to the drug
problem. Rather than trying to reduce the supply of drugs, policymakers might try to reduce the demand by pursuing a
policy of drug education. Successful drug education has the effects shown in panel (b) of Figure 9. The demand curve
shifts to the left from $D_1$ to $D_2$. As a result, the equilibrium quantity falls from $Q_1$ to $Q_2$, and the
equilibrium price falls from $P_1$ to $P_2$. Total revenue, $P \times Q$, also falls. Thus, in contrast to drug
interdiction, drug education can reduce both drug use and drug-related crime. \v

Advocates of drug interdiction might argue that the long-run effects of this policy are different from the short-run
effects because the elasticity of demand depends on the time horizon. The demand for drugs is probably inelastic over
short periods because higher prices do not substantially affect drug use by established addicts. But it may be more
elastic over longer periods because higher prices would discourage experimentation with drugs among the young and,
over time, lead to fewer drug addicts. In this case, drug interdiction would increase drug-related crime in the short
run but decrease it in the long run. \v

Now let's switch gears, and examine what happens under price controls.

\subsection{Price Controls}

\bd[Price Controls]
The term \textbf{price controls} refers to the legal minimum or maximum prices set for specified goods.
\ed

Price controls are normally mandated by the government in the free market. They are usually implemented as a means of
direct economic intervention to manage the affordability of certain goods and services, including rent, gasoline, and
food. Although it may make certain goods and services more affordable, price controls can often lead to disruptions
in the market, losses for producers, and a noticeable change in quality. \v

To see how price controls affect market outcomes, let's look once again at the market for ice cream. As we saw
previously, if ice cream is sold in a competitive market free of government regulation, the price of ice cream
adjusts to balance supply and demand: At the equilibrium price, the quantity of ice cream that buyers want to buy
exactly equals the quantity that sellers want to sell. To be concrete, let's suppose that the equilibrium price is
\$3 per cone. \v

Some people may not like the outcome of this free-market process. The American Association of Ice-Cream Eaters
complains that the \$3 price is too high for everyone to enjoy a cone a day (their recommended daily allowance).
Meanwhile, the National Organization of Ice-Cream Makers complains that the \$3 price—the result of ``cutthroat
competition''—is too low and is depressing the incomes of its members. Each of these groups lobbies the government to
pass laws that alter the market outcome by directly controlling the price of an ice-cream cone. \v

Because buyers of any good always want a lower price while sellers want a higher price, the interests of the two
groups conflict. If the Ice-Cream Eaters are successful in their lobbying, the government imposes a legal maximum on
the price at which ice-cream cones can be sold. Because the price is not allowed to rise above this level, the
legislated maximum is called a price ceiling.

\bd[Price Ceiling]
A \textbf{price ceiling} is the mandated maximum amount a seller is allowed to charge for a product or service.
\ed

By contrast, if the Ice-Cream Makers are successful, the government imposes a legal minimum on the price. Because the
price cannot fall below this level, the legislated minimum is called a price floor. Let us consider the effects of
these policies in turn.

\bd[Price Floor]
A \textbf{price floor} is the mandated minimum amount a seller is allowed to charge for a product or service.
\ed

When the government, moved by the complaints and campaign contributions of the Ice-Cream Eaters, imposes a price
ceiling in the market for ice cream, two outcomes are possible. In panel (a) of Figure 1, the government imposes a
price ceiling of \$4 per cone. In this case, because the price that balances supply and demand (\$3) is below the
ceiling, the price ceiling is not binding. Market forces move the economy to the equilibrium, and the price ceiling
has no effect on the price or the quantity sold. Panel (b) of Figure 1 shows the other, more interesting, possibility.
In this case, the government imposes a price ceiling of \$2 per cone. Because the equilibrium price of \$3 is above
the price ceiling, the ceiling is a binding constraint on the market. The forces of supply and demand tend to move
the price toward the equilibrium price, but when the market price hits the ceiling, it cannot, by law, rise any
further. Thus, the market price equals the price ceiling. At this price, the quantity of ice cream demanded (125
cones in the figure) exceeds the quantity supplied (75 cones). \v

Because of this excess demand of 50 cones, some people who want to buy ice cream at the going price are unable to do
so. In other words, there is a shortage of ice cream. In response to this shortage, some mechanism for rationing ice
cream will naturally develop. The mechanism could be long lines: Buyers who are willing to arrive early and wait in
line get a cone, while those unwilling to wait do not. Alternatively, sellers could ration ice-cream cones according
to their own personal biases, selling them only to friends, relatives, or members of their own racial or ethnic group.
Notice that even though the price ceiling was motivated by a desire to help buyers of ice cream, not all buyers
benefit from the policy. Some buyers pay a lower price, although they may have to wait in line to do so, but other
buyers cannot get any ice cream at all.

\fig{e33}{0.75}

This example in the market for ice cream shows a general result: When the government imposes a binding price ceiling
on a competitive market, a shortage of the good arises, and sellers must ration the scarce goods among the large
number of potential buyers. The rationing mechanisms that develop under price ceilings are rarely desirable. Long
lines are inefficient because they waste buyers' time. Discrimination according to seller bias is both inefficient
(because the good may not go to the buyer who values it most) and often unfair. By contrast, the rationing mechanism
in a free, competitive market is both efficient and impersonal. When the market for ice cream reaches its
equilibrium, anyone who wants to pay the market price can get a cone. Free markets ration goods with prices. \v

Now let's examine the effects of price floors, to the market for ice cream. Imagine now that the National
Organization of Ice-Cream Makers persuades the government that the \$3 equilibrium price is too low. In this case,
the government might institute a price floor. Price floors, like price ceilings, are an attempt by the government to
maintain prices at other than equilibrium levels. Whereas a price ceiling places a legal maximum on prices, a price
floor places a legal minimum. \v

When the government imposes a price floor on the ice-cream market, two outcomes are possible. If the government
imposes a price floor of \$2 per cone when the equilibrium price is \$3, we obtain the outcome in panel (a) of Figure
4. In this case, because the equilibrium price is above the floor, the price floor is not binding. Market forces
move the economy to the equilibrium, and the price floor has no effect. Panel (b) of Figure 4 shows what happens when
the government imposes a price floor of \$4 per cone. In this case, because the equilibrium price of \$3 is below the
floor, the price floor is a binding constraint on the market. The forces of supply and demand tend to move the price
toward the equilibrium price, but when the market price hits the floor, it can fall no further. The market price
equals the price floor. At this floor, the quantity of ice cream supplied (120 cones) exceeds the quantity demanded
(80 cones). \v

Because of this excess supply of 40 cones, some people who want to sell ice cream at the going price are unable to do
so. Thus, a binding price floor causes a surplus. Just as the shortages resulting from price ceilings can lead to
undesirable rationing mechanisms, so can the surpluses resulting from price floors. The sellers who appeal to the
personal biases of the buyers, perhaps due to racial or familial ties, may be better able to sell their goods than
those who do not. By contrast, in a free market, the price serves as the rationing mechanism, and sellers can sell
all they want at the equilibrium price.

\fig{e38}{0.75}

One of the Ten Principles of Economics is that markets are usually a good way to organize economic activity. This
principle explains why economists often oppose price ceilings and price floors. To economists, prices are not the
outcome of some haphazard process. Prices, they contend, are the result of the millions of business and consumer
decisions that lie behind the supply and demand curves. Prices have the crucial job of balancing supply and demand
and, thereby, coordinating economic activity. When policymakers set prices by legal decree, they obscure the signals
that normally guide the allocation of society's resources. \v

Another one of the Ten Principles of Economics is that governments can sometimes improve market outcomes. Indeed,
policymakers are motivated to control prices because they view the market's outcome as unfair. Price controls are
often aimed at helping the poor. For instance, rent-control laws try to make housing affordable for everyone, and
minimum-wage laws to help people escape poverty. \v

Yet when policymakers impose price controls, they can hurt some people they are trying to help. Rent control keeps
rents low, but it also discourages landlords from maintaining their buildings and makes housing hard to find. Minimum
wage laws raise the incomes of some workers, but they also cause other workers to become unemployed. \v

Helping those in need can be accomplished in ways other than controlling prices. For instance, the government can
make housing more affordable by paying a fraction of the rent for poor families. Unlike rent control, such rent
subsidies do not reduce the quantity of housing supplied and, therefore, do not lead to housing shortages. Similarly,
wage subsidies raise the living standards of the working poor without discouraging firms from hiring them. An example
of a wage subsidy is the earned income tax credit, a government program that supplements the incomes of low-wage
workers. \v

Although these alternative policies are often better than price controls, they are not perfect. Rent and wage
subsidies cost the government money and, therefore, require higher taxes. As we see in the next section, taxation has
costs of its own.

\subsubsection*{Lines At The Gas Pump (Price Ceiling)}

As we discussed in a previous application, in 1973 the Organization of Petroleum Exporting Countries (OPEC) reduced
production of crude oil, thereby increasing its price in world oil markets. Because crude oil is used to make
gasoline, the higher oil prices reduced the supply of gasoline. Long lines at gas stations became common, with
motorists often waiting for hours to buy a few gallons of gas. \v

What was responsible for the long gas lines? Most people blame OPEC. Surely, if OPEC had not reduced production of
crude oil, the shortage of gasoline would not have occurred. Yet economists blame the U.S government regulations that
limited the price oil companies could charge for gasoline. Figure 2 reveals what happened. As panel (a) shows, before
OPEC raised the price of crude oil, the equilibrium price of gasoline, $P_1$, was below the price ceiling. The price
regulation, therefore, had no effect. When the price of crude oil rose, however, the situation changed. The increase
in the price of crude oil raised the cost of producing gasoline and thereby reduced the supply of gasoline. As panel
(b) shows, the supply curve shifted to the left from $S_1$ to $S_2$. In an unregulated market, this shift in supply
would have raised the equilibrium price of gasoline from $P_1$ to $P_2$, and no shortage would have occurred.
Instead, the price ceiling prevented the price from rising to the equilibrium level. At the price ceiling, producers
were willing to sell QS, but consumers were willing to buy QD. Thus, the shift in supply caused a severe shortage at
the regulated price.

\fig{e34}{0.75}

Eventually, the laws regulating the price of gasoline were repealed. Lawmakers came to understand that they were
partly responsible for the many hours Americans lost waiting in line to buy gasoline. Today, when the price of crude
oil changes, the price of gasoline can adjust to bring supply and demand into equilibrium.

\subsubsection*{Rent Control In The Short Run And Long Run (Price Ceiling)}

One common example of a price ceiling is rent control. In many cities, the local government places a ceiling on rents
that landlords may charge their tenants. The goal of this policy is to help the poor by making housing more
affordable. Economists often criticize rent control, arguing that it is a highly inefficient way to help the poor
raise their standard of living. One economist called rent control ``the best way to destroy a city, other than
bombing''. \v

The adverse effects of rent control are less apparent to the general population because these effects occur over many
years. In the short run, landlords have a fixed number of apartments to rent, and they cannot adjust this number
quickly as market conditions change. Moreover, the number of people searching for housing in a city may not be highly
responsive to rents in the short run because people take time to adjust their housing arrangements. Therefore, the
short-run supply and demand for housing are both relatively inelastic. Panel (a) of Figure 3 shows the short-run
effects of rent control on the housing market. As with any binding price ceiling, rent control causes a shortage. But
because supply and demand are inelastic in the short run, the initial shortage caused by rent control is small. The
primary result in the short run is a reduction in rents.

\fig{e35}{0.75}

The long-run story is very different because the buyers and sellers of rental housing respond more to market
conditions as time passes. On the supply side, landlords respond to low rents by not building new apartments and by
failing to maintain existing ones. On the demand side, low rents encourage people to find their own apartments
(rather than living with roommates or their parents) and induce more people to move into the city. Therefore, both
supply and demand are more elastic in the long run. Panel (b) of Figure 3 illustrates the housing market in the long
run. When rent control depresses rents below the equilibrium level, the quantity of apartments supplied falls
substantially and the quantity of apartments demanded rises substantially. The result is a large shortage of housing. \v

In cities with rent control, landlords use various mechanisms to ration housing. Some landlords keep long waiting
lists. Others give preference to tenants without children. Still others discriminate on the basis of race. Sometimes
apartments are allocated to those willing to offer under-the-table payments to building superintendents. In essence,
these bribes bring the total price of an apartment closer to the equilibrium price. \v

To fully understand the effects of rent control, recall one of the Ten Principles of Economics: ``People respond to
incentives''. In free markets, landlords try to keep their buildings clean and safe because desirable apartments
command higher prices. But when rent control creates shortages and waiting lists, landlords lose their incentive to
respond to tenants' concerns. Why should a landlord spend money to maintain and improve the property when people are
waiting to move in as it is? In the end, tenants get lower rents, but they also get lower-quality housing. \v

Policymakers often react to the adverse effects of rent control by imposing additional regulations. For example,
various laws make racial discrimination in housing illegal and require landlords to provide minimally adequate living
conditions. These laws, however, are difficult and costly to enforce. By contrast, without rent control, such laws
are less necessary because the market for housing is regulated by the forces of competition. In a free market, the
price of housing adjusts to eliminate the shortages that give rise to undesirable landlord behavior.

\subsubsection*{The Minimum Wage (Price Floor)}

An important example of a price floor is the minimum wage.

\bd[Minimum Wage]
A \textbf{minimum wage} is the lowest wage per hour that a worker may be paid, as mandated by federal law.
\ed

Minimum-wage laws dictate the lowest price for labor that any employer may pay. To examine the effects of a minimum
wage, we must consider the market for labor. Panel (a) of Figure 5 shows the labor market, which, like all markets,
is subject to the forces of supply and demand. Workers supply labor, and firms demand labor. If the government
doesn't intervene, the wage adjusts to balance labor supply and labor demand. Panel (b) of Figure 5 shows the labor
market with a minimum wage. If the minimum wage is above the equilibrium level, as it is here, the quantity of labor
supplied exceeds the quantity demanded. The result is a surplus of labor, or unemployment. While the minimum wage
raises the incomes of those workers who have jobs, it lowers the incomes of would-be workers who now cannot find jobs.

\fig{e39}{0.75}

To fully understand the minimum wage, keep in mind that the economy contains not a single labor market but many labor
markets for different types of workers. The impact of the minimum wage depends on the skill and experience of the
worker. Highly skilled and experienced workers are not affected because their equilibrium wages are well above the
minimum. For these workers, the minimum wage is not binding. \v

The minimum wage has its greatest impact on the market for teenage labor. The equilibrium wages of teenagers are low
because teenagers are among the least skilled and least experienced members of the labor force. In addition,
teenagers are often willing to accept a lower wage in exchange for on-the-job training. (Some teenagers, including
many college students, are willing to work as interns for no pay at all. Because internships pay nothing,
minimum-wage laws often do not apply to them. If they did, these internship opportunities might not exist.) As a
result, the minimum wage is binding more often for teenagers than for other members of the labor force. \v

Many economists have studied how minimum-wage laws affect the teenage labor market. These researchers compare the
changes in the minimum wage over time with the changes in teenage employment. Although there is some debate about the
effects of minimum wages, the typical study finds that a 10 percent increase in the minimum wage depresses teenage
employment by 1 to 3 percent. \v

One drawback of most minimum-wage studies is that they focus on the effects over short periods of time. For example,
they might compare employment the year before and the year after a change in the minimum wage. The longer-term
effects on employment are harder to reliably estimate, but they are more relevant for evaluating the policy. Because
it takes time for firms to reorganize the workplace, the long-run decline in employment from a higher minimum wage is
likely larger than the estimated short-run decline. \v

In addition to altering the quantity of labor demanded, the minimum wage alters the quantity supplied. Because the
minimum wage raises the wage that teenagers can earn, it increases the number of teenagers who choose to look for
jobs. Studies have found that a higher minimum wage also influences which teenagers are employed. When the minimum
wage rises, some teenagers who are still attending high school choose to drop out and take jobs. With more people
vying for the available jobs, some of these new dropouts displace other teenagers who had already dropped out of
school, and these displaced teenagers now become unemployed. \v

The minimum wage is a frequent topic of debate. Advocates of the minimum wage view the policy as one way to raise the
income of the working poor. They correctly point out that workers who earn the minimum wage can afford only a meager
standard of living. In 2018, for instance, when the minimum wage was \$7.25 per hour, two adults working 40 hours a
week for every week of the year at minimum wage jobs had a joint annual income of only \$30,160. This amount was only
about 40 percent of the median family income in the United States. Many proponents of the minimum wage admit that it
has some adverse effects, including unemployment, but they believe that these effects are small and that, all things
considered, a higher minimum wage makes the poor better off. \v

Opponents of the minimum wage contend that it is not the best way to combat poverty. They note that a high minimum
wage causes unemployment, encourages teenagers to drop out of school, and prevents some unskilled workers from
getting on-the-job training. Moreover, opponents of the minimum wage point out that it is a poorly targeted policy.
Not all minimum-wage workers are heads of households trying to help their families escape poverty. In fact, less than
a third of minimum-wage earners are in families with incomes below the poverty line. Many are teenagers from
middle-class homes working at part-time jobs for extra spending money.

\subsection{Taxes}

All governments—from national governments around the world to local governments in small towns—use taxes to raise
revenue for public projects, such as roads, schools, and national defense.

\bd[Taxes]
\textbf{Taxes} are mandatory contributions levied on individuals or corporations by a government entity—whether local,
regional, or national.
\ed

Because taxes are such an important policy instrument and affect our lives in many ways, we return to the study of
taxes several times throughout these notes. In this section, we begin our study of how taxes affect the economy. \v

To set the stage for our analysis, imagine that a local government decides to hold an annual ice-cream celebration,
with a parade, fireworks, and speeches by town officials. To raise revenue to pay for the event, the town decides to
place a \$0.50 tax on each sale of ice-cream cones. When the plan is announced, our two lobbying groups swing into
action. The American Association of Ice-Cream Eaters claims that consumers of ice cream are having trouble making
ends meet, and it argues that sellers of ice cream should pay the tax. The National Organization of Ice-Cream Makers
claims that its members are struggling to survive in a competitive market, and it argues that buyers of ice cream
should pay the tax. The town mayor, hoping for a compromise, suggests that half the tax be paid by the buyers and
half be paid by the sellers. \v

To analyze these proposals, we need to address a simple but subtle question: When the government levies a tax on a
good, who actually bears the burden of the tax? The people buying the good? The people selling the good? Or if buyers
and sellers share the tax burden, what determines how the burden is divided? Can the government legislate the
division of the burden, as the mayor is suggesting, or is the division determined by market forces? The term tax
incidence refers to how the burden of a tax is distributed among the various people who make up the economy. As we
will see, some surprising lessons about tax incidence can be learned by applying the tools of supply and demand.

\bd[Tax Incidence]
\textbf{Tax incidence} is an economic term for understanding the division of a tax burden between stakeholders, such as
buyers and sellers or producers and consumers.
\ed

We begin by considering a tax levied on sellers of a good. Suppose the local government passes a law requiring
sellers of ice-cream cones to send \$0.50 to the government for every cone they sell. How does this law affect the
buyers and sellers of ice cream? To answer this question, we can follow the three steps for analyzing supply and
demand: (1) We decide whether the law affects the supply curve or the demand curve. (2) We decide which way the curve
shifts. (3) We examine how the shift affects the equilibrium price and quantity. \v

Step one. The immediate impact of the tax is on the sellers of ice cream. Because the tax is not levied on buyers,
the quantity of ice cream demanded at any given price remains the same; thus, the demand curve does not change. By
contrast, the tax on sellers makes the ice-cream business less profitable at any given price, so it shifts the supply
curve. \v

Step two. Because the tax on sellers raises the cost of producing and selling ice cream, it reduces the quantity
supplied at every price. The supply curve shifts to the left (or, equivalently, upward). In addition to determining
the direction in which the supply curve moves, we can also be precise about the size of the shift. For any market
price of ice cream, the effective price to sellers—the amount they get to keep after paying the tax—is \$0.50 lower.
For example, if the market price of a cone happened to be \$2.00, the effective price received by sellers would be
\$1.50. Whatever the market price, sellers will supply a quantity of ice cream as if the price were \$0.50 lower than
it is. Put differently, to induce sellers to supply any given quantity, the market price must now be \$0.50 higher to
compensate for the effect of the tax. Thus, as shown in Figure 6, the supply curve shifts upward from $S_1$ to $S_2$
by the exact size of the tax (\$0.50). \v

Step three. Having determined how the supply curve shifts, we can now compare the initial and the new equilibria.
Figure 6 shows that the equilibrium price of ice cream rises from \$3.00 to \$3.30, and the equilibrium quantity
falls from 100 to 90 cones. Because sellers sell less and buyers buy less in the new equilibrium, the tax reduces the
size of the ice-cream market.

\fig{e36}{0.8}

Implications. We can now return to the question of tax incidence: Who pays the tax? Although sellers send the entire
tax to the government, buyers and sellers share the burden. Because the market price rises from \$3.00 to \$3.30 when
the tax is introduced, buyers pay \$0.30 more for each ice-cream cone than they did without the tax. Thus, the tax
makes buyers worse off. Sellers get a higher price (\$3.30) from buyers than they previously did, but what they get
to keep after paying the tax is only \$2.80 ($\$3.30 - \$0.50 = \$2.80$), less than the \$3.00 they pocketed before
the tax. Thus, the tax also makes sellers worse off.

To sum up, this analysis yields two lessons:
\bit
\item Taxes discourage market activity. When a good is taxed, the quantity of the good sold is smaller in the new
equilibrium.
\item Buyers and sellers share the burden of taxes. In the new equilibrium,buyers pay more for the good, and sellers
receive less.
\eit

Now consider a tax levied on buyers of a good. Suppose that our local government passes a law requiring buyers of
ice-cream cones to send \$0.50 to the government for each ice-cream cone they buy. What are the effects of this law?
Again, we apply our three steps. \v

Step one. The immediate impact of the tax is on the demand for ice cream. The supply curve is not affected because,
for any given price of ice cream, sellers have the same incentive to provide ice cream to the market. By contrast,
buyers now have to pay a tax to the government (as well as the price to the sellers) whenever they buy ice cream.
Thus, the tax shifts the demand curve for ice cream. \v

Step two. Next, we determine the direction of the shift. Because the tax on buyers makes buying ice cream less
attractive, buyers demand a smaller quantity of ice cream at every price. As a result, the demand curve shifts to the
left (or, equivalently, downward), as shown in Figure 7. Once again, we can be precise about the size of the shift.
Because of the \$0.50 tax levied on buyers, the effective price that buyers pay is now \$0.50 higher than the market
price (whatever the market price happens to be). For example, if the market price of a cone happened to be \$2.00,
the effective price to buyers would be \$2.50. Because buyers look at their total cost including the tax, they demand
a quantity of ice cream as if the market price were \$0.50 higher than it actually is. In other words, to induce
buyers to demand any given quantity, the market price must now be \$0.50 lower to make up for the effect of the tax.
Thus, the tax shifts the demand curve downward from $D_1$ to $D_2$ by the exact size of the tax (\$0.50). \v

\fig{e37}{0.88}

Step three. Having determined how the demand curve shifts, we can now see the effect of the tax by comparing the
initial equilibrium and the new equilibrium. You can see in Figure 7 that the equilibrium price of ice cream falls
from \$3.00 to \$2.80, and the equilibrium quantity falls from 100 to 90 cones. Once again, the tax on ice cream
reduces the size of the ice-cream market. And once again, buyers and sellers share the burden of the tax. Sellers get
a lower price for their product; buyers pay a lower market price to sellers than they previously did, but the
effective price (including the tax buyers have to pay) rises from \$3.00 to \$3.30. \v

Implications. If you compare Figures 6 and 7, you will notice a surprising conclusion. Taxes levied on sellers and
taxes levied on buyers are equivalent. In both cases, the tax places a wedge between the price that buyers pay and
the price that sellers receive. The wedge between the buyers' price and the sellers' price is the same whether the
tax is levied on buyers or sellers. In either case, the wedge shifts the relative position of the supply and demand
curves. In the new equilibrium, buyers and sellers share the burden of the tax. The only difference between a tax
levied on sellers and a tax levied on buyers is who sends the money to the government. The equivalence of these two
taxes is easy to understand if we imagine that the government collects the \$0.50 ice-cream tax in a bowl on the
counter of each ice-cream store. When the government levies the tax on sellers, the seller is required to place \$0.50
in the bowl after the sale of each cone. When the government levies the tax on buyers, the buyer is required to
place \$0.50 in the bowl every time a cone is bought. Whether the \$0.50 goes directly from the buyer's pocket into
the bowl, or indirectly from the buyer's pocket into the seller's hand and then into the bowl, does not matter. Once
the market reaches its new equilibrium, buyers and sellers share the burden, regardless of how the tax is levied. \v

When a good is taxed, buyers and sellers share the burden of the tax. But the tax only rarely will be shared equally.
To see that, consider the impact of taxation in the two markets in Figure 9.

\fig{e40}{0.84}

In both cases, the figure shows the initial demand curve, the initial supply curve, and a tax that drives a wedge
between the amount paid by buyers and the amount received by sellers. (Not drawn in either panel of the figure is the
new supply or demand curve. Which curve shifts depends on whether the tax is levied on buyers or sellers. As we have
seen, this is irrelevant for determining the incidence of the tax.) The difference between the two panels is the
relative elasticity of supply and demand. \v

Panel (a) of Figure 9 shows a tax in a market with very elastic supply and relatively inelastic demand. That is,
sellers are very responsive to changes in the price of the good (so the supply curve is relatively flat), whereas
buyers are not very responsive (so the demand curve is relatively steep). When a tax is imposed on a market with
these elasticities, the price received by sellers does not fall by much, so sellers bear only a small burden. By
contrast, the price paid by buyers rises substantially, indicating that buyers bear most of the burden of the tax. \v

Panel (b) of Figure 9 shows a tax in a market with relatively inelastic supply and very elastic demand. In this case,
sellers are not very responsive to changes in the price (so the supply curve is steeper), whereas buyers are very
responsive (so the demand curve is flatter). The figure shows that when a tax is imposed, the price paid by buyers
does not rise by much, but the price received by sellers falls substantially. Thus, sellers bear most of the burden
of the tax. \v

The two panels of Figure 9 show a general lesson about how the burden of a tax is divided: A tax burden falls more
heavily on the side of the market that is less elastic. Why is this true? In essence, the elasticity measures the
willingness of buyers or sellers to leave the market when conditions become unfavorable. A small elasticity of demand
means that buyers do not have good alternatives to consuming this particular good. A small elasticity of supply means
that sellers do not have good alternatives to producing this particular good. When the good is taxed, the side of the
market with fewer good alternatives is less willing to leave the market and, therefore, bears more of the burden of
the tax.

\section{Consumers, Producers, And The Efficiency Of Markets}

In previous sections, we saw how, in market economies, the forces of supply and demand determine the prices of goods
and services and the quantities sold. So far, however, we have described the way markets allocate scarce resources
without addressing the question of whether these market allocations are desirable. In other words, our analysis has
been positive (what is) rather than normative (what should be). We know that the price of a good adjusts to ensure
that the quantity of the good supplied equals the quantity of the good demanded. But at this equilibrium, is the
quantity of the good produced and consumed too small, too large, or just right? \v

In this chapter, we take up the topic of welfare economics, the study of how the allocation of resources affects
economic well-being.

\bd[Welfare Economics]
\textbf{Welfare economics} is the study of how the allocation of resources and goods affects social well-being.
\ed

One of the Ten Principles of Economics is that markets are usually a good way to organize economic activity. The
study of welfare economics explains this principle more fully.

\subsection{Consumer Surplus}

A buyer's maximum price for a good is called her willingness to pay, and it measures how much that buyer values the
good.

\bd[Willingness To Pay]
\textbf{Willingness to pay} is the maximum price at or below which a consumer will definitely buy one unit of product.
\ed

Each buyer would be eager to buy a good at a price less than her willingness to pay, and each would refuse to buy the
good at a price greater than her willingness to pay. At a price equal to her willingness to pay, the buyer would be
indifferent about buying the good: if the price is exactly the same as the value she places on the good, she would be
equally happy buying it or keeping her money. \v

Consumer surplus is the amount a buyer is willing to pay for a good minus the amount the buyer actually pays for it
and measures the benefit buyers receive from participating in a market.

\bd[Consumer Surplus]
A \textbf{consumer surplus} happens when the price that consumers pay for a product or service is less than the price
they're willing to pay. It's a measure of the additional benefit that consumers receive because they're paying less
for something than what they were willing to pay.
\ed

Consumer surplus is closely related to the demand curve for a product. Because the demand curve reflects buyers'
willingness to pay, we can also use it to measure consumer surplus. Namely, the area below the demand curve and above
the price measures the consumer surplus in a market. This is true because the height of the demand curve represents
the value buyers place on the good, as measured by their willingness to pay for it. The difference between this
willingness to pay and the market price is each buyer's consumer surplus. Thus, the total area below the demand curve
and above the price is the sum of the consumer surplus of all buyers in the market for a good or service. Let's see
all of these in an example. \v

Imagine that you own a mint-condition recording of Elvis Presley's first album. Because you are not an Elvis Presley
fan, you decide to sell it. One way to do so is to hold an auction. Four Elvis fans show up for your auction: Taylor,
Carrie, Rihanna, and Gaga. They would all like to own the album, but each of them has a limit on the amount she is
willing to pay for it. \$100, \$80, \$70 and \$50 are the maximum prices that Taylor, Carrie, Rihanna, and Gaga would
pay respectively. To sell your album, you begin the bidding process at a low price, say, \$10. Because all four
buyers are willing to pay much more, the price rises quickly. The bidding stops when Taylor bids \$80 (or slightly
more). At this point, Carrie, Rihanna, and Gaga have all dropped out of the bidding because they are unwilling to
offer any more than \$80. Taylor pays you \$80 and gets the album. Note that the album has gone to the buyer who
values it most. Taylor receives a \$20 benefit from participating in the auction because she pays only \$80 for a
good she values at \$100. Carrie, Rihanna, and Gaga get no consumer surplus from participating in the auction because
they left without the album and without paying anything. \v

Now consider a somewhat different example. Suppose that you had two identical Elvis Presley albums to sell. Again,
you auction them off to the four possible buyers. To keep things simple, we assume that both albums are to be sold
for the same price and that no one is interested in buying more than one album. Therefore, the price rises until two
buyers are left. In this case, the bidding stops when Taylor and Carrie bid \$70 (or slightly higher). At this price,
Taylor and Carrie are each happy to buy an album, and Rihanna and Gaga are not willing to bid any higher. Taylor and
Carrie each receive consumer surplus equal to her willingness to pay minus the price. Taylor's consumer surplus is
\$30, and Carrie's is \$10. Taylor's consumer surplus is higher now than in the previous example because she gets the
same album but pays less for it. The total consumer surplus in the market is \$40. \v

Let's consider the demand curve for this rare Elvis Presley album. We begin by using the willingness to pay of the
four possible buyers to find the market demand schedule for the album. The table in Figure 1 shows the demand
schedule that corresponds to Table 1. If the price is above \$100, the quantity demanded in the market is 0 because
no buyer is willing to pay that much. If the price is between \$80 and \$100, the quantity demanded is 1 because only
Taylor is willing to pay such a high price. If the price is between \$70 and \$80, the quantity demanded is 2 because
both Taylor and Carrie are willing to pay the price. We can continue this analysis for other prices as well. In this
way, the demand schedule is derived from the willingness to pay of the four possible buyers.

\fig{e60}{0.8}

The graph in Figure 1 shows the demand curve that corresponds to this demand schedule. Note the relationship between
the height of the demand curve and the buyers' willingness to pay. At any quantity, the price given by the demand
curve shows the willingness to pay of the marginal buyer, the buyer who would leave the market first if the price
were any higher. At a quantity of 4 albums, for instance, the demand curve has a height of \$50, the price that Gaga
(the marginal buyer) is willing to pay for an album. At a quantity of 3 albums, the demand curve has a height of
\$70, the price that Rihanna (who is now the marginal buyer) is willing to pay. \v

Figure 2 uses the demand curve to compute consumer surplus in our two examples. In panel (a), the price is \$80 (or
slightly above) and the quantity demanded is 1. Note that the area above the price and below the demand curve equals
\$20. This amount is exactly the consumer surplus we computed earlier when only 1 album is sold. Panel (b) of Figure
2 shows consumer surplus when the price is \$70 (or slightly above). In this case, the area above the price and below
the demand curve equals the total area of the two rectangles: Taylor's consumer surplus at this price is \$30 and
Carrie's is \$10. This area equals a total of \$40. Once again, this amount is the consumer surplus we computed earlier.

\fig{e61}{0.8}

Because buyers always want to pay less for the goods they buy, a lower price makes buyers of a good better off. But
how much does buyers' well-being rise in response to a lower price? We can use the concept of consumer surplus to
answer this question precisely. \v

Figure 3 shows a typical demand curve. You may notice that this curve gradually slopes downward instead of taking
discrete steps as in the previous two figures in the example. In a market with many buyers, the resulting steps from
each buyer dropping out are so small that they form a smooth demand curve. Although this curve has a different shape,
the ideas we have just developed still apply: Consumer surplus is the area above the price and below the demand curve.
In panel (a), consumer surplus at a price of P1 is the area of triangle ABC\@.

\fig{e52}{0.8}

Now suppose that the price falls from $P_1$ to $P_2$, as shown in panel (b). The consumer surplus now equals area ADF\@.
The increase in consumer surplus attributable to the lower price is the area BCFD. This increase in consumer
surplus is composed of two parts. First, those buyers who were already buying $Q_1$ of the good at the higher price
$P_1$ are better off because now they pay less. The increase in consumer surplus of existing buyers is the reduction
in the amount they pay; it equals the area of the rectangle BCED. Second, some new buyers enter the market because
they are willing to buy the good at the lower price. As a result, the quantity demanded in the market increases from
$Q_1$ to $Q_2$. The consumer surplus these newcomers receive is the area of the triangle CEF. \v

Consumer surplus is a good measure of economic well-being if policymakers want to satisfy the preferences of buyers.
In some circumstances, policymakers might choose to disregard consumer surplus because they do not respect the
preferences that drive buyer behavior.

\be For example, drug addicts are willing to pay a high price for heroin. Yet we would not say that addicts get a
large benefit from being able to buy heroin at a low price (even though addicts might say they do). From the
standpoint of society, willingness to pay in this instance is not a good measure of the buyers' benefit, and consumer
surplus is not a good measure of economic well-being, because addicts are not looking after their own best interests.
\ee

In most markets, however, consumer surplus does reflect economic well-being. Economists normally assume that buyers
are rational when they make decisions Rational people do the best they can to achieve their objectives, given their
opportunities. Economists also normally assume that people's preferences should be respected. In this case, consumers
are the best judges of how much benefit they receive from the goods they buy.

\subsection{Producer Surplus}

A seller's maximum price for a good is called her willingness to sell, and it measures how much that seller values
the good.

\bd[Willingness To Sell]
\textbf{Willingness to pay} is the maximum price at or above which a producer will definitely sell one unit of product.
\ed

Each seller would be eager to sell a good at a price more than her willingness to sell, and each would refuse to sell
the good at a price lower than her willingness to sell. At a price equal to her willingness to sell, the seller would
be indifferent about selling the good: if the price is exactly the same as the value she places on the good, she
would be equally happy selling it or keeping her money. \v

Producer surplus is the amount a seller is willing to sell for a good plus the amount the seller actually sells,
minus the cost of producing the good.

\bd[Producer Surplus]
\textbf{Producer surplus} is the difference between how much a producer would be willing to accept for given quantity
of a good versus how much they can receive by selling the good at the market price minus the cost of the good.
\ed

Producer surplus is closely related to the supply curve for a product. Because the supply curve reflects sellers'
willingness to sell, we can also use it to measure producer surplus. Namely, the area above the supply curve and
below the price measures the producer surplus in a market. This is true because the height of the supply curve
represents the value sellers place on the good, as measured by their willingness to sell for it. The difference
between this willingness to sell and the market price is each seller's producer surplus. Thus, the total area below
the supply curve and above the price is the sum of the producer surplus of all sellers in the market for a good or
service. Once again, let's see all of these in an example. \v

Imagine now that you are a homeowner and want to get your house painted. You turn to four sellers of painting
services: Vincent, Claude, Pablo, and Andy. Each painter is willing to do the work for you if the price is right. You
decide to take bids from the four painters and auction off the job to the painter who will do the work for the lowest
price. Each painter is willing to take the job if the price he would receive exceeds his cost of doing the work. Here
the term cost should be interpreted as the painter's opportunity cost: it includes the painter's out-of-pocket
expenses (for paint, brushes, and so on) as well as the value that the painter places on his time. \$900, \$800,
\$600 and \$500 are the minimum prices that Vincent, Claude, Pablo, and Andy would receive respectively. When you
take bids from the painters, the price might start high, but it quickly falls as the painters compete for the job.
Once Andy has bid \$600 (or slightly less), he is the sole remaining bidder. Andy is happy to do the job for this
price because his cost is only \$500. Vincent, Claude, and Pablo are unwilling to do the job for less than \$600.
Note that the job goes to the painter who can do the work at the lowest cost. What benefit does Andy receive from
getting the job? Because he is willing to do the work for \$500 but gets \$600 for doing it, we say that he receives
producer surplus of \$100. \v

Now consider a somewhat different example. Suppose that you have two houses that need painting. Again, you auction
off the jobs to the four painters. To keep things simple, let's assume that no painter is able to paint both houses
and that you will pay the same amount to paint each house. Therefore, the price falls until two painters are left. In
this case, the bidding stops when Andy and Pablo each offer to do the job for a price of \$800 (or slightly less).
Andy and Pablo are willing to do the work at this price, while Vincent and Claude are not willing to bid a lower
price. At a price of \$800, Andy receives producer surplus of \$300 and Pablo receives producer surplus of \$200. The
total producer surplus in the market is \$500. \v

Let's consider the supply curve for this painting house example. We begin by using the costs of the four painters to
find the supply schedule for painting services. The table in Figure 4 shows the supply schedule that corresponds to
the costs in Table 2. If the price is below \$500, none of the four painters is willing to do the job, so the
quantity supplied is zero. If the price is between \$500 and \$600, only Andy is willing to do the job, so the
quantity supplied is 1. If the price is between \$600 and \$800, Andy and Pablo are willing to do the job, so the
quantity supplied is 2, and so on. Thus, the supply schedule is derived from the costs of the four painters.

\fig{e62}{0.7}

The graph in Figure 4 shows the supply curve that corresponds to this supply schedule. Note that the height of the
supply curve is related to the sellers' costs. At any quantity, the price given by the supply curve shows the cost of
the marginal seller, the seller who would leave the market first if the price were any lower. At a quantity of 4
houses, for instance, the supply curve has a height of \$900, the cost that Vincent (the marginal seller) incurs to
provide his painting services. At a quantity of 3 houses, the supply curve has a height of \$800, the cost that
Claude (who is now the marginal seller) incurs. \v

Figure 5 uses the supply curve to compute producer surplus in our two examples. In panel (a), we assume that the
price is \$600 (or slightly less). In this case, the quantity supplied is 1. Note that the area below the price and
above the supply curve equals \$100. This amount is exactly the producer surplus we computed earlier for Andy. Panel
(b) of Figure 5 shows producer surplus at a price of \$800 (or slightly less). In this case, the area below the price
and above the supply curve equals the total area of the two rectangles. This area equals \$500, the producer surplus
we computed earlier for Pablo and Andy when two houses needed painting.

\fig{e63}{0.7}

Because sellers always want to sell more for the goods they sell, a higher price makes sellers of a good better off.
But how much does sellers' well-being rise in response to a higher price? We can use the concept of producer surplus
to answer this question precisely. \v

Figure 6 shows a typical upward-sloping supply curve that would arise in a market with many sellers. You may notice
that this supply gradually slopes upward instead of taking discrete steps as in the previous two figures in the
example. In a market with many sellers, the resulting steps from each seller dropping out are so small that they form
a smooth supply curve. Although this supply curve differs we measure producer surplus in the same way: producer
surplus is the area below the price and above the supply curve. In panel (a) the price is $P_1$ and producer surplus
is the area of triangle ABC\@.

\fig{e64}{0.7}

Panel (b) shows what happens when the price rises from $P_1$ to $P_2$. Producer surplus now equals area ADF. This
increase in producer surplus has two parts. First, those sellers who were already selling Q1 of the good at the lower
price P1 are better off because they now get more for what they sell. The increase in producer surplus for existing
sellers equals the area of the rectangle BCED. Second, some new sellers enter the market because they are willing to
produce the good at the higher price, resulting in an increase in the quantity supplied from $Q_1$ to $Q_2$. The
producer surplus of these newcomers is the area of the triangle CEF. \v

As this analysis shows, we use producer surplus to measure the well-being of sellers in much the same way as we use
consumer surplus to measure the well-being of buyers. Because these two measures of economic welfare are so similar,
it is natural to consider them together. Indeed, that is exactly what we do in the next section.

\subsection{Market Efficiency}

Consumer surplus and producer surplus are the basic tools that economists use to study the welfare of buyers and
sellers in a market. These tools can help us address a fundamental economic question: Is the allocation of resources
determined by free markets desirable? \v

To answer this question, we must first decide how to measure the economic well-being of a society. Since consumer
surplus is the benefit that buyers receive from participating in a market, and producer surplus is the benefit that
sellers receive, one natural measure of society's economic well-being is the sum of consumer and producer surplus,
which we call total surplus.

\bd[Total Surplus]
\textbf{Total surplus} is the sum of consumer and producer surplus.
\ed

By ``manipulating'' a bit the total surplus we can show that the total surplus in a market is the total value to
buyers of the goods, as measured by their willingness to pay, minus the total cost to sellers of providing those goods:

\begin{align}
\text{Total Surplus } & \text{= (Consumer Surplus) + (Producer Surplus)} \nonumber \\
& \text{= (Willignes To Pay - Amount Paid) + (Amount Received - Cost)} \nonumber \\
& \text{= Willignes To Pay - Amount Paid + Amount Received - Cost} \nonumber \\
& \text{= Willignes To Pay - Cost} \nonumber
\end{align}

If an allocation of resources maximizes total surplus, we say that the allocation exhibits efficiency.

\bd[Efficiency]
\textbf{Efficiency} is the property of a resource allocation of maximizing the total surplus received by all members of
society.
\ed

If an allocation is not efficient, then some of the potential gains from trade among buyers and sellers are not being
realized.

\be
For example, an allocation is inefficient if a good is not being produced by the sellers with the lowest costs. In this
case, moving production from a high-cost producer to a lower-cost producer will reduce the total cost to sellers and
raise total surplus. \v

Similarly, an allocation is inefficient if a good is not being consumed by the buyers who value it most. In this case,
moving consumption of the good from a buyer with a low valuation to a buyer with a higher valuation will raise total
surplus.
\ee

In addition to efficiency, the social planner might also care about equality —that is, whether the various buyers and
sellers in the market have similar levels of economic well-being.

\bd[Equality]
\textbf{Equality} is the property of distributing economic prosperity uniformly among the members of society.
\ed

In essence, the gains from trade in a market are like a pie to be shared among the market participants. The question
of efficiency concerns whether the pie is as big as possible. The question of equality concerns how the pie is sliced
and distributed among members of society. In what follows, we focus on efficiency as the social planner's goal. Keep
in mind, however, that real policymakers often care about equality as well. \v

Figure 7 shows consumer and producer surplus when a market reaches the equilibrium of supply and demand. Recall that
consumer surplus equals the area above the price and under the demand curve and producer surplus equals the area
below the price and above the supply curve. Thus, the total area between the supply and demand curves up to the point
of equilibrium represents the total surplus in this market.

\fig{e59}{0.7}

Is this equilibrium allocation of resources efficient? That is, does it maximize total surplus? To answer this
question, recall that when a market is in equilibrium, the price determines which buyers and sellers participate in
the market. Those buyers who value the good more than the price (represented by the segment AE on the demand curve)
choose to buy the good; buyers who value it less than the price (represented by the segment EB) do not. Similarly,
those sellers whose costs are less than the price (represented by the segment CE on the supply curve) choose to
produce and sell the good; sellers whose costs are greater than the price (represented by the segment ED) do not. \v

These observations lead to two insights about market outcomes.
\bit
\item Free markets allocate the supply of goods to the buyers who value them most, as measured by their willingness to
pay.
\item Free markets allocate the demand for goods to the sellers who can produce them at the lowest cost.
\eit

Thus, given the quantity produced and sold in a market equilibrium, the social planner cannot increase economic
well-being by changing the allocation of consumption among buyers or the allocation of production among sellers. But
can the social planner raise total economic well-being by increasing or decreasing the quantity of the good? The
answer is no, as stated in this third insight about market outcomes.
\bit
\item Free markets produce the quantity of goods that maximizes the sum of consumer and producer surplus.
\eit

Figure 8 illustrates why this is true. To interpret this figure, keep in mind that the demand curve reflects the
value to buyers and the supply curve reflects the cost to sellers. At any quantity below the equilibrium level, such
as $Q_1$,the value to the marginal buyer exceeds the cost to the marginal seller. As a result, increasing the
quantity produced and consumed raises total surplus. This continues to be true until the quantity reaches the
equilibrium level. Similarly, at any quantity beyond the equilibrium level, such as $Q_2$, the value to the marginal
buyer is less than the cost to the marginal seller. In this case, decreasing the quantity raises total surplus, and
this continues to be true until quantity falls to the equilibrium level. To maximize total surplus, the social
planner would choose the quantity at which the supply and demand curves intersect.

\fig{e65}{0.7}

Together, these three insights tell us that the market outcome makes the sum of consumer and producer surplus as
large as it can be. In other words, the equilibrium outcome is an efficient allocation of resources. The policy of
leaving well enough alone goes by the French expression laissez-faire, which literally translates to ``leave to do''
but is more broadly interpreted as ``let people do as they will''.

\bd[Laissez-faire]
\textbf{Laissez-faire} is an economic theory from the 18th century that opposed any government intervention in
business affairs. The driving principle behind laissez-faire, a French term that translates to ``leave alone''
(literally, ``let you do''), is that the less the government is involved in the economy, the better off business will
be, and by extension, society as a whole. Laissez-faire economics is a key part of free-market capitalism.
\ed

Society is lucky that a social planner doesn't need to intervene. Suppose our social planner tried to choose an
efficient allocation of resources on her own, instead of relying on market forces. To do so, she would need to know
the value of a particular good to every potential consumer in the market and the cost for every potential producer.
And she would need this information not only for this market but for every one of the many thousands of markets in
the economy. This task is practically impossible, which explains why centrally planned economies never work well. The
planner's job becomes easy, however, once she takes on a partner: Adam Smith's invisible hand of the marketplace. The
invisible hand takes all the information about buyers and sellers into account and guides everyone in the market to
the best outcome as judged by the standard of economic efficiency. It is a remarkable feat. That is why economists so
often advocate free markets as the best way to organize economic activity.

\subsection{Application: Efficiency Of Markets \& Externalities}

Let's suppose that steel factories emit pollution: for each unit of steel produced, a certain amount of smoke enters
the atmosphere. Because this smoke creates a health risk for those who breathe the air, it is a negative externality.
Because of this externality, the cost of producing steel to society as a whole exceeds the cost incurred by the steel
producers. For each unit of steel produced, the social cost equals the private costs of the steel producers plus the
costs to those bystanders harmed by the pollution. \v

Figure 2 shows the social cost of producing steel. The social-cost curve is above the supply curve because it takes
into account the external costs imposed on society by steel production. The difference between these two curves
reflects the cost of the pollution emitted.

\fig{e66}{0.8}

A hypothetical planner understands, that the cost of producing steel includes the external costs of the pollution.
The planner would choose the level of steel production at which the demand curve crosses the social-cost curve. This
intersection determines the optimal amount of steel from the standpoint of society as a whole. Below this level of
production, the value of the steel to consumers (as measured by the height of the demand curve) exceeds the social
cost of producing it (as measured by the height of the social-cost curve). Above this level of production, the social
cost of producing additional steel exceeds the value to consumers. \v

Note that the equilibrium quantity of steel, $Q_\text{MARKET}$, is larger than the socially optimal quantity,
$Q_\text{OPTIMUM}$. This inefficiency occurs because the market equilibrium reflects only the private costs of
production. In the market equilibrium, the marginal consumer values steel at less than the social cost of producing
it. That is, at $Q_\text{MARKET}$, the demand curve lies below the social-cost curve. Thus, reducing steel
production and consumption below the market equilibrium level raises total economic well-being. \v

How can the social planner achieve the optimal outcome? One way would be to tax steel producers for each ton of steel
sold. The tax would shift the supply curve for steel upward by the size of the tax. If the tax accurately reflected
the external cost of pollutants released into the atmosphere, the new supply curve would coincide with the
social-cost curve. In the new market equilibrium, steel producers would produce the socially optimal quantity of
steel. \v

The use of such a tax is called internalizing the externality because it gives buyers and sellers in the market an
incentive to take into account the external effects of their actions.

\bd[Internalization Of Externalities]
\textbf{Internalization of externalities} refers to all measures (public or private) that guarantee that unpaid benefits
or costs are taken into account in the composition of goods and services prices.
\ed

Steel producers would, in essence, take the costs of pollution into account when deciding how much steel to supply
because the tax would make them pay for these external costs. And, because the market price would reflect the tax on
producers, consumers of steel would have an incentive to buy a smaller quantity. The policy is based on one of the
Ten Principles of Economics: People respond to incentives. \v

Although some activities impose costs on third parties, others yield benefits. Consider education, for example.
Education yields positive externalities such as that more educated population leads to more informed voters, which
means better government for everyone, tends to result in lower crime rates and encourages the development and
dissemination of technological advances, leading to higher productivity and wages for everyone. Given these positive
externalities, people may prefer to have neighbors who are well educated. The analysis of positive externalities is
similar to the analysis of negative externalities, and for this reason we will skip it for now.

\section{Consumer Theory}

Thus, far, we have explained consumers' decisions with the demand curve. As we have seen, the demand curve for a good
reflects consumers' willingness to pay for that good. When the price of the good rises, consumers are willing to pay
for fewer units, so the quantity demanded falls. We now look more deeply at the decisions that lie behind the demand
curve. The theory of consumer choice presented in this section describes how consumers make decisions about what to
buy and provides a more complete understanding of demand.

\bd[Consumer Theory]
\textbf{Consumer theory} is the study of how people make choices to spend their money based on their individual
preferences, their income and the prices of goods and services.
\ed

One of the Ten Principles of Economics is that people face trade-offs. When a consumer buys more of one good, she can
afford less of other goods. When she spends more time enjoying leisure and less time working, she earns less and
therefore consumes less. When she spends more of her income in the present and saves less of it, she reduces the
amount she will be able to consume in the future. The theory of consumer choice examines how consumers facing these
trade-offs make decisions and how they respond to changes in their environment.

\subsection{The Budget Constraint}

Most people would like to increase the quantity or quality of the goods they consume. People consume less than they
desire because their spending is constrained, or limited, by their income. We begin our study of consumer choice by
examining this link between income and spending, i.e.\ the budget.

\bd[Budget]
A \textbf{budget} is an estimation of revenue and expenses over a specified future period of time and is usually
compiled and re-evaluated on a periodic basis. Budgets can be made for a person, a group of people, a business, a
government, or just about anything else that makes and spends money.
\ed

To keep things simple, we examine the decision facing a consumer who buys only two goods: pizza and Pepsi. Although
real people buy hundreds of different kinds of goods, assuming there are only two goods simplifies the problem
without altering the basic insights about consumer choice. \v

We first consider how the consumer's income constrains the amount she spends on pizza and Pepsi. Suppose the consumer
has an income of \$1,000 per month and spends her entire income on pizza and Pepsi. The price of a pizza is \$10, and
the price of a liter of Pepsi is \$2. \v

The table in Figure 1 shows some of the many combinations of pizza and Pepsi that the consumer can buy. The first row
in the table shows that if the consumer spends all her income on pizza, she can eat 100 pizzas during the month, but
she would not be able to buy any Pepsi at all. The second row shows another possible consumption bundle: 90 pizzas
and 50 liters of Pepsi. And so on. Each consumption bundle in the table costs exactly \$1,000. \v

The graph in Figure 1 illustrates the consumption bundles that the consumer can choose. The vertical axis measures
the number of liters of Pepsi, and the horizontal axis measures the number of pizzas. Three points are marked on this
figure. At point A, the consumer buys no Pepsi and consumes 100 pizzas. At point B, the consumer buys no pizza and
consumes 500 liters of Pepsi. At point C, the consumer buys 50 pizzas and 250 liters of Pepsi. Point C, which is
exactly at the middle of the line from A to B, is the point at which the consumer spends an equal amount (\$500) on
pizza and Pepsi. These are only three of the many combinations of pizza and Pepsi that the consumer can choose. All
the points on the line from A to B are possible. This line, called the ``budget constraint'', shows the consumption
bundles that a consumer can afford. In this case, it shows the trade-off between pizza and Pepsi that the consumer
faces.

\fig{e50}{0.75}

\bd[Budget constraint]
\textbf{Budget constraint} is the limit on the consumption bundles that a consumer can afford based on his budget (i.e
their income and expenses)
\ed

The slope of the budget constraint measures the rate at which the consumer can trade one good for the other. Because
the budget constraint slopes downward, the slope is a negative number. But for our purposes we can ignore the minus
sign. Notice that the slope of the budget constraint equals the relative price of the two goods, i.e.\ the price of one
good compared to the price of the other. A pizza costs five times as much as a liter of Pepsi, so the opportunity
cost of a pizza is 5 liters of Pepsi. The budget constraint's slope of 5 reflects the trade-off the market is
offering the consumer: 1 pizza for 5 liters of Pepsi.

\bd[Relative Price]
A \textbf{relative price} is the price of a commodity such as a good or service in terms of another; i.e.\ the ratio
of two prices. The relative price equals the slope of the budget constraint.
\ed

The budget constraint shows the opportunities available to the consumer. It is drawn given the consumer's income and
given the prices of the two goods. If the consumer's income or the prices change, the budget constraint shifts. Let's
consider three examples of how such a shift might occur. \v

Suppose first that the consumer's income increases from \$1,000 to \$2,000 while prices remain the same. With higher
income, the consumer can afford more of both goods. The increase in income, therefore, shifts the budget constraint
outward, as in panel (a) of Figure 2. Because the relative price of the two goods has not changed, the slope of the
new budget constraint is the same as the slope of the initial budget constraint. That is, an increase in income leads
to a parallel shift in the budget constraint. \v

Now suppose that the price of Pepsi falls from \$2 to \$1 while the consumer's income remains at \$1,000 and the
price of pizza remains at \$10. If the consumer spends her entire income on pizza, the price of Pepsi is irrelevant.
In this case, she can still buy only 100 pizzas, so the point on the horizontal axis representing 100 pizzas and 0
liters of Pepsi stays the same. But as long as the consumer was buying some Pepsi, the lower price of Pepsi expands
her set of opportunities. The budget constraint shifts outward, as shown in panel (b) of Figure 2. The lower price
allows her to buy the same amount of pizza as before and more Pepsi, the same amount of Pepsi as before and more
pizza, or more of both goods. Note that because the slope reflects the relative price of pizza and Pepsi, it changes
when the price of Pepsi falls. With a lower price of Pepsi, the consumer can now trade a pizza for 10 liters of Pepsi
rather than 5. As a result, the new budget constraint is steeper. The expansion in the consumer's opportunities is
represented by a rotational shift rather than a parallel shift. \v

For our third example, suppose that the price of pizza falls from \$10 to \$5 while the consumer's income remains at
\$1,000 and the price of Pepsi remains at \$2. Once again, the lower price expands the consumer's set of buying
opportunities and leads to a rotational outward shift in the budget constraint, as shown in panel (c) of Figure 2.
Now, with a lower price of pizza, the consumer can now trade a pizza for 2.5 liters of Pepsi rather than 5, and so
the budget constraint becomes flatter.

\fig{e41}{0.75}

\subsection{Indifference Curves}

Our goal in this section is to understand how consumers make choices. The budget constraint is one piece of the
analysis: It shows the combinations of goods a consumer can afford given her income and the prices of the goods. The
consumer's choices, however, depend not only on her budget constraint but also on her preferences regarding the two
goods. Therefore, the consumer's preferences are the next piece of our analysis. \v

The consumer's preferences allow her to choose among different bundles of pizza and Pepsi. If you offer the consumer
two different bundles, she chooses the bundle that best suits her tastes. If the two bundles suit her tastes equally,
we say that the consumer is indifferent between the two bundles. \v

Just as we have represented the consumer's budget constraint graphically, we can also represent her preferences
graphically. We do this with indifference curves. An indifference curve shows the various bundles of consumption that
make the consumer equally happy. In this case, the indifference curves show the combinations of pizza and Pepsi with
which the consumer is equally satisfied.

\bd[Indifference Curve]
An \textbf{indifference curve}, with respect to two commodities, is a graph showing those combinations of the two
commodities that leave the consumer equally well off or equally satisfied—Hence, indifferent—in having any combination
on the curve.
\ed

Figure 3 shows two of the consumer's many indifference curves. We can see that the consumer is indifferent among
combinations A, B, and C because they are all on the same curve. Not surprisingly, if the consumer's consumption of
pizza decreases, say, from point A to point B, consumption of Pepsi must increase to keep her equally happy. If
consumption of pizza decreases again, from point B to point C, the amount of Pepsi consumed must increase yet again. \v

The slope at any point on an indifference curve equals the rate at which the consumer is willing to substitute one
good for the other. (Once again the slope is negative, but for our purposes, we can ignore the minus sign.) This rate
is called the ``marginal rate of substitution'' (MRS).

\fig{e42}{0.7}

\bd[Marginal Rate of Substitution (MRS)] The \textbf{marginal rate of substitution} (\textbf{MRS}) is the amount of a
good that a consumer is willing to consume compared to another good, as long as the new good is equally satisfying.
The MRS is equal to the slope of consumer's indifference curve. \ed

In this case, the marginal rate of substitution measures how much additional Pepsi the consumer requires to be
compensated for a one unit reduction in pizza consumption. Notice that because the indifference curves are not
straight lines, the marginal rate of substitution is not the same at all points on a given indifference curve. The
rate at which a consumer is willing to trade one good for the other depends on the amounts of the goods she is
already consuming. In other words, the rate at which a consumer is willing to trade pizza for Pepsi depends on
whether she is hungrier or thirstier, and her hunger and thirst in turn depend on her current consumption of pizza
and Pepsi. \v

The consumer is equally happy at all points on any given indifference curve, but she prefers some indifference curves
to others. Because she prefers more consumption to less, higher indifference curves are preferred to lower ones. In
Figure 3, any point on curve $I_2$ is preferred to any point on curve $I_1$. \v

A consumer's set of indifference curves gives a complete ranking of the consumer's preferences. That is, we can use
the indifference curves to rank any two bundles of goods. For example, the indifference curves tell us that the
bundle at point D is preferred to the bundle at point A because point D is on a higher indifference curve than point
A. The indifference curves also tell us that the bundle at point D is preferred to the bundle at point C because
point D is on a higher indifference curve. Even though point D has less Pepsi than point C, it has more than enough
extra pizza to make the consumer prefer it. By seeing which point is on the higher indifference curve, we can use the
set of indifference curves to rank any combination of pizza and Pepsi. \v

Because indifference curves represent a consumer's preferences, they have certain properties that reflect those
preferences. Here we consider four properties that describe most indifference curves:
\bit
\item \textbf{Property 1: Higher indifference curves are preferred to lower ones.}
People usually prefer to consume more rather than less. This preference for greater quantities is reflected in the
indifference curves. As Figure 3 shows, higher indifference curves represent larger quantities of goods than lower
indifference curves. Thus, a consumer prefers being on higher indifference curves.
\item \textbf{Property 2: Indifference curves slope downward.}
The slope of an indifference curve reflects the rate at which a consumer is willing to substitute one good for the
other. In most cases, the consumer likes both goods. Therefore, if the quantity of one good decreases, the quantity
of the other good must increase for the consumer to be equally happy. For this reason, most indifference curves slope
downward.
\item \textbf{Property 3: Indifference curves do not cross.}
To see why this is true, suppose that two indifference curves did cross, as in Figure 4. Then, because point A is on
the same indifference curve as point B, the two points would make the consumer equally happy. In addition, because
point B is on the same indifference curve as point C, these two points would make the consumer equally happy. But
these conclusions imply that points A and C would also make the consumer equally happy, even though point C has more
of both goods. This contradicts our assumption that the consumer always prefers more of both goods to less. Thus,
indifference curves cannot cross.

\fig{e43}{0.8}

\item \textbf{Property 4: Indifference curves are bowed inward.}
The marginal rate of substitution usually depends on the amount of each good the consumer is currently consuming. In
particular, because people are more willing to trade away goods that they have in abundance and less willing to trade
away goods of which they have little, the indifference curves are bowed inward toward the graph's origin. As an
example, consider Figure 5. At point A,the consumer has a lot of Pepsi and only a little pizza, so she is very hungry
but not very thirsty. To willingly give up 1 pizza, she would have to receive 6 liters of Pepsi: The MRS is 6 liters
of Pepsi per pizza. By contrast, at point B, the consumer has little Pepsi and a lot of pizza, so she is very thirsty
but not very hungry. At this point, she would be willing to give up 1 pizza to get 1 liter of Pepsi: The MRS is 1
liter of Pepsi per pizza. Thus, the bowed shape of the indifference curve reflects the consumer's greater willingness
to give up a good that she already has in abundance.

\fig{e44}{0.8}
\eit

The shape of an indifference curve reveals the consumer's willingness to trade one good for the other. When the goods
are easy to substitute for each other, the indifference curves are less bowed; when the goods are hard to substitute,
the indifference curves are very bowed. To see why this is true, let's consider two extreme cases. \v

Suppose that someone offered you bundles of nickels and dimes. How would you rank the different bundles? Most likely,
you would care only about the total monetary value of each bundle. If so, you would always be willing to trade 2
nickels for 1 dime. Your marginal rate of substitution between nickels and dimes would be a fixed number: $MRS = 2$,
regardless of the number of nickels and dimes in the bundle. We can represent your preferences for nickels and dimes
with the indifference curves in panel (a) of Figure 6. Because the marginal rate of substitution is constant, the
indifference curves are straight lines. In this case of straight indifference curves, we say that the two goods are
perfect substitutes. \v

\fig{e45}{0.75}

\bd[Perfect Substitutes]
\textbf{Perfect substitutes} are two goods which are indistinguishable to each other. Their prices must be the same
if both are to be used: the elasticity of substitution between them is infinite, and any price difference will lead
to all consumers choosing the cheaper. An indifference curve between them is a straight line.
\ed

Suppose now that someone offered you bundles of shoes. Some of the shoes fit your left foot, others your right foot.
How would you rank these different bundles? In this case, you might care only about the number of pairs of shoes. In
other words, you would judge a bundle based on the number of pairs you could assemble from it. A bundle of 5 left
shoes and 7 right shoes yields only 5 pairs. Getting 1 more right shoe has no value if there is no left shoe to go
with it. We can represent your preferences for right and left shoes with the indifference curves in panel (b) of
Figure 6. In this case, a bundle with 5 left shoes and 5 right shoes is just as good as a bundle with 5 left shoes
and 7 right shoes. It is also just as good as a bundle with 7 left shoes and 5 right shoes. The indifference curves,
therefore, are right angles. In this case of right-angle indifference curves, we say that the two goods are perfect
complements.

\bd[Perfect Complements]
\textbf{Perfect complements} are two goods that must be consumed to each other. The indifference curve of a perfect
complement exhibits a right angle
\ed

In the real world, most goods are neither perfect substitutes nor perfect complements. Perfect substitutes and
perfect complements are extreme cases. They are introduced here not because they are common but because they
illustrate how indifference curves reflect a consumer's preferences. For most goods, the indifference curves are
bowed inward, but not so bowed that they become right angles.

\subsection{Optimization}

The goal of this chapter is to understand how a consumer makes choices. We have the two pieces necessary for this
analysis: the consumer's budget constraint (which shows what bundles of goods she can afford) and the consumer's
preferences (which show what bundles of goods she most likes). Now we put these two pieces together and consider the
consumer's decision about what to buy. \v

Once again, consider our pizza and Pepsi example. The consumer would like to end up with the best possible
combination of pizza and Pepsi for her—that is, the combination on her highest possible indifference curve. But the
consumer must also end up on or below her budget constraint, which measures the total resources available to her. \v

Figure 7 shows the consumer's budget constraint and three of her many indifference curves.

\fig{e47}{0.85}

The highest indifference curve that the consumer can reach ($I_2$ in the figure) is the one that just barely touches
her budget constraint. The point at which this indifference curve and the budget constraint touch is called the
optimum. The consumer would prefer point A, but she cannot afford that bundle of goods because it lies above her
budget constraint. The consumer can afford point B, but that bundle of goods is on a lower indifference curve and,
therefore, provides the consumer less satisfaction. The optimum represents the best bundle of pizza and Pepsi that
the consumer can afford. \v

Notice that, at the optimum, the slope of the indifference curve equals the slope of the budget constraint. We say
that the indifference curve is tangent to the budget constraint. The slope of the indifference curve is the marginal
rate of substitution between pizza and Pepsi, and the slope of the budget constraint is the relative price of pizza
and Pepsi. \textbf{Thus, the consumer chooses the quantities of the two goods so that the marginal rate of
substitution equals the relative price.} \v

In a previous section, we saw how market prices reflect the marginal value that consumers place on goods. This
analysis of consumer choice shows the same result in another way. In making her consumption choices, the consumer
takes the relative price of the two goods as given and then chooses an optimum bundle of goods at which her marginal
rate of substitution equals this relative price. The relative price is the rate at which the market is willing to
trade one good for the other, whereas the marginal rate of substitution is the rate at which the consumer is willing
to trade one good for the other. At the consumer's optimum, her valuation of the two goods (as measured by the
marginal rate of substitution) equals the market's valuation (as measured by the relative price). As a result of this
consumer optimization, market prices of different goods reflect the value that consumers place on those goods. \v

Now that we have seen how the consumer makes a consumption decision, let's examine how this decision responds to
changes in the consumer's income. To be specific, suppose that income increases. As we have discussed, an increase in
income leads to a parallel outward shift in the budget constraint, as in Figure 8. Because the relative price of the
two goods has not changed, the slope of the new budget constraint is the same as the slope of the initial budget
constraint.

\fig{e70}{0.9}

The expanded budget constraint allows the consumer to choose a more desirable combination of pizza and Pepsi and
therefore reach a higher indifference curve. Given the shift in the budget constraint and the consumer's preferences
as represented by her indifference curves, the consumer's optimum moves from the point labeled ``initial optimum'' to
the point labeled ``new optimum''. \v

Notice that, in Figure 8, the consumer chooses to consume more Pepsi and more pizza. The logic of the model does not
require increased consumption of both goods in response to increased income, but this situation is the most common.
As you may recall, if a consumer wants more of a good when her income rises, economists call it a normal good. The
indifference curves in Figure 8 are drawn under the assumption that both pizza and Pepsi are normal goods. Figure 9
shows an example in which an increase in income induces the consumer to buy more pizza but less Pepsi. If a consumer
buys less of a good when her income rises, economists call it an inferior good. Figure 9 is drawn under the
assumption that pizza is a normal good and Pepsi is an inferior good.

\fig{e71}{0.9}

Let's now use this model of consumer choice to consider how a change in the price of one of the goods alters the
consumer's choices. Suppose, in particular, that the price of Pepsi falls. As we discussed earlier, a fall in the
price of either good shifts the budget constraint outward and, by changing the relative price of the two goods,
changes the slope of the budget constraint as well. Figure 10 shows how the fall in the price of Pepsi rotates the
budget constraint and thus changes the consumer's optimum.

\fig{e72}{0.9}

How such a change in the budget constraint alters the quantities of the two goods purchased depends on the consumer's
preferences. For the indifference curves drawn in this figure, the consumer buys more Pepsi and less pizza. But it
takes only a little creativity to draw indifference curves with other outcomes. A consumer could plausibly respond to
the lower price of Pepsi by buying more of both goods. \v

The impact of a change in the price of a good on the quantities purchased can be decomposed into two effects: an
``income effect'' and a ``substitution effect''.

\bd[Income Effect]
The \textbf{income effect} is the change in demand for a good or service caused by a change in a consumer's
purchasing power resulting from a change in real income.
\ed

\bd[Substitution Effect]
The \textbf{substitution effect} is the decrease in sales for a product that can be attributed to consumers switching
to cheaper alternatives when its price rises.
\ed

To see what these two effects are, consider how our consumer might respond when she learns that the price of Pepsi
has fallen. She might reason in the following ways:
\bit
\item \textbf{Income Effect} ``Great news! Now that Pepsi is cheaper, my income has greater purchasing power. I am, in
effect, richer than I was. Because I am richer, I can buy both more pizza and more Pepsi''.
\item \textbf{Substitution Effect} ``Now that the price of Pepsi has fallen,I get more liters of Pepsi for every pizza
that I give up. Because pizza is now relatively more expensive, I should buy less pizza and more Pepsi''.
\eit

Both of these statements make sense. The decrease in the price of Pepsi makes the consumer better off. If pizza and
Pepsi are both normal goods, the consumer will want to spread this improvement in her purchasing power over both
goods. This income effect tends to make the consumer buy more pizza and more Pepsi. Yet at the same time, consumption
of Pepsi has become less expensive relative to consumption of pizza. This substitution effect tends to make the
consumer choose less pizza and more Pepsi. \v

Now consider the result of these two effects working at the same time. The consumer certainly buys more Pepsi because
the income and substitution effects both act to increase consumption of Pepsi. But for pizza, the income and
substitution effects work in opposite directions. As a result, whether the consumer buys more or less pizza is not
clear. The outcome could go either way, depending on the relative magnitudes of the income and substitution effects. \v

We can interpret the income and substitution effects using indifference curves. The income effect is the change in
consumption that results from the movement to a new indifference curve. The substitution effect is the change in
consumption that results from moving to a new point on the same indifference curve with a different marginal rate of
substitution.

\fig{e74}{0.9}

Figure 11 shows graphically how to decompose the change in the consumer's decision into the income effect and the
substitution effect. When the price of Pepsi falls, the consumer moves from the initial optimum, point A, to the new
optimum, point C. We can view this change as occurring in two steps. First, the consumer moves along the initial
indifference curve, $I_1$, from point A to point B. The consumer is equally happy at these two points, but at point
B, the marginal rate of substitution reflects the new relative price. (The dashed line through point B is parallel to
the new budget constraint and thus reflects the new relative price.) Next, the consumer shifts to the higher
indifference curve, $I_2$, by moving from point B to point C. Even though point B and point C are on different
indifference curves, they have the same marginal rate of substitution. That is, the slope of the indifference curve
$I_1$ at point B equals the slope of the indifference curve $I_2$ at point C. \v

The consumer never actually chooses point B, but this hypothetical point is useful to clarify the two effects that
determine the consumer's decision. Notice that the change from point A to point B represents a pure change in the
marginal rate of substitution without any change in the consumer's welfare. Similarly, the change from point B to
point C represents a pure change in welfare without any change in the marginal rate of substitution. Thus, the
movement from A to B shows the substitution effect, and the movement from B to C shows the income effect. \v

We have just seen how changes in the price of a good alter the consumer's budget constraint and, therefore, the
quantities of the two goods that she chooses to buy. The demand curve for any good reflects these consumption
decisions because it shows the quantity demanded of a good for any given price. A consumer's demand curve is a
summary of the optimal decisions that arise from her budget constraint and indifference curves. \v

For example, Figure 12 considers the demand for Pepsi. Panel (a) shows that when the price of a liter falls from \$2
to \$1, the consumer's budget constraint shifts outward. Because of both income and substitution effects, the
consumer increases her purchases of Pepsi from 250 to 750 liters. Panel (b) shows the demand curve that results from
this consumer's decisions. In this way, the theory of consumer choice provides the theoretical foundation for the
consumer's demand curve.

\fig{e75}{0.9}

It may be comforting to know that the demand curve arises naturally from the theory of consumer choice, but this
exercise by itself does not justify developing the theory. There is no need for a rigorous, analytic framework just
to establish that people respond to changes in prices. The theory of consumer choice is, however, useful in studying
various decisions that people make as they go about their lives.

\section{Frontiers Of Microeconomics}

Economics is a study of the choices that people make and the interactions among people as they go about their lives.
As the preceding sections demonstrate, the field has many facets. Yet it would be a mistake to think that the facets
we have seen make up a finished jewel, perfect and unchanging. Like all scientists, economists are always looking for
new areas to study and new phenomena to explain. This final section on microeconomics discusses three topics at the
discipline's frontier to show how microeconomics is trying to expand its understanding of human behavior and society.

\subsection{Asymmetric Information}

Many times in life, one person knows more about what is going on than another. A difference in access to knowledge
that is relevant to an interaction is called an ``information asymmetry''.

\bd[Information Asymmetry]
\textbf{Information asymmetry} occurs when one party to an economic transaction possesses greater material knowledge
than the other party.
\ed

Asymmetric information typically manifests when the seller of a good or service possesses greater knowledge than the
buyer; however, the reverse dynamic is also possible. Almost all economic transactions involve information
asymmetries. Because asymmetric information is so prevalent, economists have devoted much effort in recent decades to
studying its effects. In what follows we will examine two kinds of asymmetric information: ``moral hazard'' and
``adverse selection'', and we will see how individuals may respond to the problem with ``signaling'' or ``screening''. \v

Moral hazard is a problem that arises when one person, called the agent, performs some task on behalf of another
person, called the principal. If the principal cannot perfectly monitor the agent's behavior, the agent tends to
undertake less effort than the principal considers desirable.

\bd[Moral Hazard]
\textbf{Moral hazard} is the risk that a party has not entered into a contract in good faith or has provided
misleading information about its assets, liabilities, or credit capacity. In addition, moral hazard also may mean a
party has an incentive to take unusual risks in a desperate attempt to earn a profit before the contract settles.
\ed

Moral hazards can be present at any time two parties come into agreement with one another. Each party in a contract
may have the opportunity to gain from acting contrary to the principles laid out by the agreement.

\bd[Agent]
An \textbf{agent} is a person who has been legally empowered to act on behalf of another person or an entity.
\ed

An agent may be employed to represent a client in negotiations and other dealings with third parties. The agent may
be given decision-making authority.

\bd[Principal]
A \textbf{principal} is the person who legally empowers an agent to act on behalf of them.
\ed

The phrase moral hazard refers to the risk, or ``hazard'' of inappropriate or otherwise ``immoral'' behavior by the
agent. In such a situation, the principal tries various ways to encourage the agent to act more responsibly. This is
the so called ``principal-agent problem''.

\bd[Principal-Agent Problem]
The \textbf{principal-agent problem} is a conflict in priorities between a person or group and the representative
authorized to act on their behalf.
\ed

An agent may act in a way that is contrary to the best interests of the principal. The principal-agent problem is as
varied as the possible roles of principal and agent. It can occur in any situation in which the ownership of an
asset,or a principal, delegates direct control over that asset to another party, or agent. \v

Another type of asymmetric information is ``adverse selection''. Adverse selection is a problem that arises in
markets in which the seller knows more about the attributes of the good being sold than the buyer does. In such a
situation, the buyer runs the risk of being sold a good of low quality. That is, the ``selection'' of goods sold may
be ``adverse'' from the standpoint of the uninformed buyer. When markets suffer from adverse selection, the invisible
hand does not necessarily work its magic.

\bd[Adverse Selection]
\textbf{Adverse selection} refers generally to a situation in which sellers have information that buyers do not have, or
vice versa, about some aspect of product quality.
\ed

Asymmetric information motivates individual behavior that otherwise might be hard to explain. Markets respond to
problems of asymmetric information in many ways. One of them is ``signaling'', which refers to actions taken by an
informed party for the sole purpose of credibly revealing his private information and another one is ``screening''
which refers to actions to induce the informed party to reveal private information.

\bd[Signaling]
\textbf{Signaling} is an action taken by an informed party to reveal private information to an uninformed party.
\ed

\bd[Screening]
\textbf{Screening} is an action taken by an uninformed party to induce an informed party to reveal information.
\ed

Adam Smith's invisible hand seemed to reign supreme but as we saw, government can sometimes improve market outcomes.
The study of asymmetric information gives us a new reason to be wary of markets. When some people know more than
others, the market may fail to put resources to their best use. Asymmetric information may justify government action
in some cases, but three facts complicate the issue. First, as we have seen, the market can sometimes deal with
information asymmetries on its own using a combination of signaling and screening. Second, the government rarely has
more information than the private parties. Even if the market's allocation of resources is not ideal, it may be the
best that can be achieved. That is, when there are information asymmetries, policymakers may find it hard to improve
upon the market's admittedly imperfect outcome. Third, the government is itself an imperfect institution, as we
discuss in the next section.

\subsection{Political Economy}

As we have seen, markets on their own do not always reach a desirable allocation of resources. When we judge the
market's outcome to be either inefficient or inequitable, there may be a role for the government to improve the
situation. Yet before embracing an activist government, we need to consider one more fact: The government is also an
imperfect institution. The field of ``political economy'' uses the methods of economics to study how government works.

\bd[Political Economy]
\textbf{Political economy} is an interdisciplinary branch of economics that focuses on the interrelationships among
individuals, governments, and public policy.
\ed

Political economy is the study of how economic systems (e.g.\ markets and national economies) and political systems
(e.g.\ law, institutions, government) are linked. Widely studied phenomena within the discipline are systems
such as labour markets and financial markets, as well as phenomena such as growth, distribution, inequality, and
trade, and how these are shaped by institutions, laws, and government policy. Originating in the 16th century, it is
the precursor to the modern discipline of economics. Political economy in its modern form is considered an
interdisciplinary field, drawing on theory from both political science and modern economics. \v

Political economy originated within 16th century western moral philosophy, with theoretical works exploring the
administration of states' wealth; political signifying the Greek word polity and economy signifying the Greek
word for household management. The earliest works of political economy are usually attributed to the British
scholars Adam Smith, Thomas Malthus, and David Ricardo, although they were preceded by the work of the French
physiocrats, such as Francois Quesnay (1694–1774) and Anne-Robert-Jacques Turgot (1727–1781). \v

Most advanced societies rely on democratic principles to set government policy. In this case, as the 18th-century
French political theorist Marquis de Condorcet famously noted, democracy might run into some problems trying to
choose the best outcome. More specifically, the ``Condorcet paradox'' is that democratic outcomes do not always obey
the transitive property which states that if A is preferred to B, and B is preferred to C, then we would expect A to
be preferred to C. Since political theorists first noticed the Condorcet paradox, they have spent much energy studying
existing voting systems and proposing new ones.

\be
For example, as an alternative to pairwise majority voting, the mayor of our town could ask each voter to rank
the possible outcomes. For each voter, we could give 1 point for last place, 2 points for second to last, 3 points
for third to last, and so on. The outcome that receives the most total points wins. With the preferences in Table 1,
outcome B is the winner. This voting method is called a ``Borda count''.
\ee

\be
For example, suppose there are three possible outcomes, labeled A, B, and C, and three voter types with the
preferences shown in Table 1. The mayor of our town wants to aggregate these individual preferences into preferences
for society as a whole. How should he do it?

\fig{e58}{0.84}

At first, he might try some pairwise votes. If he asks voters to choose first between B and C, voter types 1 and 2
will vote for B, giving B the majority. If he then asks voters to choose between A and B, voter types 1 and 3 will
vote for A, giving A the majority. Observing that A beats B and that B beats C, the mayor might conclude that A is
the voters' clear choice. But wait: Suppose the mayor then asks voters to choose between A and C. In this case, voter
types 2 and 3 vote for C, giving C the majority. That is, under pairwise majority voting, A beats B, B beats C, and C
beats A. \v

One implication of the Condorcet paradox is that the order in which things are voted on can affect the result. If the
mayor suggests choosing first between A and B and then comparing the winner to C, the town ends up choosing C. But if
the voters choose first between B and C and then compare the winner to A, the town ends up with A. And if the voters
choose first between A and C and then compare the winner to B, the town ends up with B. The Condorcet paradox teaches
two lessons. The narrow lesson is that when there are more than two options, setting the agenda (that is, deciding
the order in which items are voted on) can have a powerful influence over the outcome of a democratic election. The
broad lesson is that majority voting by itself does not tell us what outcome a society really wants.
\ee

Given the discussion so far an inevitable questions arises. Is there a perfect voting system? Economist Kenneth Arrow
took up this question in his 1951 book Social Choice and Individual Values. Arrow started by defining what a perfect
voting system would be. He assumes that individuals in society have preferences over the various possible outcomes:
A, B, C, and so on. He then assumes that society wants a voting system to choose among these outcomes that satisfies
several properties:
\bit
\item{Unanimity}: If everyone prefers A to B, then A should beat B\@.
\item{Transitivity}: If A beats B, and B beats C, then A should beat C\@.
\item{Independence Of Irrelevant Alternatives}: The ranking between any two outcomes A and B should not depend on
whether some third outcome C is also available.
\item{No Dictators}: There is no person who always gets his way, regardless of everyone else's preferences.
\eit

These all seem like desirable properties of a voting system. Yet Arrow proved, mathematically and incontrovertibly,
that no voting system can satisfy all these properties. This amazing result is called Arrow's impossibility
theorem. Despite Arrow's theorem, voting is how most societies choose their leaders and public policies, often by
majority rule. However, politicians also have objectives. It would be nice to assume that political leaders always
look out for the well-being of society as a whole, and that they aim for an optimal combination of efficiency and
equality. Nice, perhaps,but not realistic. Self-interest is as powerful a motive for political actors as it is for
consumers and firm owners. Some politicians, motivated by a desire for reelection, are willing to sacrifice the
national interest to solidify their base of voters. Others are motivated by simple greed. If you have any doubt, you
should look at the world's poor nations, where corruption among government officials is a common impediment to
economic development.

\subsection{Behavioral Economics}

Economics is a study of human behavior, but it is not the only field that can make that claim. The social science of
psychology also sheds light on the choices that people make in their lives. The fields of economics and psychology
usually proceed independently, in part because they address different questions. But recently, a field called
behavioral economics has emerged in which economists are making use of psychological insights to better understand
the decisions that people make. Let's consider some of these insights.

\bd[Behavioral Economics]
\textbf{Behavioral economics} is the study of psychology as it relates to the economic decision-making processes of
individuals and institutions.
\ed

Economic theory is populated by a particular species of organism, sometimes called Homo economicus. Members of this
species are always rational. As firm owners, they maximize profits. As consumers, they maximize utility (or
equivalently, pick the point on the highest indifference curve). Given the constraints they face, they rationally
weigh all the costs and benefits and always choose the best possible course of action. \v

Real people, however, are Homo sapiens. Although in many ways they resemble the rational, calculating people assumed
in economic theory, they are more complex. They can be forgetful, impulsive, confused, emotional, and shortsighted.
These imperfections of human reasoning are a central focus of psychologists but, until recently, have often been
neglected by economists. Herbert Simon, one of the first social scientists to work at the boundary of economics and
psychology, suggested that humans should be viewed not as rational maximizers but as satisficers. Instead of always
choosing the best course of action, they make decisions that are merely good enough. Similarly, other economists have
suggested that humans are only ``near rational'' or that they exhibit ``bounded rationality''. \v

Since deviations from rationality are important for understanding economic phenomena, then one might ask, is
economics built on the rationality assumption when psychology and common sense cast doubt on it? \v

One answer is that the assumption, even if not exactly true, may be true enough that it yields reasonably accurate
models of behavior. Incorporating complex psychological deviations from rationality into the story might have added
realism, but it would have also muddied the waters and made those insights harder to find. \v

Another reason economists often assume rationality may be that economists are themselves not rational maximizers.
Like most people, they are overconfident and reluctant to change their minds. Their choice among alternative theories
of human behavior may exhibit excessive inertia. Moreover, economists may be content with a theory that is not
perfect but is good enough. The model of rational man may be the theory of choice for a satisficing social scientist.