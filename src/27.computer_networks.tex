\section{Networks \& Nodes}

\bd[Computer Network]
A \textbf{computer network} (or more simply \textbf{network}) is a set of devices able to communicate with one another,
exchange data, and share information.
\ed

\bd[Network Node]
A \textbf{network node} (or more simply \textbf{node}) is a piece of hardware that is connected to a network and is
capable of sending and receiving information.
\ed

Nodes can be either ``end nodes'' (``end devices''), or ``intermediate nodes'' (``intermediary devices'').

\bd[End Node / End Device]
An \textbf{end node} (or \textbf{end device}) is a node that is either the source or the destination of a message 
transmitted over the network.
\ed

\bd[Intermediate Node / Intermediary Device]
An \textbf{intermediate node} (or \textbf{intermediate device}) is a node that forwards messages over the network.
\ed

\subsection{Client-Server Model}

Nodes are divided into two categories: ``servers'' and ``clients''.

\bd[Server]
A \textbf{server} is a node that provides functionality for other nodes in a network.
\ed

These functionalities are often called ``services''.

\bd[Service]
A \textbf{service} is a piece of functionality a server provides.
\ed

\bd[Client]
A \textbf{client} is a node that accesses a service made available by a server in a computer network.
\ed

Servers can provide various services such as sharing data or resources among multiple clients, or performing
computation for a client. A single server can serve multiple clients, and a single client can use multiple servers. A
client process may run on the same device or may connect over a network to a server on a different device. Servers
and clients are integrated in the so called ``client-server'' model.

\bd[Client–Server Model]
The \textbf{client–server model} is a distributed application structure that partitions tasks or workloads between the
providers of a resource or service, called servers, and service requesters, called clients.
\ed

\section{Characteristics Of Networks}

There are several characteristics that can be used to describe a network.

\bd[Network Size]
The \textbf{size} of a network is the number of nodes it contains.
\ed

\bd[Network Traffic]
The \textbf{traffic} of a network is the amount of data that is transmitted over the network.
\ed

\bd[Network Speed]
The \textbf{speed} of a network is the rate at which data is transmitted over the network.
\ed

Moving on to some more advanced characteristics, we have the concepts of ``fault tolerance'' and ``network reliability'',
``network scalability'', ``quality of service'' and ``network security''.

\bd[Fault Tolerance]
The \textbf{fault tolerance} of a network is the ability of the network to continue to function and ensure no loss of
of service even if some of its nodes fail.
\ed

Close related to fault tolerance is the concept of ``network reliability''.

\bd[Network Reliability]
The \textbf{network reliability} of a network is the probability that the network will work as expected.
\ed

\bd[Network Scalability]
The \textbf{network scalability} is the ability of the network to expand and support more nodes, as needed, without
degrading its performance.
\ed

\bd[Quality Of Service (QoS)]
The \textbf{quality of service} (or \textbf{QoS}) of a network is the ability of the network to provide different
priority to different entities of the network, and to guarantee a certain level of performance to a data flow.
\ed

\bd[Network Security]
The \textbf{network security} of a network is the ability of the network to protect itself from unauthorized access,
modification and withholding of information.
\ed

\section{Types Of Networks}

\bd[Network Range]
The \textbf{range} of a network is the geographical area it covers.
\ed

There are several types of networks depending on their range. Here are the most important ones.
\bit
\item \textbf{Body Area Network (BAN)}: A Body Area Network (BAN) (or in the case of wireless a Wireless Body Area
Network (WBAN)) is a wired/wireless network of wearable computing devices.
\item \textbf{Personal Area Network (PAN)}: A Personal Area Network (PAN) is a network for interconnecting electronic
devices within an individual person's workspace. A PAN provides data transmission among devices such as computers,
smartphones, tablets and personal digital assistants.
\item \textbf{Small Office/Home Office (SOHO)}: Small Office/Home Office (SOHO) networks are small networks typically
consisted of less than 10 computers.
\item \textbf{Local Area Network (LAN)}: A Local Area Network (LAN) (or in the case of wireless a Wireless Local Area
Network (WLAN)) is a network that interconnects computers within a limited area such as a residence, school,
laboratory, university campus or office building.
\item \textbf{Campus Area Network (CAN)}: A Campus Area Network (CAN) is a network made up of an interconnection of
local area networks (LANs) within a limited geographical area.
\item \textbf{Metropolitan Area Network (MAN)}: A Metropolitan Area Network (MAN) is a computer network that
interconnects users with computer resources in a geographic region of the size of a metropolitan area.
\item \textbf{Wide Area Network (WAN)}: A Wide Area Network (WAN) is a telecommunications network that extends over a
large geographic area for the primary purpose of computer networking.
\eit

\fig{net}{0.8}

\section{Network Topologies}

\bd[Network Topology]
The arrangement of the nodes in a network is called \textbf{network topology}.
\ed

In what follows we will introduce the most basic and most heavily used network topologies\footnote{Topology can
either refer to the physical or logical layout of a network. Physical topology means the physical design of a network
including the devices, locations and cables. Logical topology refers to how data is actually transferred in a network
as opposed to its physical design.}.

\bd[Bus Topology]
A \textbf{bus topology} is a network topology in which nodes are connected in a daisy chain by a linear sequence of
communication links.
\ed

In a bus topology all data transmitted between nodes in the network are transmitted over a common transmission medium
and are able to received by all nodes in the network simultaneously. A signal containing the address of the intended
receiving machine travels from a source machine in both directions to all machines connected to the bus until it
finds the intended recipient. \v

Bus networks are relatively inexpensive and easy to install for small, usually temporary, networks since they require
less cable than other topologies. However, if the backbone cable fails, the entire network effectively becomes
unusable, has they do not have fault tolerance. In addition, bus networks are difficult to administer and maintain,
and they are not secure. \v

Bus networks are used in small networks, such as workgroups, departments, or classrooms, where the number of nodes is
limited. \v

\bd[Ring Topology]
A \textbf{ring topology} is a network topology in which each node is connected to exactly two other nodes, forming a
single continuous pathway for signals through each node.
\ed

In a ring topology all data transmitted between nodes in the network are transmitted over a common transmission medium
and are able to received by all nodes in the network simultaneously. In simple words, a ring topology is a bus
topology in a closed, unidirectional loop. \v

Ring networks perform better than bus networks under heavy network load. However, similar to bus networks, a single
point of failure will affect the whole network, making the network unreliable with no fault tolerance. It is also a
difficult network topology to administer and maintain, and it is not secure. \v

\bd[Star Topology]
A \textbf{star topology} is a network topology in which each node is connected to a central hub or switch.
\ed

In a star topology every node is connected to a central node called a ``hub'' or ``switch'' (more on that later). It is
a centralized management topology since all traffic must pass through the hub or switch. \v

Star networks are relatively easy to install and manage, and they are scalable. However, they are more expensive due to
the cost of the central hub or switch, and they are dependent on the hub or switch. If the hub or switch fails, the
entire network becomes unusable. \v

\bd[Mesh Topology]
A \textbf{mesh topology} is a network topology in which each node is connected to every other node in the network.
\ed

In a mesh topology every node is connected to every other node in the network, making it a fully connected topology.
As a topolgy it is fault tolerant and reliable. However, they have issues with broadcasting messages, and they are
expensive and impractical for large networks.

\bd[Tree Topology]
A \textbf{tree topology} is a network topology in which nodes are arranged in a hierarchy, where each node is
connected to a parent node and each parent node can have several child nodes.
\ed

In a tree topology every node is connected to a parent node, except the root node which has no parent. It is a
centralized management, hierarchical, topology where the nodes are arranged in a tree-like structure. \v

\bd[Hybrid Topology]
A \textbf{hybrid topology} is a network topology that is a combination of two or more basic network topologies.
\ed

\fig{network_topologies}{0.45}

\section{Network Communication}

As we have seen, a network is a set of nodes that are able to communicate with one another. All communication systems
have the following things in common: a ``source'' (or ``sender''), a ``destination'' (or ``receiver''), a ``medium''
(or ``chanell''), and ``rules'' (or ``protocols'') that govern the communication. \v

Let's begin by defining the first 3 terms before focusing on the fourth and most important one.

\bd[Source]
The \textbf{source} (or \textbf{sender}) is the node that sends the data.
\ed

\bd[Destination]
The \textbf{destination} (or \textbf{receiver}) is the node that receives the data.
\ed

\bd[Medium]
The \textbf{medium} (or \textbf{chanell}) is the physical path by which a message travels from sender to receiver.
\ed

Specifically in compouter networks, the medium where the exhcange of information between nodes takes place is called a
``communication link''.

\bd[Communication Link]
A \textbf{communication link} is the medium that connects two or more nodes in a network, and carries the exchanged
information.
\ed

Communication among nodes can be established over a wired or a wireless communication link. Based on these, networks
are divided into two categories: ``wired'' and ``wireless''.

\bd[Wired Network]
A \textbf{wired network} is a network that uses wired communication links to connect the nodes.
\ed

\bd[Wireless Network]
A \textbf{wireless network} is a network that uses wireless communication links to connect the nodes.
\ed

We formally define the communication of the nodes through a communication link as ``data communication'' or ``data
transmission''.

\bd[Data Transmission / Data Communication]
\textbf{Data transmission} or \textbf{data communication} is the exchange of data between two nodes via some form of a
communication link.
\ed

Depending on the number of  senders and receivers, data transmission can be either ``unicast'', ``multicast'', or
``broadcast'':
\bit
\item \textbf{Unicast}: Data is transmitted from a single sender to a single receiver. (E.g.\ Telephone)
\item \textbf{Multicast}: Data is transmitted from a single sender to multiple receivers. (E.g.\ Radio)
\item \textbf{Broadcast}: Data is transmitted from a single sender to all possible receivers. (E.g.\ Television)
\eit

Closely related to data transmission is the concept of ``data flow''.

\bd[Data Flow]
The \textbf{data flow} is the movement of data between two nodes during a data transmission.
\ed

Data flow can be either ``simplex'', ``half-duplex'' or ``full-duplex'':
\bit
\item \textbf{Simplex}: Data flows in only one direction. (E.g.\ Radio)
\item \textbf{Half-Duplex}: Data flows in both directions, but only in one direction at a time. (E.g.\ Walkie-Talkie)
\item \textbf{(Full) Duplex}: Data flows in both directions at the same time. (E.g.\ Telephone)
\eit

Having defined the basic terms, we can now move on to the most important one. Let's start by providing an example to
motivate the definition of a ``network protocol''.

\be
A telephone conversation between two people can be considered as a communication system. The two people are the
source and the destination, the telephone line is the channel, and the rules that govern the communication are the
spoken language. \v

Similarly, in a network, nodes act as sources and destinations, and the communications links act as the channels. The
only thing missing is to assure that all nodes in a network are able to communicate with each other. \v

In order for this to be able they must all ``speak'' the same ``language''. In networks, these ``languages'' are
called ``protocols''.
\ee

\bd[Network Protocol]
A \textbf{network protocol} (or simply \textbf{protocol}) is a set of rules that governs the data transmission between
nodes in a network.
\ed

\bd[Protocol Data Unit (PDU)]
A \textbf{Protocol Data Unit (PDU)} is a single unit of information transmitted among peer entities of a network.
\ed

A protocol determines what is communicated, how it is communicated, and when it is communicated. Among others, it is
responsible for the message's encoding, formatting, encapsulation, timing, size and delivery options.

\bd[Standard]
A \textbf{standard} is a formalized protocol accepted by most of the parties that implement it.
\ed

There are many protocols and standards depending on the task in hand.

\section{Open Systems Interconnection (OSI)}

\bd[Open Systems Interconnection (OSI)]
The \textbf{Open Systems Interconnection} (\textbf{OSI}) model is a conceptual model that provides a common basis for
the coordination of standards development for the purpose of systems interconnection.
\ed

The OSI model was introduced in 1983 by representatives of the major computer and telecom companies, and was adopted
by the International Organization for Standardization (ISO)\footnote{The International Organization for
Standardization (ISO) is an international standard development organization, founded on 23 February 1947, composed of
representatives from the national standards organizations of member countries. The organization develops and
publishes standardization in all technical and nontechnical fields other than electrical and electronic engineering.
Since its inception it has published over 24.500 international standards covering almost all aspects of technology
and manufacturing.} in 1984. It is by far the most widely used network model, and it is considered the primary
architectural model for intercomputer communications. \v

OSI was developed to support the emergence of the diverse networks that were competing for application in the large
national networking efforts in the world In simple words, it is a conceptual model that characterizes and
standardizes the communication functions of a network without regard to its underlying internal structure and
technology. Its goal is the interoperability of diverse communication systems with standard communication protocols. \v

It is important to note that the OSI model is a conceptual model, not an implementation of protocols. It is not a
protocol by itself, and it does not specify the services and protocols that are to be used in a communication system.
It is rather a reference model that describes how protocols should operate in order to facilitate communication
between different systems without requiring changes to the logic of the underlying hardware and software. \v

The OSI model uses layering to decompose the problem of system interconnection into more manageable components. That
way it provides a more modular design, which makes it easier to troubleshoot potentila problems. More specifically,
the model partitions the flow of data in a network into seven abstraction layers, from the physical implementation of
transmitting bits across a communications medium to the highest-level representation of data of a distributed
application. Each intermediate layer serves a class of functionality to the layer above it and is served by the layer
below it. \v

In what follows we will examine each layer of the OSI model in detail, and we will also take a look at some of the
most important protocols and standards of each layer.

\fig{net1}{0.5}

\subsection{Layer 1: Physical Layer}

\bd[Physical Layer]
The \textbf{physical layer} is the first layer of the OSI model and defines the means of transmitting a stream of raw
bits over a physical data link connecting network nodes. Its PDU is the bit.
\ed

The physical layer is a fundamental layer underlying the higher level functions in a network, and can be implemented
through a great number of different electronic circuit transmission technologies with widely varying characteristics.
It is the first and lowest layer in the OSI model, and it is the layer most closely associated with the physical
connection between devices, providing an electrical, mechanical, and procedural interface to the transmission medium.
Among others, it is responsible for the transmission and reception of unstructured streams of raw bits (physical layer's
PDU) between a device and a physical transmission medium. It then converts the digital bits into electrical, radio, or
optical signals. Layer specifications define all technical characteristics of a network. The components of a physical
layer can be described in terms of a network topology. \v

Here follows a list of the most common protocols and standards of the physical layer.
\bit
\item \textbf{Integrated Services Digital Network (ISDN)} is a set of communication standards for simultaneous
digital transmission of voice, video, data, and other network services over the digitalised circuits of the public
switched telephone network. Work on the standard began in 1980 at Bell Labs and was formally standardized in 1988. By
the time the standard was released, newer networking systems with much greater speeds were available, and ISDN saw
relatively little uptake in the wider market. ISDN has largely been replaced with digital subscriber line (DSL)
systems of much higher performance.
\item \textbf{Digital Subscriber Line (DSL)} is a family of technologies that are used to transmit digital data over
telephone lines. DSL service can be delivered simultaneously with wired telephone service on the same telephone line
since DSL uses higher frequency bands for data. The bit rate of consumer DSL services typically ranges from 256
kbit/s to over 100 Mbit/s in the direction to the customer (downstream), depending on DSL technology, line
conditions, and service-level implementation.
\item \textbf{Asymmetric Digital Subscriber Line (ADSL)} the most commonly installed DSL technology, for Internet
access. In ADSL, the data throughput in the upstream direction (the direction to the service provider) is lower,
hence the designation of asymmetric service.
\item \textbf{IEEE 802.3\footnote{IEEE 802 is a family of IEEE (Institute of Electrical and Electronics Engineers)
protocols and standards for local area networks (LAN), personal area network (PAN), and metropolitan area networks
(MAN). The number 802 has no significance, it was simply the next number in the sequence that the IEEE used for
standards projects. The services and protocols specified in IEEE 802 map to the lower two layers(physical and data
link) of OSI. The IEEE 802 family of standards has twelve members, numbered 802.1 through 802.12}} or \textbf{
Ethernet} where nodes in the LAN connect to a central router, hub, switch, or gateway which acts as a connecting
point between devices on the network through Unshielded Twisted Pair (UTP) cables. In this protocol, when a node on
the network wants to send data to another node, it senses the carrier, which is the main wire connecting the devices.
If it is free, meaning no one is sending anything, it sends the data packet on the network, and the other devices
check the packet to see whether they are the recipient. The recipient consumes the packet. If there is a packet on
the highway, the device that wants to send holds back for some thousandths of a second to try again until it can send.
\item \textbf{IEEE 802.11} or \textbf{Wireless LAN (WLAN)} tandards are the most widely used computer networks in the
world. These are commonly called Wi-Fi, which is a trademark belonging to the Wi-Fi Alliance.
\item \textbf{Universal Serial Bus (USB)} is an industry standard that establishes specifications for cables,
connectors and protocols for connection, communication and power supply (interfacing) between computers, peripherals
and other computers. Released in 1996, there have been four generations of USB specifications: USB 1.x, USB 2.0, USB
3.x, and USB4. A broad variety of USB hardware exists, including fourteen different connectors, of which USB-C, as of
now, is the most recent and best one.
\eit

\subsection{Layer 2: Data Link Layer}

\bd[Data Link Layer]
The \textbf{data link layer} is the second layer of the OSI model, and it is the protocol layer that transfers data
in the form of frames between nodes on a network segment across the physical layer. Its PDU is the frame.
\ed

The data link layer is the second layer and is the protocol layer that provides the functional and procedural means
to transfer data between directly connected nodes on a network segment across the physical layer. It also detects and
possibly corrects errors that may occur in the physical layer. The data link layer is concerned with local delivery
of frames between nodes on the same level of the network. Data-link frames, as these protocol data units are called,
do not cross the boundaries of a LAN. \v

IEEE 802 divides the data link layer into two sub-layers.

\bd[Medium Access Control (MAC) Sublayer]
The \textbf{Medium Access Control (MAC) sublayer} is the sublayer that controls the hardware which is responsible for
controlling how devices in a network gain access to a medium and permission to transmit data.
\ed

MAC sublayer provides flow control and multiplexing for the transmission medium. When sending data to another device
on the network, the MAC sublayer encapsulates higher-level frames into frames appropriate for the transmission
medium, adds a frame check sequence to identify transmission errors, and then forwards the data to the physical layer
as soon as the appropriate channel access method permits it. Additionally, the MAC is also responsible for
compensating for collisions by initiating retransmission if a jam signal is detected. When receiving data from the
physical layer, the MAC block ensures data integrity by verifying the sender's frame check sequences, and strips off
the sender's preamble and padding before passing the data up to the higher layers.

\bd[Logical Link Control (LLC) Sublayer]
The \textbf{Logical link control (LLC) sublayer} is the upper sublayer of the data link layer, which acts as an
interface between the MAC sublayer and the network layer.
\ed

LLC sublayer is responsible for identifying and encapsulating network layer protocols, and controls error checking
and frame synchronization. It provides multiplexing mechanisms that make it possible for several network protocols to
coexist within a multipoint network and to be transported over the same network medium. It can also provide
node-to-node flow control and error management. \v

There is a great number of protocols that belong to the data link layer such as: Address Resolution Protocol (ARP),
AppleTalk Link Access Protocol (ALAP), AppleTalk Link Access Protocol (LLAP), Asynchronous Transfer Mode (ATM),
BitTorrent, Carrier Sense Multiple Access with Collision Detection (CSMA/CD), Carrier Sense Multiple Access with
Collision Avoidance (CSMA/CA), Controller Area Network (CAN), Controller Area Network with Flexible Data-Rate (CAN
FD), Distributed Queue Dual Bus (DQDB), Ethernet, Fiber Distributed Data Interface (FDDI), Frame Relay, High-Level
Data Link Control (HDLC), IEEE 802.11, IEEE 802.15.4, IEEE 802.16, IEEE 1394, Inter-Switch Link (ISL), Link Access
Procedure, Balanced (LAPB), Link Access Procedure, D channel (LAPD), Link Access Procedure for Frame Relay (LAPF),
Link Access Procedure, Modem (LAPM), Link Control Protocol (LCP), LocalTalk, Media Access Control (MAC), Multiprotocol
Label Switching (MPLS), Point-to-Point Protocol (PPP), Point-to-Point Protocol over Ethernet (PPPoE), Serial Line
Internet Protocol (SLIP).

\subsection{Layer 3: Network Layer}

\bd[Network Layer]
The \textbf{network layer} is the third layer of the OSI model, and it is responsible for transferring variable length
data sequences from a source on one network to a destination on a different network, including routing through
intermediate nodes. Its PDU is the packet.
\ed

The network layer is the third layer of the OSI model, and it provides the functional and procedural means of
transferring packets from one node to another connected in different networks. It is also responsible for the routing
of packets, and the fragmentation and reassembly of data before it is passed to the transport layer. If the message
is too large to be transmitted from one node to another on the data link layer between those nodes, the network may
implement message delivery by splitting the message into several fragments at one node, sending the fragments
independently, and reassembling the fragments at another node. It may, but does not need to, report delivery errors. The
network layer responds to requests from the transport layer and issues requests to the data link layer.
\v

A number of protocols belong to the network layer, including routing protocols, multicast group management,
network-layer information and error, and network-layer address assignment. Some of the most important protocols of the
network layer include: Connectionless-mode Network Service (CLNS), Datagram Delivery Protocol (DDP), Exterior Gateway
Protocol (EGP), Enhanced Interior Gateway Routing Protocol (EIGRP), Internet Control Message Protocol (ICMP),
Internet Group Management Protocol (IGMP), Internet Protocol Security (IPsec), Internet Protocol (IPv4/IPv6),
Internetwork Packet Exchange (IPX), Low Latency Anonymous Routing Protocol (LLARP), Open Shortest Path First (OSPF),
Protocol Independent Multicast (PIM), Routing Information Protocol (RIP).

\subsection{Layer 4: Transport Layer}

\bd[Transport Layer]
The \textbf{transport layer} is the fourth layer of the OSI model, and it is the one responsible for the end-to-end
communication between the source and the destination. Its PDU is the segment.
\ed

The transport layer is the fourth layer of the OSI model, and it provides the functional and procedural means of
transferring variable-length data sequences from a source host to a destination host from one application to another
across a network, while maintaining the quality of service functions by ensuring that messages are delivered error-free,
in sequence, and with no losses or duplications. It is the layer where the actual transport of data happens. It also
controls the reliability of a given link between a source and destination host through flow control, end-to-end error
control and recovery, and acknowledgments of sequence and existence. It is responsible for the segmentation of data,
and the reassembly of data before it is passed to the session layer. \v

Some of the most important protocols of the transport layer include: TCP, UDP, SCTP, DCCP, SPX, ATP, IL, RDP, and
RUDP\@.

\subsection{Layer 5: Session Layer}

\bd[Session Layer]
The \textbf{session layer} is the fifth layer of the OSI model, and it is the one responsible for establishing,
managing, synchronizing, and terminating semi-permanent interactive information exchange among communicating devices
called sessions. Its PDU is data.
\ed

The session layer is the fifth layer of the OSI model, and it responds to service requests from the presentation
layer and issues service requests to the transport layer. Its main functionality is to provide the mechanism for
opening, closing, synchornising, and managing sessions among end-user application processes, i.e., a semi-permanent
dialogues. In addition, it establishes, manages and terminates the connections between local and remote applications.
The session layer also provides for duplex, half-duplex, or simplex operation, and establishes procedures for
checkpointing, suspending, restarting, and terminating a session between two related streams of data, such as an audio
and a video stream in a web-conferencing application. In case of a connection loss this protocol may try to recover the
connection. If a connection is not used for a long period, the session layer may close it and re-open it. \v

Some of the most important protocols of the session layer include: ADSP, ASP, H.245, ISO-SP, iSNS, L2F, L2TP, NetBIOS,
PAP, PPTP, RPC, RTCP, SMPP, SCP, SOCKS, ZIP, and SDP\@.

\subsection{Layer 6: Presentation Layer}

\bd[Presentation Layer]
The \textbf{presentation layer} is the sixth layer of the OSI model, and it is the one responsible for the
transformation of data into the form that the application layer can accept. Its PDU is data.
\ed

The presentation layer is the sixth layer of the OSI model, and it establishes data formatting and data translation
into a format specified by the application layer during the encapsulation of outgoing messages while being passed
down the protocol stack, and possibly reversed during the deencapsulation of incoming messages when being passed up
the protocol stack. In more simple words, the presentation layer ensures the information that the application layer
of one system sends out is readable by the application layer of another system. Among others, the presentation layer
handles protocol conversion, data encryption, data decryption, data compression, data decompression, incompatibility
of data representation between operating systems, and graphic commands. The presentation layer transforms data into
the form that the application layer accepts, to be sent across a network. \v

In many widely used applications and protocols no distinction is actually made between the presentation and
application layers (which we will see next). For this reason, presentation layer tends to have no specific protocols
associated with it, and only provides a service to the application layer. However, many protocols generally regarded
as application layer protocols, have presentation layer aspects.

\subsection{Layer 7: Application Layer}

\bd[Application Layer]
The \textbf{application layer} is the seventh and topmost layer of the OSI model, and it is the only layer that
directly interacts with data from the user and enables them to access the network resources. Its PDU is data.
\ed

The application layer is the topmost layer of the OSI model, and it is the one closest to the end user, directly
interacting with data coming from them, and data coming from the actual software application\footnote{The application
layer is not the application itself, but rather the means by which the application communicates with the network.
Actual applications fall outside the scope of the OSI model unless they are directly integrated into the application
layer.}. In more simple words, the application layer provides network services directly to the end-user. It differs
from the other layers in that it does not provide services to any other OSI layers, but rather to applications
outside the OSI model. It is responsible for identifying and establishing the availability of the intended
communication partner and determining whether sufficient resources for the intended communication exist. It is also
responsible for determining the identity and availability of communication partners for an application with data to
transmit. \v

Application-layer functions typically include file sharing, message handling, and database access, through the most
common protocols at the application layer.
\bit
\item \textbf{Hypertext Transfer Protocol (HTTP)} is an application-layer protocol for transmitting hypermedia
documents, such as HTML. It was designed for communication between web browsers and web servers, but it can also be
used for other purposes. HTTP follows a classical client-server model, with a client opening a connection to make a
request, then waiting until it receives a response. HTTP is a stateless protocol, meaning that the server does not
keep any data (state) between two requests.
\item \textbf{File Transfer Protocol (FTP)} is a standard network protocol used for the transfer of computer files
between a client and server on a computer network. FTP is built on a client-server model architecture using separate
control and data connections between the client and the server. FTP users may authenticate themselves with a clear-text
sign-in protocol, normally in the form of a username and password, but can connect anonymously if the server is
configured to allow it.
\item \textbf{SSH File Transfer Protocol (SFTP)} is a network protocol that provides file access, file transfer, and
file management over any reliable data stream. It was designed by the Internet Engineering Task Force (IETF) as an
extension of the Secure Shell protocol (SSH) version 2.0 to provide secure file transfer capability, but is also
intended to be usable with other protocols as well. The SSH file transfer protocol is not related to the FTP protocol,
which does not provide for file transfers via SSH\@.
\item \textbf{Trivia File Transfer Protocol (TFTP)} is a simple protocol that provides basic file transfer function
with no user authentication. TFTP is intended for applications that do not need the sophisticated interactions that
File Transfer Protocol (FTP) provides. One of its primary uses is in the early stages of nodes booting from a local
area network. TFTP has been used for this application because it is very simple to implement.
\item \textbf{Server Message Block (SMB)} is a network file sharing protocol used for file sharing among different
operating systems. SMB is a client-server interaction protocol in a client-server model of networking. It provides the
Read and Write operations on network devices.
\item \textbf{Simple Mail Transfer Protocol (SMTP)} is an Internet standard communication protocol for electronic mail
transmission. Mail servers and other message transfer agents use SMTP to send and receive mail messages. User-level
email clients typically use SMTP only for sending messages to a mail server for relaying, and typically submit outgoing
email to the mail server
\eit

\section{Network Devices (Hardware)}

\bd[Network Devices (Hardware)]
\textbf{Network devices} (or \textbf{hardware}) are physical devices that are required for communication and interaction
between hardware on a computer network.
\ed

Network devices or hardware are the physical representation of nodes. There are many network devices. Let's take a
look at some of the most important ones.

\bd[Cables]
\textbf{Cables} are networking devices used to connect one network device to other network devices.
\ed

Cables can be made either of copper or fiber. Copper cables use electric signals, are cheaper and used for short
distances. Due to their nature they can be affected by electromagnetic interference. Fiber cables are made of glass,
use light signals, are more expensive and used for longer distances. Due to their nature they are not affected by
electromagnetic interference.

\bd[Hub]
A \textbf{hub} is the most basic hardware that connects multiple network devices together and acts as a repeater in
that it amplifies signals that deteriorate after traveling long distances over connecting cables.
\ed

\fig{hub}{0.2}

A hub is the simplest in the family of network connecting devices because it connects LAN components with identical
protocols. A hub can be used with both digital and analog data, provided its settings have been configured to prepare
for the formatting of the incoming data. \v

Hubs do not perform packet filtering or addressing functions; they just send data packets to all connected devices.
Hubs operate at the Physical layer of the Open Systems Interconnection (OSI) model. (More on that later) \v

Hubs are quite old technology, and they are not used anymore. Their main problem is something called collision domain.

\bd[Collision Domain]
A \textbf{collision domain} is a network segment connected by a shared medium or through repeaters where simultaneous
data transmissions collide with one another.
\ed

A network collision occurs when more than one device attempts to send a packet on a network segment at the same time.
Members of a collision domain may be involved in collisions with one another. Devices outside the collision domain do
not have collisions with those inside.

\bd[Bridge]
A \textbf{bridge} is a computer networking device that creates a single,aggregate network from multiple communication
networks or network segments.
\ed

\fig{bridge}{0.6}

Bridges are used to connect two or more hosts or network segments together.The basic role of bridges in network
architecture is storing and forwarding frames between the different segments that the bridge connects. They use
hardware Media Access Control (MAC) addresses for transferring frames. By looking at the MAC address of the devices
connected to each segment, bridges can forward the data or block it from crossing. Bridges can also be used to
connect two physical LANs into a larger logical LAN. \v

Bridges work only at the Physical and Data Link layers of the OSI model. Bridges are used to divide larger networks
into smaller sections by sitting between two physical network segments and managing the flow of data between the two. \v

Bridges are like hubs in many respects, including the fact that they connect LAN components with identical protocols.
However, bridges filter incoming data packets, known as frames, for addresses before they are forwarded. As it
filters the data packets, the bridge makes no modifications to the format or content of the incoming data. The bridge
filters and forwards frames on the network with the help of a dynamic bridge table. The bridge table, which is
initially empty, maintains the LAN addresses for each computer in the LAN and the addresses of each bridge interface
that connects the LAN to other LANs. Bridges, like hubs, can be either simple or multiple port. \v

Bridges have mostly fallen out of favor in recent years and have been replaced by switches, which offer more
functionality. In fact, switches are sometimes referred to as ``multiport bridges'' because of how they operate.

\bd[Switch]
A \textbf{switch} is hardware that connects devices on a computer network by using packet switching to receive and
forward data to the destination device.
\ed

\fig{switch}{0.65}

Switches generally have a more intelligent role than hubs. A switch is a multiport device that improves network
efficiency. The switch maintains limited routing information about nodes in the internal network, and it allows
connections to systems like hubs. Strands of LANs are usually connected using switches. Generally, switches can read
the hardware addresses of incoming packets to transmit them to the appropriate destination. \v

Using switches improves network efficiency over hubs because of the virtual circuit capability. Switches also improve
network security because the virtual circuits are more difficult to examine with network monitors. You can think of a
switch as a device that has some of the best capabilities of hubs. A switch can work at either the Data Link layer or
the Network layer of the OSI model. A multilayer switch is one that can operate at both layers, which means that it
can operate as both a switch and a router which we will define right next. \v

Hubs, bridges and switches are the primary devices used to connect network devices on a single network, usually a LAN\@.
However, we often want to send or receive data between independent layers. That's where routers are coming in use.

\bd[Router]
A \textbf{router} is a networking device that forwards data packets between computer networks.
\ed

\fig{router}{0.34}

Routers help transmit packets to their destinations by charting a path through the sea of interconnected networking
devices using different network topologies. Routers are intelligent devices, and they store information about the
networks they're connected to. Most routers can be configured to operate as packet-filtering firewalls and use access
control lists. Routers are also used to translate from LAN framing to WAN framing. This is needed because LANs and
WANs use different network protocols. \v

Router are also used to divide internal networks into two or more subnetworks. Routers can also be connected
internally to other routers, creating zones that operate independently. Routers establish communication by
maintaining tables about destinations and local connections. A router contains information about the systems
connected to it and where to send requests if the destination isn't known. \v

Routers are your first line of defense, and they must be configured to pass only traffic that is authorized by
network administrators. The routes themselves can be configured as static or dynamic. If they are static, they can
only be configured manually and stay that way until changed. If they are dynamic, they learn of other routers around
them and use information about those routers to build their routing tables. Routers work at the Network layer of the
OSI model.

\bd[Gateway]
A \textbf{gateway} is a piece of hardware used in that allows data to flow from one discrete network to another.
\ed

\fig{gateway}{0.2}

Gateways are distinct from routers or switches in that they communicate using more than one protocol to connect
multiple networks and can operate at any of the seven layers of OSI, although they normally work at the Transport and
Session layers. At the Transport layer and above, there are numerous protocols and standards from different vendors;
gateways are used to deal with them. Gateways provide translation between networking technologies such as OSI and
TCP/IP. Because of this, gateways connect two or more autonomous networks, each with its own routing algorithms,
protocols, topology, domain name service, and network administration procedures and policies. \v

Gateways perform all the functions of routers and more. In fact, a router with added translation functionality is a
gateway. The function that does the translation between different network technologies is called a protocol converter. \v

The term gateway can also loosely refer to a computer or computer program configured to perform the tasks of a
gateway, such as a default gateway or router, and in the case of HTTP, gateway is also often used as a synonym for
reverse proxy.

\bd[Modem]
A \textbf{modulator-demodulator} (or simply \textbf{modem}), is a hardware device that converts data from a digital
format into a format suitable for an analog device such as computers.
\ed

\fig{modem}{0.24}

A modem transmits data by modulating one or more carrier wave signals to encode digital information, while the
receiver demodulates the signal to recreate the original digital information. The goal is to produce an electrical
signal that can be transmitted easily and decoded reliably. Modems can be used with almost any means of transmitting
analog signals, from light-emitting diodes to radio. Modems work on both the Physical and Data Link layers. \v

The Institute of Electrical and Electronics Engineers (IEEE) is a professional association for electronic and
electrical engineering formed in 1963 with its corporate office in New York City. IEEE 802 is a family of IEEE's
standards for local area networks (LAN), personal area network (PAN), and metropolitan area networks (MAN). (The
number 802 has no significance, it was simply the next number in the sequence that the IEEE used for standards
projects). The IEEE 802 family of standards has had twenty-four members, numbered 802.1 through 802.24, however, not
all of these are currently active. Two of the most widely used, and still active, of these standards are the IEEE 802.3
for the Ethernet family, and the IEEE 802.11 for Wireless LAN (WLAN) \& Mesh (Wi-Fi certification). \v

IEEE 802.3 is a working group and a collection of Institute of Electrical and Electronics Engineers (IEEE) standards
produced by the working group defining the physical layer and data link layer's media access control (MAC) of wired
Ethernet. This is generally a local area network (LAN) technology with some wide area network (WAN) applications.
Physical connections are made between nodes and/or infrastructure devices (hubs, switches, routers) by various types
of copper or fiber cable. \v

The services and protocols specified in IEEE 802 map to the lower two layers (data link and physical) of the
seven-layer Open Systems Interconnection (OSI) networking reference model. More specifically IEEE 802 divides the OSI
data link layer into two sub-layers: logical link control (LLC) and medium access control (MAC).