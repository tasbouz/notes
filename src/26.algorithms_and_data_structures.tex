%! suppress = EnDash
% TODO: Finish up this chapter
\section{Basics of Algorithms and Data Structures}

\bd[Data]
\textbf{Data} is a collection of discrete or continuous values.
\ed

If data are structured in a systematic way then through this structure they become meaningful, and turn from data to
information. Information is useful cause it provides context for data and enables decision-making process. For this
reason, a systematic way to organize data, so they can be used efficiently, was invented. This systematic way is
called a ``data structure''.

\bd[Data Structure (DS)]
A \textbf{data structure} (\textbf{DS}) is a way of structuring data so that it can be used effectively.
\ed

More precisely, a DS is a collection of data values, the relationships among them, and the functions or operations that
can be applied to the data, i.e.\ it is an algebraic structure about data. DSs are essential ingredients in creating 
fast and powerful algorithms, they help to manage and organize data, and they make code cleaner and easier to understand.
\v

Before we start talking about DSs, we need to introduce the concepts of ``data types'' and most importantly that of
``abstract data types''. Let's start with the first one.

\bd[Data Type / Type]
A \textbf{data type} (or simply \textbf{type}) is a collection or grouping of data values, usually specified by a set
of possible values, a set of allowed operations on these values, and/or a representation of these values as machine
types.
\ed

In simple words, a data type is an attribute of data, as defined by the domain of values data can take, or the
operations that can be performed on the data. A data type specification constrains the possible values that an
expression, such as a variable or a function call, might take. On literal data, it tells the compiler or interpreter
how the programmer intends to use the data. \v

Most programming languages support basic data types called ``primitive data types''.

\bd[Primitive Data Type]
\textbf{Primitive data types} are types that are built-in or basic to a language implementation.
\ed

Classic basic primitive data types that are generally supported more or less directly by computer hardware and any
programming language are:
\bit
\item \textbf{Character}: A unit of information that roughly corresponds to a symbol, such as in an alphabet, in the
written form of a natural language.
\item \textbf{Integer}: A range of mathematical integers that are commonly represented in a computer as a group of bits.
The size of the grouping varies so the set of integer sizes available varies between different types of computers.
\item \textbf{Floating-point}: An arithmetic unit using formulaic representation of real numbers as an approximation
to support a trade-off between range and precision.
\item \textbf{Fixed-point}: A method of representing fractional numbers by storing a fixed number of digits of their
fractional part.
\item \textbf{Boolean}: A data type that has one of two possible values, usually denoted ``true'' and ``false'', which
is intended to represent the two truth values of logic and Boolean algebra.
\item \textbf{Reference}: A value that enables a program to indirectly access a particular data, such as a variable's
value or a record, in the computer's memory or in some other storage device.
\eit

Opposite to primitive data types, we also have the so called ``composite data types``.

\bd[Composite Data Type]
A \textbf{composite data type} is any data type which can be constructed in a program using the programming language's
primitive data types or other composite data types.
\ed

Moving on, we can now introduce the concept of ``abstract data types''.

\bd[Abstract Data Type (ADT)]
An \textbf{abstract data type} (\textbf{ADT}) is a mathematical model for data types which is defined by its behavior
from the point of view of a user of the data and more specifically in terms of possible values, possible operations on
data of this type, and the behavior of these operations.
\ed

In simple words, an ADT is an abstraction of a DS which provides only the interface to which a DS must adhere to. The
interface does not give any specific details about how something should be implemented or in what programming
language. ADTs contrast with DSs, which are concrete representations of data, and are the point of view of an
implementer, not a user. \v

DSs are used to implements ADTs. There are multiple ways (i.e.\ multiple DSs) to implement a specific ADT. In simple
words, ADTs tell us what is to be done and DSs tell us how to do it. In reality, different implementations of ADTs
(a.k.a.\ different DSs) are compared for time and space efficiency. The one best suited depends on the requirements of
the project.

\be
For an analogy, consider a car. A car is a concrete object, with a specific implementation. It has a certain weight,
a certain color, a certain number of doors, and so on. A car is an instance of the abstract concept of a vehicle. The
abstract concept of a vehicle is an ADT, and the concrete car is an instance of that ADT, i.e.\ a DS\@.
\ee

DSs are divided into categories based on their characteristics. One basic division is between ``linear''
and ``non-linear'' DSs.

\bd[Linear Data Structure]
A \textbf{linear data structures} has data elements arranged in linear (sequential) order and each member element is
connected to its previous and next element, except from the first and the last.
\ed

\bd[Non-Linear Data Structure]
A \textbf{non-linear data structures} is a DS in which data items are not arranged in a sequence.
\ed

Another important division is between ``static'' and ``dynamic'' DSs.

\bd[Static Data Structures]
A \textbf{static data structure} is a DS where the data in memory are fixed in size and allocated during compile time.
\ed

In a static DS the maximum size is needed to be known in advance, as memory cannot be reallocated at a later point. Such
DSs are easy to implement as computer memory is also sequential. Static DSs have the advantage of fast access, but they
are slower in insertion and deletion.

\bd[Dynamic Data Structures]
A \textbf{dynamic data structure} is a DS where the data in memory are allocated in real time.
\ed

A dynamic DS refers to a collection of data in memory that has the flexibility to grow or shrink in size, enabling a
programmer to control exactly how much memory is utilised. Dynamic DSs have the advantage of fast insertion and
deletion, but they are much slower in access. \v

Efficiency of data structures is always measured in terms of time and (memory) space. An ideal data structure would
be the one that takes the least possible time for all its operations and consumes the least memory space.